%GregorioTeX file.
%Copyright (C) 2007 Elie Roux <elie.roux@enst-bretagne.fr>
%
%This program is free software: you can redistribute it and/or modify
%it under the terms of the GNU General Public License as published by
%the Free Software Foundation, either version 3 of the License, or
%(at your option) any later version.
%
%This program is distributed in the hope that it will be useful,
%but WITHOUT ANY WARRANTY; without even the implied warranty of
%MERCHANTABILITY or FITNESS FOR A PARTICULAR PURPOSE.  See the
%GNU General Public License for more details.
%
%You should have received a copy of the GNU General Public License
%along with this program.  If not, see <http://www.gnu.org/licenses/>.

%%%%%%%%%%%%%%%%%%%%%%%%%%%%%%%%%
%% macros for the font dimensions
%%%%%%%%%%%%%%%%%%%%%%%%%%%%%%%%%


%%%%%%%%%%%%%%%%%%%%%%%%%%%%%%%%%%%%%
%% macros for the typesetting of text
%%%%%%%%%%%%%%%%%%%%%%%%%%%%%%%%%%%%%

% \textlower is the height of the separation between the bottom line (which is invisible : for the notes which are very low) and the bottom of the text
\newdimen\textlower \textlower=6pt

%% macro that sets \temp to the width of its argument

\newbox\Tempwidth
\newdimen\tempwidth

\def\widthof#1{%
\setbox\Tempwidth=\hbox{#1}%
\tempwidth=\wd\Tempwidth%
\relax%
}


%% macro that typesets the text of the syllable, and sets \textaligncenter to the middle of the middle letters, it is needed because we align the note (often the middle of the note) with the middle of the middle letters
%% warning : textaligncenter is the width from the beginning of the letters to the middle of the middle letters

\newdimen\textaligncenter

\def\findtextaligncenter#1#2#3{%
\widthof{\textfont #1#2}%
\textaligncenter=\the\tempwidth %
\widthof{\textfont #2}%
\divide\tempwidth by 2 %
\advance\textaligncenter by -\the\tempwidth%
\relax%
}

% definition of the dash that we will use after the text if necessary
\def\gregoriotextdash{-}
\newdimen\gregoriotextdashwidth
\widthof{\gregoriotextdash}
\gregoriotextdashwidth=\tempwidth

%%%%%%%%%%%%%%%%%%%%%%%%%%%%%%%%%%%%%%%%%%%%
%% macros for the typesetting of the initial
%%%%%%%%%%%%%%%%%%%%%%%%%%%%%%%%%%%%%%%%%%%%

%% box containing the initial, and dimen containing its width (and the width of the space after)
\newbox\Initial
\newdimen\initialwidth
\initialwidth=0pt

\def\initial#1{%
\setbox\Initial=\hbox{\fontofinitial #1}%
\global\initialwidth=\the\wd\Initial%
\lower\textlower\copy\Initial%
\hskip\afterinitialshift %
\advance\initialwidth by \afterinitialshift%
\relax%
}

%%%%%%%%%%%%%%%%%%%%%%%%%%%%%%%%%%%%%%%%%%%%%%%%%%%%%%%%%%
%% macros for the typesetting the things above the initial
%%%%%%%%%%%%%%%%%%%%%%%%%%%%%%%%%%%%%%%%%%%%%%%%%%%%%%%%%%

\def\writemode#1{%
\relax%
}

\def\gregorianmode#1{%
\ifcase#1%
\or%
\writemode{I.}%
\or%
\writemode{II.}%
\or%
\writemode{III.}%
\or%
\writemode{IV.}%
\or%
\writemode{V.}%
\or%
\writemode{VI.}%
\or%
\writemode{VII.}%
\or%
\writemode{VIII.}%
\fi%
\relax%
}

\def\scorereference#1{%
\relax%
}

\def\beginscore{%
\drawlines%
\relax%
}

%%%%%%%%%%%%%%%%%%%%%%%%%%%%%%%%%%%%%%%%%%%%
%% macros for the typesetting of the lines
%%%%%%%%%%%%%%%%%%%%%%%%%%%%%%%%%%%%%%%%%%%%

% factor is the factor with which you open you font (the number after the at)
\newcount\factor
\factor=17

% maybe a good idea, but hard to exploit...
%\def\gscale#1{%
%\global\multiply#1 by \the\factor %
%\global\divide#1 by 10 %
%\relax%
%}

% \stafflineheight is the height of a staff line
\newdimen\stafflineheight
\stafflineheight=1500 sp 
\multiply\stafflineheight by \the\factor 
% \interstafflinespace is the space between two lines of staff
%\newdimen\interstafflinespace \interstafflinespace=0.315\the\factor em
\newdimen\interstafflinespace
\interstafflinespace=30000 sp 
\multiply\interstafflinespace by \the\factor 
% \interstafflinespace is the width of a line
\newdimen\linewidth \linewidth=\hsize 
% \stafflinewidth is the width of a line of staff, this can vary, for example at the first line
\newdimen\stafflinewidth \stafflinewidth=\linewidth 
% \staffheight is the total height of the staff : that is to say the four written lines plus two invisible lines plus the space to type one note above the top line and one note beneath the bottom line
\newdimen\staffheight \staffheight=6\stafflineheight 
\advance\staffheight by 7\interstafflinespace

%% macro that draws the stafflines on the first line, it is different from others due to the initial that can take some place, without lines
\def\drawfirstlines{%
\advance\stafflinewidth by -\initialwidth%
\initialwidth=0pt%
\hbox to0pt{%
\vbox{%
\kern\the\interstafflinespace %
\kern\the\stafflineheight %
\kern\the\interstafflinespace %
\hrule height \stafflineheight width \stafflinewidth %
\kern\the\interstafflinespace %
\hrule height \stafflineheight width \stafflinewidth %
\kern\the\interstafflinespace %
\hrule height \stafflineheight width \stafflinewidth %
\kern\the\interstafflinespace %
\hrule height \stafflineheight width \stafflinewidth %
\kern\the\interstafflinespace %
\kern\the\stafflineheight %
\kern\the\interstafflinespace %
}%
\hss%
}%
\stafflinewidth=\linewidth%
\relax%
}

%% box containing the stafflines for other lines than the first
\newbox\Lines
\setbox\Lines=\hbox to0pt{%
\vbox{%
\kern \the\interstafflinespace %
\kern\the\stafflineheight %
\kern \the\interstafflinespace %
\hrule height \stafflineheight width \stafflinewidth %
\kern\the\interstafflinespace %
\hrule height \stafflineheight width \stafflinewidth %
\kern\the\interstafflinespace %
\hrule height \stafflineheight width \stafflinewidth %
\kern\the\interstafflinespace %
\hrule height \stafflineheight width \stafflinewidth %
\kern \the\interstafflinespace %
\kern\the\stafflineheight %
\kern \the\interstafflinespace %
}%
\hss%
\relax%
}

%% macro that draws the lines : starts by the first and then draws the lines of every line.
%% has to be called before drawing the key, after drawing the initial
\def\drawlines{%
\drawfirstlines %
%%localeleftbox is a primitive of Omega, it draws the same box at the beginning of new lines (here after the first)
\localleftbox{%
\copy\Lines %
}%
\relax%
}

%%%%%%%%%%%%%%%%%%%%%%%%%%%%%%%%%%%%%%%%%%%%%%%%%%%%%%%%%%%%%%%%%%%%%%%%%%%%%%%%%%
%% macros for the typesetting of the clefs of the beginning of lines and custos
%%%%%%%%%%%%%%%%%%%%%%%%%%%%%%%%%%%%%%%%%%%%%%%%%%%%%%%%%%%%%%%%%%%%%%%%%%%%%%%%%%

%% marcro to define the clef that will appear at the beginning of the lines
% the first argument is the type : f or c, and the second is the height

\def\setlinesclef#1#2{%
\localleftbox{%
\copy\Lines% draws the lines
\unkern %
\typekey{#1}{#2}{0}%
}%
\relax%
}

% macro that typesets the key
% arguments are : 
%% #1: the type of the key : c or f
%% #2: the line of the key (1 is the lowest)
%% #3: if we must use small key characters (inside a line) or not 0 if not inside, 1 if inside
\def\typekey#1#2#3{%
\ifcase#2 %
\or%
\calculateglyphraisevalue{c}{0}%
\or%
\calculateglyphraisevalue{e}{0}%
\or%
\calculateglyphraisevalue{g}{0}%
\or%
\calculateglyphraisevalue{i}{0}%
\fi%
\ifx c#1% we check if it is a c key
\ifcase#3%
\raise\glyphraisevalue\hbox{\gregorianfont \char 1\hskip\spaceafterlineclef}%
\or%
\raise\glyphraisevalue\hbox{\gregorianfont \char 3\hskip\spaceafterlineclef}%
\fi%
\else % we consider that it is a f key
\ifcase#3%
\raise\glyphraisevalue\hbox{\gregorianfont \char 2\hskip\spaceafterlineclef}%
\or%
\raise\glyphraisevalue\hbox{\gregorianfont \char 4\hskip\spaceafterlineclef}%
\fi%
\fi%
\relax%
}

% macro that writes the initial key, and sets the next keys to the same value

\def\setinitialclef#1#2{%
\typekey{#1}{#2}{0}%
\setlinesclef{#1}{#2}%
\relax%
}

% macro called when the key changes

\def\changeclef#1#2{%
\clefchangespace = 16500 sp plus 1650 sp minus 16500 sp%
\multiply\clefchangespace by \the\factor %
\hskip\clefchangespace %
\typekey{#1}{#2}{1}%
\hskip\clefchangespace %
\setlinesclef{#1}{#2}%
\relax%
}

% macro called when the key changes inside a syllable

\def\inchangeclef#1#2{%
\clefchangespace = 16500 sp plus 1650 sp minus 16500 sp%
\multiply\clefchangespace by \factor %
\hskip\clefchangespace %
\typekey{#1}{#2}{1}%
\hskip\clefchangespace %
\setlinesclef{#1}{#2}%
\relax%
}

% the argument is the height

\def\setcusto#1{%
\calculateglyphraisevalue{#1}{0}%
\localrightbox{%
\raise \glyphraisevalue%
\hbox{%
% we type a hskip and the we type the custo
\hskip\spacebeforecusto %
\gregorianfont %
\ifx a#1%
\char 62%
\fi%
\ifx b#1%
\char 60%
\fi%
\ifx c#1%
\char 61%
\fi%
\ifx d#1%
\char 60%
\fi%
\ifx e#1%
\char 61%
\fi%
\ifx f#1%
\char 60%
\fi%
\ifx g#1%
\char 61%
\fi%
\ifx h#1%
\char 60%
\fi%
\ifx i#1%
\char 61%
\fi%
\ifx j#1%
\char 63%
\fi%
\ifx k#1%
\char 64%
\fi%
\ifx l#1%
\char 63%
\fi%
\ifx m#1%
\char 65%
\fi%
}%
}%
\relax%
}

\def\removecusto{%
\localrightbox{}%
\relax%
}

%%%%%%%%%%%%%%%%%%%%%%%%%%%%%%%%%%%%%%%%%%%%%%%%%%%%%
%% macros for the typesetting of the different glyphs
%%%%%%%%%%%%%%%%%%%%%%%%%%%%%%%%%%%%%%%%%%%%%%%%%%%%%

% \glyphraisevalue is the value of which we must raise one glyph (that will vary with every glyph)
\newdimen\glyphraisevalue 

% \addedraisevalue is for the vertical episema and the puncta
\newdimen\addedraisevalue

% a temporary count
\newcount\tempcount

% a boolean : 0 if the note is not a line, else 1
\newcount\isonaline

% a very useful macro : it determines the good height of a glyph : the argument is the "number" where the glyph should be : 4 for the first line, 6 for the second, etc.
% the second argument is for the cases of signs: for example if the note is on a line, the punctummora will be above, and the auctus duplex beneath. the possible values are:
%% 0: no modification
%% 1: puts the value on the interline just above if it is on a line
%% 2: puts the value on the interline just beneath if it is on a line
%% 3: case of the vertical episemus, which is not placed at the same place if the corresponding note is on a line or not
%% 4: case of the punctum mora, for the same reason
\def\calculateglyphraisevalue#1#2{%
\global\isonaline=\number 0%
\ifx a#1%
\global\tempcount=\number 1%
\fi%
\ifx b#1%
\global\tempcount=\number 2%
\global\isonaline=1 %
\fi%
\ifx c#1%
\global\tempcount=\number 3%
\fi%
\ifx d#1%
\global\tempcount=\number 4%
\global\isonaline=1 %
\fi%
\ifx e#1%
\global\tempcount=\number 5%
\fi%
\ifx f#1%
\global\tempcount=\number 6%
\global\isonaline=1 %
\fi%
\ifx g#1%
\global\tempcount=\number 7%
\fi%
\ifx h#1%
\global\tempcount=\number 8 %
\global\isonaline=1 %
\fi%
\ifx i#1%
\global\tempcount=\number 9%
\fi%
\ifx j#1%
\global\tempcount=\number 10%
\global\isonaline=1 %
\fi%
\ifx k#1%
\global\tempcount=\number 11%
\fi%
\ifx l#1%
\global\tempcount=\number 12%
\global\isonaline=1 %
\fi%
\ifx m#1%
\global\tempcount=\number 13%
\fi%
% if the note is on a line, we change its height
\ifcase\isonaline%
\or%
\ifcase#2 %
\or% 1
\advance\tempcount by 1%
\or% 2
\advance\tempcount by -1%
\or% 3
\advance\tempcount by -1%
\or% 4
\advance\tempcount by 1%
\fi%
\fi%
\advance\tempcount by -7 %
\glyphraisevalue = 15750 sp %
\multiply\glyphraisevalue by \the\factor %
\multiply\glyphraisevalue by \the\tempcount %
\ifcase#2 % 
\or\or\or%3: if it is a vertical episemus on a line, we shift it a bit higher, so that it's more beautiful
\ifnum\isonaline=1%
\addedraisevalue=5000 sp%
\multiply\addedraisevalue by \the\factor %
\advance\glyphraisevalue by \addedraisevalue %
\fi
\or% 4: if it is a punctum mora on a line, we shift it a bit lower, for the same reason
\ifnum\isonaline=1%
\addedraisevalue=-4000 sp%
\multiply\addedraisevalue by \the\factor %
\advance\glyphraisevalue by \addedraisevalue %
\fi%
\fi%
\addedraisevalue=102400 sp%
\multiply\addedraisevalue by \the\factor %
\advance\glyphraisevalue \addedraisevalue % where the 0 must be
\global\tempcount=0%
\global\isonaline=0%
\relax%
}

% count that tells us if the current glyph is the first glyph or not. It it is the case, we determine
\newcount\firstglyph

\newcount\firstnote
\newcount\secondnote
\newcount\thirdnote
\newcount\fourthnote
\newcount\glyphnumber
\newcount\glyphnumbertemp
\newcount\interval
\newcount\notetemp

\def\assignnote#1#2{%
\ifcase#1%
\or%
\global\firstnote=#2%
\or%
\global\secondnote=#2%
\or%
\global\thirdnote=#2%
\or%
\global\fourthnote=#2%
\fi%
\relax%
}

\def\notetonumber#1#2{%
\ifx a#1%
\assignnote{#2}{1}%
\fi%
\ifx b#1%
\assignnote{#2}{2}%
\fi%
\ifx c#1%
\assignnote{#2}{3}%
\fi%
\ifx d#1%
\assignnote{#2}{4}%
\fi%
\ifx e#1%
\assignnote{#2}{5}%
\fi%
\ifx f#1%
\assignnote{#2}{6}%
\fi%
\ifx g#1%
\assignnote{#2}{7}%
\fi%
\ifx h#1%
\assignnote{#2}{8}%
\fi%
\ifx i#1%
\assignnote{#2}{9}%
\fi%
\ifx j#1%
\assignnote{#2}{10}%
\fi%
\ifx k#1%
\assignnote{#2}{11}%
\fi%
\ifx l#1%
\assignnote{#2}{12}%
\fi%
\ifx m#1%
\assignnote{#2}{13}%
\fi%
\relax%
}

% not working yet, do not use
\def\tor#1#2#3{%
\notetonumber{#1}{1}%
\notetonumber{#2}{2}%
\notetonumber{#3}{3}%
\interval=\secondnote%
\advance\interval by -\firstnote %
\multiply\interval by 25 %
\tempnote=\secondnote%
\advance\tempnote by -\thirdnote %
\advance\interval by \tempnote %
\advance\interval by 28672 %
\glyph{\char \the\interval}{#1}{#3}{0}%
\relax%
}

% the width of the last glyph, including the width of the text which is after
%\newdimen\lastglyphwidth
% the width of the text which is after the glyph
%\newdimen\additionalwidth

% macro to typeset the glyph. attributes are :
% #1: character : the character that it must call
% #2: height : the height it must be raised : can be negative (must be calculated by a preprocessor)
% #3: height of the next note : we define the custo with that
% #4: type : the type of glyph, to determine the aligncenter; can be :
%%%%% 0 : one-note glyph or more than two notes glyph except porrectus : here we must put the aligncenter in the middle of the first note
%%%%% 1 : two notes glyph (podatus is considered as a one-note glyph : here we put the aligncenter in the middle of the glyph
%%%%% 2 : porrectus : has a special align center
%%%%% 3 : initio-debilis : same as 1 but the first note is much smaller
\def\glyph#1#2#3#4{%
\calculateglyphraisevalue{#2}{0}%
\setbox\Tempwidth=\hbox{\gregorianfont #1}%
\tempwidth=\wd\Tempwidth%
\raise \glyphraisevalue%
\copy\Tempwidth%
\ifnum\the\endofscore=0 %
\setcusto{#3}%
\fi %
\ifnum\the\firstglyph=1% we check if it is the first glyph
\findnotesaligncenter{#4}%
\global\firstglyph=0%
\fi%
%\lastglyphwidth=\tempwidth%
\relax%
}

\newdimen\notesaligncenter

% we define the different alignments possible, of course they depend on the font
\def\findnotesaligncenter#1{%
\ifcase#1%
%case of punctum
\setbox\Tempwidth=\hbox{\gregorianfont \char 17}%
\global\notesaligncenter=\wd\Tempwidth%
\global\divide\notesaligncenter by 2%
\or%
%case of flexus and flexus deminutus
\global\notesaligncenter=\tempwidth%
\global\divide\notesaligncenter by 2%
\or%
%case of porrectus (we consider it to have the same alignment as punctum)
\setbox\Tempwidth=\hbox{\gregorianfont \char 17}%
\global\notesaligncenter=\wd\Tempwidth%
\global\divide\notesaligncenter by 2%
\or%
%case of a initio debilis
\setbox\Tempwidth=\hbox{\gregorianfont \char 13}%
\global\notesaligncenter=\wd\Tempwidth%
\global\divide\notesaligncenter by 2%
\or
\fi%
\relax%
}


% box that we will use to determine the width of the notes, to determine wether we typeset a - or not after the letters
\newbox\Syllablenotes
\def\syllablenotes#1{%
\setbox\Syllablenotes=\hbox{#1}%
\relax%
}


%%%%%%%%%%%%%%%%%%%%%%%%%%%%%%%%%%%%%%%%%%
%% macros for the typesetting of the signs
%%%%%%%%%%%%%%%%%%%%%%%%%%%%%%%%%%%%%%%%%%

% a function to typeset a punctum mora, the argument is the letter of the height of the punctum mora

\def\punctummora#1{%
\hskip\spacebeforesigns%
\calculateglyphraisevalue{#1}{4}%
\raise \glyphraisevalue \hbox{\gregorianfont \char 14}%
\relax%
}

% a function to typeset a augmentum duplex, the argument is the letter of the height of the augmentum duplex

\def\augmentumduplex#1{%
\hskip\spacebeforesigns%
\calculateglyphraisevalue{#1}{2}%
\raise \glyphraisevalue \hbox{\gregorianfont \char 15}%
\relax%
}

\newbox\Tempsign
\newdimen\tempsignwidth

%3: si on revient au début ou pas : 0 on revient au début
% a macro to help typesetting vertical episemus. The third argument is 0 when we go back to the beginning of the glyph. If it is 2, it means that we must go back first of width #1, and then forward of #2. If it is 1, it means that we only need to go back of #2. It is very hard to understand, sorry.
\def\vepisemusaux#1#2#3{%
\setbox\Tempsign=\hbox{\gregorianfont #2}%
\tempsignwidth=\wd\Tempsign%
\divide\tempsignwidth by 2 %
\ifcase#3%
\advance\tempwidth by -\tempsignwidth %
\or%
\tempwidth=\tempsignwidth %
\or%
\setbox\Tempsign=\hbox{\gregorianfont #1}%
\tempwidth=\wd\Tempsign %
\advance\tempwidth by -\tempsignwidth %
\fi%
% then we draw the sign
\setbox\Tempsign=\hbox{\gregorianfont \char 33}%
% we set tempwidth to half a punctum malus half the sign width, so that the centers are aligned
\tempsignwidth=\wd\Tempsign %
\divide\tempsignwidth by 2 %
\advance\tempwidth by \tempsignwidth %
\kern -\tempwidth%
\raise \glyphraisevalue \copy\Tempsign %
% and finally we go back to the end of the glyph, where we were first
\advance\tempwidth by -2\tempsignwidth %
\kern \tempwidth%
\relax%
}

% a function to typeset a vertical episemus. The firts argument is the letter of the height of the episemus (not the height of the note it corresponds to. This function must be called after a call to \glyph. The second argument is the type of glyph it was, more precisely the kind of space there is between the end (or in special cases the beginning) of the glyph and the place where we will typeset the episemus. The possible values are:
%% 0: the episemus will be typeset in the middle of the last note, which is a standard punctum (works with pes)
%% 1: the same, but the last note is a deminutus
%% 2: the episemus is typeset in the middle of the note before the last note, which is a standard punctum
%% 3: idem, but the note is the note preceding a deminutus
%% 4: the note before the note before the last note (for torculus resupinus)
%% 5: idem, but when the two last notes are a deminutus
%% 6: the first note, if it is a standard punctum
%% 7: the first note, if it is an initio debilis
%% 8: the first note, if it is a porrectus or a torculus resupinus
%% 9: the last note, if it is a punctum inclinatum
%% 10: idem, if it is a punctum inclinatum debilis
%% 11: idem, if it is a stropha
%% 12: idem, with a quilisma
%% 13: idem, with an oriscus
%% 14: idem of 2, but with ambitus of 1 between the two last notes (so no link between them) 
\def\vepisemus#1#2{%
\calculateglyphraisevalue{#1}{3}%
\ifcase#2 %
%case 0
\vepisemusaux{0}{\char 17}{0}%
\or%
%case 1
\vepisemusaux{0}{\char 13}{0}%
\or%
%case 2
% \char 26626 is a kind of flexus, it has the good width
\vepisemusaux{\char 26626}{\char 17}{1}%
\or%
%case 3
% in order to go to the good place, we first make a kern of - the glyph before deminutus, which has the same width as a standard flexus deminutus, that is to say \char 14849
\vepisemusaux{0}{\char 14849}{1}%
\or%
%case 4
% \char 28678 is a torculus, it has the good width
\vepisemusaux{\char 28678}{\char 17}{1}%
\or%
%case 5
% \char 29190 is a torculus deminutus, it has the good width
\vepisemusaux{\char 29190}{\char 17}{1}%
\or%
%case 6
\vepisemusaux{0}{\char 17}{2}%
\or%
%case 7
\vepisemusaux{0}{\char 13}{2}%
\or%
%case 8, in which we do (for now) the same as case 6
\vepisemusaux{0}{\char 17}{2}%
\or%
%case 9
\vepisemusaux{0}{\char 19}{0}%
\or%
%case 10
\vepisemusaux{0}{\char 32}{0}%
\or%
%case 11
\vepisemusaux{0}{\char 20}{0}%
\or%
%case 12
\vepisemusaux{0}{\char 26}{0}%
\or%
%case 13
\vepisemusaux{0}{\char 27}{0}%
\or%
%case 14
\vepisemusaux{\char 26625}{\char 17}{1}%
\fi%
\relax%
}


% a macro that will help in the typesetting of a horizontal episemus, the first argument is a glyph that have the same width as the width between the end of the glyph and the beginning of the episemus, and the second argument is the character of the episemus. If the third argument is 0, we go directly to the beginning of the glyph, else we don't change anything
\def\hepisemusaux#1#2#3{%
\ifnum#3=0%
%remember, \tempwidth has the value of the last glyph width (cool isn't it?)
\else%
\setbox\Tempsign=\hbox{\gregorianfont #1}%
\tempwidth=\wd\Tempsign%
\fi%
% then we draw the sign, and go back to the beginning of the sign
\setbox\Tempsign=\hbox{\gregorianfont \char #2}%
% we set tempwidth to half a punctum malus half the sign width, so that the centers are aligned
\tempsignwidth=\wd\Tempsign %
\kern -\tempwidth%
\raise \glyphraisevalue \copy\Tempsign %
% and finally we go back to the end of the glyph, where we were first
\advance\tempwidth by -\tempsignwidth %
\kern \tempwidth%
\relax%
}

% a function to typeset a horizontal episemus. The firts argument is the letter of the height of the episemus (not the height of the note it corresponds to. This function must be called after a call to \glyph. The second argument is the type of glyph it was, more precisely the kind of space there is between the end (or in special cases the beginning) of the glyph and the place where we will typeset the episemus. The possible values are:
%% 0: the episemus will be typeset at the beginning of the last note, which is a standard punctum (works with pes)
%% 1: the same, but the last note is a deminutus
%% 2: the episemus is typeset at the beginning of the note before the last note, which is a standard punctum
%% 3: idem, but the note is the note preceding a deminutus
%% 4: the note before the note before the last note (for torculus resupinus)
%% 5: idem, but when the two last notes are a deminutus
%% 6: the first note, if it is a standard punctum
%% 7: the first note, if it is an initio debilis
%% 8: the two first notes, if it is a porrectus
%% 9: the two first notes, if it is a porrectus flexus
% the third argument is a bit particular, it is the ambitus of the porrectus or porrectus flexus if the second argument is 8 or 9, otherwise it is useless
%% 10: the last note, if it is a punctum inclinatum
%% 11: idem, if it is a punctum inclinatum debilis
%% 12: idem, if it is a stropha
%% 13: idem, with a quilisma
%% 14: idem, with an oriscus
%%% cases for the special spaces between the notes (for ambitus of one)
%% 15: same of 2
\def\hepisemus#1#2#3{%
\calculateglyphraisevalue{#1}{1}%
\ifcase#2 %
%case 0
\hepisemusaux{\char 17}{40}{1}%
\or%
%case 1
\hepisemusaux{\char 13}{42}{1}%
\or%
%case 2
% \char 26626 is a kind of flexus, it has the good width
\hepisemusaux{\char 26626}{40}{1}%
\or%
%case 3
% in order to go to the good place, we first make a kern of - the glyph before deminutus, which has the same width as a standard flexus deminutus, that is to say \char 14849
\hepisemusaux{\char 14849}{40}{1}%
\or%
%case 4
% \char 28678 is a torculus, it has the good width
\hepisemusaux{\char 28678}{40}{1}%
\or%
%case 5
% \char 29190 is a torculus deminutus, it has the good width
\hepisemusaux{\char 29190}{40}{1}%
\or%
%case 6
\hepisemusaux{0}{40}{0}%
\or%
%case 7
%we assume that the initio-debilis has the same width as a punctum deminutus
\hepisemusaux{0}{41}{0}%
\or%
%case 8
\ifcase#3%
\or%
\hepisemusaux{0}{45}{0}%
\or%
\hepisemusaux{0}{46}{0}%
\or%
\hepisemusaux{0}{47}{0}%
\or%
\hepisemusaux{0}{48}{0}%
\or%
\hepisemusaux{0}{49}{0}%
\fi%
\or%
%case 9
\ifcase#3%
\or%
\hepisemusaux{0}{50}{0}%
\or%
\hepisemusaux{0}{51}{0}%
\or%
\hepisemusaux{0}{52}{0}%
\or%
\hepisemusaux{0}{53}{0}%
\or%
\hepisemusaux{0}{54}{0}%
\fi%
\or%
%case 10
\hepisemusaux{\char 19}{43}{1}%
\or%
%case 11
\hepisemusaux{\char 32}{44}{1}%
\or%
%case 12
\hepisemusaux{\char 20}{45}{1}%
\or%
%case 13
\hepisemusaux{\char 26}{56}{1}%
\or%
%case 14
\hepisemusaux{\char 27}{47}{1}%
\or%
%case 15
\hepisemusaux{\char 26625}{58}{1}%
\fi%
\relax%
}

%%%%%%%%%%%%%%%%%%%%%%%%%%%%%%%%%%%%%%%%%%%%%%%%%%%%%%%%%%
%% macros for the typesetting of glyphs and notes together
%%%%%%%%%%%%%%%%%%%%%%%%%%%%%%%%%%%%%%%%%%%%%%%%%%%%%%%%%%

% a dimen that will contain the difference between the end of the text and the end of the notes for the previous syllable (if we are in the same word) : positive if notes go further than text. We will use it for space adjustment between syllables of the same word

\newdimen\enddifference 

% a dimen that will contain the enddifference of the previous glyph

\newdimen\previousenddifference 

% \begindifference is the difference between the begginning of the text and the beginning of the notes. Warning : it can be negative.

\newdimen\begindifference

\newdimen\temp

% box that will contain the text of the syllable

\newbox\Syllabletext

% count that will be 0 if in the last text there was no dash (or if it is the beginning of a word, and 1 if there was
\newcount\previousdash

% a value that will allow us to have regular spaces between notes of difference syllables
\newskip\intersyllableeffectivespace

%\attributedef\potentialdash 1
%\potentialdash=1
%\attributedef {potentialdash} 16

% \zerowidthdash is a special character that will be treated by lua after the linebreak algorithm, see lua part of the file for more details
%\newbox\Zerowidthdash
%\setbox\Zerowidthdash=\hbox to 0pt{\hbox to 1pt{}}

% we specify if it is the last syllable of the score with that value
\newcount\endofscore

%% general macro : it will typeset the syllable : arguments are :
% #1 : the first letters of the syllable, that don't count for the alignment
% #2 : the middle letters of the syllable, we must align in the middle of them
% #3 : the end letters, they don't count
% #4 : end of word : if it is 0 it means it is not an end of word, if it is 1 it is
% TODO: find another system for the end syllable
%% #5 (not used anymore) : beggining of word : if it is 0 it means it is not a beginning of word, if it is 1 it is
% #5 : glyphs : all the notes
\def\syllable#1#2#3#4#5{%
\firstglyph=1 %
\findtextaligncenter{#1}{#2}{#3}% we first get the width between the alignment point and the end of the syllable
% now we calculate the begin difference, that is to say \notesaligncenter - \textaligncenter
\begindifference=\notesaligncenter %
\advance\begindifference by -\textaligncenter %
% then we do a kind of trick to separate more the notes of two different syllables : we put a space if we are in the same word (enddifference!=0) if necessary
% it is necessary if previousenddifference - begindifference < intersyllablespace
\temp=\enddifference %
\advance\temp by -\begindifference
\ifdim\temp<\intersyllablespace
\intersyllableeffectivespace=\intersyllablespace %
\advance\intersyllableeffectivespace by -\temp %
\hskip\intersyllableeffectivespace %
\fi% end \ifdim\temp<\intersyllablespace
\syllablenotes{#5}% we put the notes in a box, so that we have the width of it
%\potentialdash=2 %
\setbox\Syllabletext=\hbox{\textfont #1#2#3%
%\hbox to 0pt{\hbox to 0pt{}}%
}%
%\potentialdash=1 %
\ifcase#4 %
% we enter here if the end of word is 0, so we must determine if we need to type the gragorio dash here
% we determine if \wd\Syllablenotes > tempwidth (width of text without dash) + \gregoriotextdashwidth - textaligncenter + notesaligncenter
\temp=\wd\Syllabletext %
\advance\temp by \gregoriotextdashwidth %
\advance\temp by -\textaligncenter %
\advance\temp by \notesaligncenter %
\ifdim\wd\Syllablenotes>\temp %
%\potentialdash=2 %
\setbox\Syllabletext=\hbox{\textfont #1#2#3%
\gregoriotextdash%
%\hbox to 0pt{\hbox to 0pt{}}%
}%
\global\previousdash=1 %
%\potentialdash=1 %
\else%
%\potentialdash=3 %
\setbox\Syllabletext=\hbox{\textfont #1#2#3%
%\gregoriotextdash%
%\hbox to 0pt{\hbox to 0pt{}}%
}%
%\potentialdash=1 %
\global\previousdash=0 %
\fi%
\fi%
% then we must typeset the two boxes, but we must begin by the one that will go the less far, so we calculate the \enddifference, and we see if it is positive or not. Settenddifferents also sets \previousenddifference to the good value
\setenddifference{\wd\Syllablenotes}{\wd\Syllabletext}{\textaligncenter}{\notesaligncenter}%
%% complicated things with the enddifference of the previous glyph, but only if it is not a begginning of word, but we don't use them anymore
%\ifcase#5 %
%\the\begindifference %
%\setsyllableshift{\begindifference}%
%\fi%
% then we reuse temp, we assign to it the \begindifference, but only if it is positive, else it is 0
\temp=\begindifference %
\ifdim\temp <0pt %
\temp=0pt%
\fi%
\kern \temp %
\lower\textlower \copy\Syllabletext %
%\-%
\kern -\wd\Syllabletext %
\kern -\begindifference %
#5% we do that instead of \unhbox\Syllablnotes, because it would not set the \localrightbox
\ifdim\the\enddifference <0pt%
%% important, else we are not really at the end of the syllable
\kern -\enddifference %
\fi%
%% then we call end of syllable or end of word
%\-%
\ifcase#4 %
\endofsyllable %
\or%
\endofword %
\fi%
\relax%
}

% macro to set \enddifference (defined above) to \wd\Syllablenotes - (\wd\Syllabletext - \textaligncenter) - \notesaligncenter
% \enddifference will be positive if text go further than the notes, and negative in the other case
% arguments are :
% #1: \wd\Syllablenotes : the total width of the notes
% #2: \wd\Syllabletext : the total width of the text
% #3: \textaligncenter (defined above)
% #4: \notesaligncenter (defined above too)
\def\setenddifference#1#2#3#4{%
\global\previousenddifference=\enddifference
\global\enddifference=#1%
\global\advance\enddifference by -#2%
\global\advance\enddifference by #3%
\global\advance\enddifference by -#4%
\relax%
}

%% Finally we don't use it, because syllables never cross, I keep it, just in case...
% macro that will calculate the shift that we apply at the beginning, to combine two syllables of the same note
% arguments are :
% #1: \begindifferrence, defined above
% but the macro also uses \previousenddifference, \previousdash (not yet)
%\def\setsyllableshift#1{%
%\the\previousenddifference %
%\ifdim\previousenddifference >0pt %
%\hskip\intersyllablenotesspace %
%\ifdim-#1<\previousenddifference %
%\kern #1%
%test1 %
%\else%
%\kern -\previousenddifference %
%test2%
%\fi%
%\else%
% we test if begin > end - intersyllablespace
%\temp=\previousenddifference %
%\advance\temp by \intersyllablenotesspace %
%\ifdim#1 >\temp %
%\kern #1 %
%test3%
%\else%
%\kern\temp%
%test4%
%\fi%
%\fi%
%}

%%%%%%%%%%%%%%%%%%%%%%%%%%%%%%%%%%%%%
%% macros for the typesetting of bars
%%%%%%%%%%%%%%%%%%%%%%%%%%%%%%%%%%%%%

% we define two types of macro for each four bar : when it is inside a syllable, and when it is not

\def\invirgula{%
\writebar{0}{1}%
\relax%
}

\def\virgula{%
\writebar{0}{0}%
\relax%
}

\def\indivisiominima{%
\writebar{1}{1}%
\relax%
}

\def\divisiominima{%
\writebar{1}{0}%
\relax%
}

\def\indivisiominor{%
\writebar{2}{1}%
\relax%
}

\def\divisiominor{%
\writebar{2}{0}%
\relax%
}

\def\indivisiomaior{%
\writebar{3}{1}%
\relax%
}

\def\divisiomaior{%
\writebar{3}{0}%
\relax%
}

\newdimen\temptwo

\def\indivisiofinalis{%
\ifcase\endofscore %
\writebar{4}{1}%
\or %
\writebar{5}{1}%
\fi %
\relax%
}

\def\divisiofinalis{%
\ifcase\endofscore %
\writebar{4}{0}%
\or %
\writebar{5}{0}%
\fi %
\relax%
}

%a macro to write a bar
%% 1: the type of the bar : 0 for virgula, 1 for minima 2 for minor, 3 for major, 4 for finalis and 5 for the last finalis
%% 2: is % for now we don't use it
%%% 0 if it is outside a syllable
%%% 1 if it is in a syllable
\def\writebar#1#2{%
\ifcase#1 % 0 : virgula
\penalty 7000 %
\hskip\spacebeforesmallbar %
\penalty 7000 %
\calculateglyphraisevalue{g}{0}% bar glyphs are made to be at this height
\raise\glyphraisevalue\hbox{\gregorianfont \char 8}%
\penalty -5000 %
\hskip\spaceaftersmallbar %
\or % 1 : minima
\penalty 7000%
\hskip\spacebeforesmallbar %
\penalty 7000%
\calculateglyphraisevalue{g}{0}% bar glyphs are made to be at this height
\raise\glyphraisevalue\hbox{\gregorianfont \char 9}%
\penalty -5000%
\hskip\spaceaftersmallbar %
\or % 2 : minor
\penalty 7000 %
\hskip\spacebeforeminor %
\penalty 7000 %
\calculateglyphraisevalue{g}{0}% bar glyphs are made to be at this height
\raise\glyphraisevalue\hbox{\gregorianfont \char 10}%
\penalty -5000 %
\hskip\spaceafterminor %
\or % 3 : maior
\penalty 7000 %
\hskip\spacebeforemaior %
\penalty 7000 %
\calculateglyphraisevalue{g}{0}% bar glyphs are made to be at this height
\raise\glyphraisevalue\hbox{\gregorianfont \char 11}%
\penalty -5000 %
\hskip\spaceaftermaior %
\or % 4 : finalis
\penalty 7000 %
\hskip\spacebeforefinalis %
\penalty 7000 %
\calculateglyphraisevalue{g}{0}% bar glyphs are made to be at this height
\raise\glyphraisevalue\hbox{\gregorianfont \char 11}%
\penalty 7000 %
\temptwo = 12000 sp%
\multiply\temptwo by \the\factor%
\kern \temptwo%
\penalty 7000 %
\raise\glyphraisevalue\hbox{\gregorianfont \char 11}%
\penalty -5000 %
\hskip\spaceafterfinalis %
\or % 5 : finalis
\penalty 7000 %
\hskip\spacebeforefinalfinalis %
\penalty 7000 %
\calculateglyphraisevalue{g}{0}% bar glyphs are made to be at this height
\raise\glyphraisevalue\hbox{\gregorianfont \char 11}%
\temptwo = 12000 sp%
\multiply\temptwo by \the\factor%
\kern \temptwo%
\penalty 10000 %
\raise\glyphraisevalue\hbox{\gregorianfont \char 11}%
\penalty -5000 %
\hskip\spaceafterfinalis %
\fi %
\relax%
}

%a count to tell if we have to keep the localrightbox until the end
\newcount\keeprightbox

%macro to end a line with a divisio finalis
\def\enddivisiofinalis{
\calculateglyphraisevalue{g}{0}
\global\keeprightbox=1 %
\localrightbox{%
\raise\glyphraisevalue\hbox{\gregorianfont \char 11}%
\temptwo = 12000 sp%
\multiply\temptwo by \the\factor%
\kern \temptwo%
\raise\glyphraisevalue\hbox{\gregorianfont \char 11}%
\kern 0 sp%
}%
\relax%
}

%%%%%%%%%%%%%%%%%%%%%%%%%%%%%%%%%%%%%%%
%% macros for typesetting alterations
%%%%%%%%%%%%%%%%%%%%%%%%%%%%%%%%%%%%%%%

\def\flat#1{%
\endofglyph{2}%
\relax%
}

\def\natural#1{%
\endofglyph{2}%
\relax%
}

%%%%%%%%%%%%%%%%%%%%%%%%%%%%%%%%%%%%%%%
%% other macros
%%%%%%%%%%%%%%%%%%%%%%%%%%%%%%%%%%%%%%%


%macro called at the beginning of a score
\def\begingregorioscore{%
\noindent%
%\directlua0 {%
%local hlist = node.id('hlist')
%tempnode=node.new(node.id('glyph'), 0)
%tempnode.font=65 % TODO : don't know why 65...
%tempnode.char=tex.defaulthyphenchar
%dashnode=node.hpack(tempnode)
%dashnode.shift=393216 % TODO : a less static value
%function addhyphen(h, groupcode, glyphes)
%    local lastseennode=nil
%    local attributeid=1
%    local potentialdashvalue=3
%    local nopotentialdashvalue=2
%    local adddash=false
%    % we explore the lines
%    for a in node.traverse_id(hlist, h) do
%        for b in node.traverse_id(hlist, a.list) do
%            if adddash == false then
%                if node.has_attribute(b, attributeid, potentialdashvalue) then
%                    adddash=true
%                    lastseennode=b
%                    attr = b.attr.next
%                    %texio.write_nl('ATTR number = ' .. attr.number .. ' value = ' .. attr.value)
%                end
%            else
%                if node.has_attribute(b, attributeid, nopotentialdashvalue) then
%                    adddash=false
%                    attr = b.attr.next
%                    %texio.write_nl('ATTR number = ' .. attr.number .. ' value = ' .. attr.value)
%                end 
%            end
%        end
%        if adddash==true then
%            local temp= node.copy(dashnode)
%            %TODO: remove the last glue in b first, and insert a glue of ancient glue - withdof(dashnode) after the dash
%            node.insert_after(a.list, b, temp)
%            addash=false
%        end
%    end
%    return true
%end 
%callback.register('post_linebreak_filter', addhyphen)
%}%
\relax%
}

%macro called at the beginning of a score
\def\endgregorioscore{%
\localleftbox{}%
\ifnum\keeprightbox=0 %
\localrightbox{}%
\fi %
\hfil %
\par%
\ifnum\keeprightbox=1 %
\localrightbox{}%
\global\keeprightbox=0 %
\fi%
\relax%
}

%macro to call if you wan to go to the next line
\def\gnewline{%
\hfil%
\penalty-10000 %
\relax%
}

%macro to call when there is just a little thing that will go to the last line, when it is not necessary
\def\gnolastline{%
\ifdim\enddifference >0pt %
\hskip-\interwordspacenotes %
\else%
\hskip-\interwordspacetext %
\fi %
\global\endofscore=1 %
\localrightbox{}%
\localleftbox{}%
\penalty 1000 %
\relax%
}

% macro called at each end of word
\def\endofword{%
\ifdim\enddifference >0pt %
\hskip\interwordspacenotes %
\else%
\hskip\interwordspacetext %
\fi %
\penalty -1000%
\global\enddifference=0pt %
\relax%
}

% macro called at each end of syllable which is not an end of word
\def\endofsyllable{%
\penalty -500%
\relax%
}

% macro to end elements, argument is the type of space, it can be : 
%% 0 : default space 
%% 1 : larger space
%% 2 : glyph space

\def\endofelement#1{%
\ifcase#1%
\hskip\interelementspace%
\or%
\hskip\largerspace%
\or%
\hskip\glyphspace%
\fi%
\penalty -100%
\relax%
}

% macro to end a glyph without ending the element, argument is the type of space, it can be : 
%% 0: default space 
%% 1: zero width space
%% 2: space between flat or natural and a note
%% 3: shift when there is a punctum_inclinatum after
%% 4: space between bivirga or bistropha
%% 5: space between tristropha or trivirga
%% 6: space after a punctum % TODO: use it

\def\endofglyph#1{%
\ifcase#1%
\hskip\interglyphspace %
\or%
\hskip\zerowidthspace %
\or%
\hskip\alterationspace %
\or%
\hskip\punctuminclinatumshift %
\or%
\hskip\bispace %
\or%
\hskip\trispace %
\or%
\hskip\spaceaftersigns %
\fi%
\penalty 10001%
\relax%
}

%%%%%%%%%%%%%%%%%%%%%%%%%%%%%%%%%%%%%%%
%% macros for the typesetting of spaces
%%%%%%%%%%%%%%%%%%%%%%%%%%%%%%%%%%%%%%%

% null space
\newskip\zerowidthspace
\zerowidthspace=0pt plus 0pt minus 0pt

% space between glyphs in the same element
\newskip\interglyphspace
\interglyphspace = 8200 sp plus 820 sp minus 820 sp
\multiply\interglyphspace by \factor

% space between an alteration (flat or natural) and the next glyph
\newskip\alterationspace
\alterationspace = 8200 sp plus 820 sp minus 820 sp
\multiply\alterationspace by \factor

% space between elements
\newskip\interelementspace
\interelementspace = 8200 sp plus 820 sp minus 820 sp
\multiply\interelementspace by \factor

% larger space between elements
\newskip\largerspace
\largerspace = 8200 sp plus 820 sp minus 820 sp
\multiply\largerspace by \factor

% space between elements which has the size of a note
\newskip\glyphspace
\glyphspace = 16500 sp plus 1650 sp minus 1650 sp
\multiply\glyphspace by \factor

% minimum space between two notes of different syllables
\newskip\intersyllablespace
\intersyllablespace=18000 sp plus 2000 sp minus 2000 sp
\multiply\intersyllablespace by \factor

% space before custo
\newskip\spacebeforecusto
\spacebeforecusto = 16500 sp plus 16500 sp minus 8000 sp
\multiply\spacebeforecusto by \factor

% space before punctum mora and augmentum duplex
\newskip\spacebeforesigns
\spacebeforesigns=5000 sp plus 500 sp minus 500 sp
\multiply\spacebeforesigns by \factor

% space after punctum mora and augmentum duplex
\newskip\spaceaftersigns
\spaceaftersigns=9000 sp plus 900 sp minus 900 sp
\multiply\spaceaftersigns by \factor

% space after a clef at the beginning of a line
\newskip\spaceafterlineclef
\spaceafterlineclef = 16500 sp plus 1650 sp minus 1650 sp
\multiply\spaceafterlineclef by \factor

% space after at the end of a word when the last written symbol is a note
\newskip\interwordspacenotes
\interwordspacenotes = 21000 sp plus 2100 sp minus 2100 sp
\multiply\interwordspacenotes by \factor

% space after at the end of a word when the last written symbol is text
\newskip\interwordspacetext
\interwordspacetext = 21000 sp plus 2100 sp minus 2100 sp
\multiply\interwordspacetext by \factor

% space between notes of a bistropha and bivirga
\newskip\bispace
\bispace = 16500 sp plus 1650 sp minus 1650 sp
\multiply\bispace by \factor

% space between notes of a tristropha and trivirga
\newskip\trispace
\trispace = 16500 sp plus 1650 sp minus 1650 sp
\multiply\trispace by \factor

% space when there is a 
\newskip\punctuminclinatumshift
\punctuminclinatumshift=-4000 sp plus 400 sp minus 400 sp
\multiply\punctuminclinatumshift by \factor

% space between the initial and the beginning of the score
\newskip\afterinitialshift
\afterinitialshift=0.6 em plus 0em minus 0em

% space for the bars
%first for virgula and divisio minima
\newskip\spacebeforesmallbar
\spacebeforesmallbar = 8500 sp plus 1650 sp minus 1650 sp
\multiply\spacebeforesmallbar by \factor

\newskip\spaceaftersmallbar
\spaceaftersmallbar = 8500 sp plus 1650 sp minus 1650 sp
\multiply\spaceaftersmallbar by \factor

%then divisio minor
\newskip\spacebeforeminor
\spacebeforeminor = 8500 sp plus 1650 sp minus 1650 sp
\multiply\spacebeforeminor by \factor

\newskip\spaceafterminor
\spaceafterminor = 8500 sp plus 1650 sp minus 1650 sp
\multiply\spaceafterminor by \factor

%divisio major
\newskip\spacebeforemaior
\spacebeforemaior = 8500 sp plus 1650 sp minus 1650 sp
\multiply\spacebeforemaior by \factor

\newskip\spaceaftermaior
\spaceaftermaior = 8500 sp plus 1650 sp minus 1650 sp
\multiply\spaceaftermaior by \factor

%divisio finalis
\newskip\spacebeforefinalis
\spacebeforefinalis = 8500 sp plus 1650 sp minus 1650 sp
\multiply\spacebeforefinalis by \factor

%a special space for finalis, for when it is the last glyph
\newskip\spacebeforefinalfinalis
\spacebeforefinalfinalis= 8000 sp plus 1000 sp minus 1000 sp
\multiply\spacebeforefinalfinalis by \factor

\newskip\spaceafterfinalis
\spaceafterfinalis = 8500 sp plus 1650 sp minus 6000 sp
\multiply\spaceafterfinalis by \factor

% space for the clef changes
\newskip\clefchangespace

%%%%%%%%
%% fonts
%%%%%%%%

% the macros to typeset the A, R and V with bar
\def\Abar{%
{\gregoriantextfont \char 66}%
\relax%
}

\def\Rbar{%
{\gregoriantextfont \char 67}%
\relax%
}

\def\Vbar{%
{\gregoriantextfont \char 68}%
\relax%
}

% the macro to write a cross
\def\gcross{%
}

%the macro to write a star
\def\gstar{%
\raise -0.1em\hbox{*}%
\relax%
}


\newcount\tempfactor %

\def\setgregorianfont#1{%
\tempfactor = \the\factor %
\multiply\tempfactor by 100000 %
\global\font\gregorianfont=#1 at \the\tempfactor sp%
\global\font\gregoriantextfont=#1 at 11pt%
\relax%
}

\tempfactor = \the\factor %
\multiply\tempfactor by 100000 %
% we open the font at \factor pt, we must know \factor because it will determine the line heigth, etc. 
\font\gregorianfont=greciliae at \the\tempfactor sp

\font\gregoriantextfont=greciliae at 11pt

\font\textfont=pncr at 11pt
\font\fontofinitial=pncr at 38pt

\def\setfactor#1{%
\global\factor=#1 %
%\global\multiply\factor by 100000 %
\relax %
}
