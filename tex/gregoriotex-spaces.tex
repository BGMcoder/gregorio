%GregorioTeX file.
%Copyright (C) 2007-2010 Elie Roux <elie.roux@telecom-bretagne.eu>
%
%This program is free software: you can redistribute it and/or modify
%it under the terms of the GNU General Public License as published by
%the Free Software Foundation, either version 3 of the License, or
%(at your option) any later version.
%
%This program is distributed in the hope that it will be useful,
%but WITHOUT ANY WARRANTY; without even the implied warranty of
%MERCHANTABILITY or FITNESS FOR A PARTICULAR PURPOSE.  See the
%GNU General Public License for more details.
%
%You should have received a copy of the GNU General Public License
%along with this program.  If not, see <http://www.gnu.org/licenses/>.

% this file contains definitions of spaces

\gredeclarefileversion{gregoriotex-spaces.tex}%
 {\directlua{tex.write(gregoriotex.get_greapiversion())}}

%%%%%%%%%%%%%%%%%%%%%%%%%%%%%%
%% macros for tuning penalties
%%%%%%%%%%%%%%%%%%%%%%%%%%%%%%

%% The following macros enable users to tune penalties used in Gregorio

% penalty to force a break on a new line
\xdef\grenewlinepenalty{-10001}
\def\greforcebreak{\grepenalty{\grenewlinepenalty}}

% penalty to prevent a line break
\xdef\grenobreakpenalty{10001}
\def\grenobreak{\grepenalty{\grenobreakpenalty}}

% called in \grenolastline (seems deprecated...)
\xdef\grenolastlinepenalty{100}

% penalty at the end of a syllable which is the end of a word
\xdef\greendofwordpenalty{-100}

% penalty at the end of a syllable which is not the end of a word
\xdef\greendofsyllablepenalty{-50}

% penalty at the end of a syllable which is just a bar, with something printed
% under it
\xdef\greendafterbarpenalty{-200}

% penalty right after a bar with nothing printed
\xdef\greendafterbaraltpenalty{-200}

% penalty at the end of a breakable neumatic element (typically at a space
% between elements)
\xdef\greendofelementpenalty{-50}

% hyphenpenalty will be used in discretionaries, in Gregorio this is used for
% a bar with clef change for example. It also set \exhyphenpenalty. It should
% be close to \greendafterbarpenalty
\xdef\grehyphenpenalty{-200}

% broken penalty is the vertical penalty inserted after a break on a clef change
% I'm not sure it should be set, but it might be useful...
\xdef\grebrokenpenalty{0}

%% The following macros cancel some useless penalties, and reinstances them
%% at the end of a score

\def\grecancelpenalties{%
  \xdef\grehyphenpenaltysave{\the\hyphenpenalty }%
  \xdef\greexhyphenpenaltysave{\the\exhyphenpenalty }%
  \xdef\gredoublehyphendemeritssave{\the\doublehyphendemerits }%
  \xdef\grefinalhyphendemeritssave{\the\finalhyphendemerits }%
  \xdef\grebrokenpenaltysave{\the\brokenpenalty }%
  \hyphenpenalty=\grehyphenpenalty\relax %
  \exhyphenpenalty=\grehyphenpenalty\relax %
  \doublehyphendemerits=0\relax %
  \finalhyphendemerits=0\relax %
  \brokenpenalty=\grebrokenpenalty\relax %
}

\def\grerestorepenalties{%
  \hyphenpenalty=\grehyphenpenaltysave %
  \exhyphenpenalty=\greexhyphenpenaltysave %
  \doublehyphendemerits=\gredoublehyphendemeritssave %
  \finalhyphendemerits=\grefinalhyphendemeritssave %
  \brokenpenalty=\grebrokenpenaltysave %
}

%% These macro enable the tuning of linepenalty, tolerance, pretolerance
%% and emergencystretch

% the macros to be modified by the users, 
\def\grelooseness{\looseness}
\def\gretolerance{\tolerance}
% Workaround for bug 842 (http://tracker.luatex.org/view.php?id=842)
% see http://tug.org/pipermail/luatex/2013-July/004516.html
\ifnum\the\luatexversion < 78\relax %
  \global\def\grepretolerance{-1}
\else %
  \global\def\grepretolerance{\pretolerance}
\fi %
\def\greemergencystretch{\emergencystretch}
\def\grewidowpenalty{\widowpenalty}
\def\greclubpenalty{\clubpenalty}

% macro called at ea
\def\gredofinetuning{%
  \xdef\greloosenesssave{\the\looseness}%
  \xdef\gretolerancesave{\the\tolerance}%
  \xdef\grepretolerancesave{\the\pretolerance}%
  \xdef\greemergencystretchsave{\the\emergencystretch}%
  \xdef\grewidowpenaltysave{\the\widowpenalty}%
  \xdef\greclubpenaltysave{\the\clubpenalty}%
  \looseness=\grelooseness %
  \tolerance=\gretolerance %
  \pretolerance=\grepretolerance %
  \emergencystretch=\greemergencystretch %
  \widowpenalty=\grewidowpenalty %
  \clubpenalty=\greclubpenalty %
}

\def\greendfinetuning{%
  \looseness=\greloosenesssave %
  \tolerance=\gretolerancesave %
  \pretolerance=\grepretolerancesave %
  \emergencystretch=\greemergencystretchsave %
  \widowpenalty=\grewidowpenaltysave %
  \clubpenalty=\greclubpenaltysave %
}


%%%%%%%%%%%%%%%%%%%%%%%%%%%%%%%%%%%%%%%
%% macros for the typesetting of spaces
%%%%%%%%%%%%%%%%%%%%%%%%%%%%%%%%%%%%%%%

% Independent default distances are defined in gsp-default.tex.  The distances defined here are calculated from those distances.

%%%%%%%%%%%%%%%%%%%%%%%%%%%%%%%%%%%%%%%%%
%%%%%%%%%%%%%%%%%%%%%%%%%%%%%%%%%%%%%%%%%
%%%%%%%%%%%%%%%%%%%%%%%%%%%%%%%%%%%%%%%%%
%%%%%%%%%%%%%%%
%%  old code which will eventually be deleted
%%%%%%%%%%%%%%%

%%%%%%%%%%%%%%%%%%%%
%% horizontal spaces
%%%%%%%%%%%%%%%%%%%%

% additional lines width
\newskip\gre@additionallineswidth

% null space
\newskip\gre@zerowidthspace

% space between glyphs in the same element
\newskip\gre@interglyphspace

% space between an alteration (flat or natural) and the next glyph
\newskip\gre@alterationspace

% space before a choral sign
\newskip\gre@beforechoralsignspace

% space between a clef and a flat (for clefs with flat)
\newskip\gre@clefflatspace

% negative space, difference between the normal space between two notes and the space between a note and a flat
\newskip\gre@beforealterationspace

% space between elements
\newskip\gre@interelementspace

% larger space between elements
\newskip\gre@largerspace

% space between elements which has the size of a note
\newskip\gre@glyphspace

% minimum space between two notes of different syllables
\newskip\gre@intersyllablespace

% space before custo
\newskip\gre@spacebeforecusto

% space before punctum mora and augmentum duplex
\newskip\gre@spacebeforesigns

% space after punctum mora and augmentum duplex
\newskip\gre@spaceaftersigns

% space after a clef at the beginning of a line
\newskip\gre@spaceafterlineclef

% space after at the end of a word when the last written symbol is a note and the first is a note
\newskip\gre@interwordspacenotes

% space after at the end of a word when the last written symbol is a note and the first is text
\newskip\gre@interwordspacenotestext

% space after at the end of a word when the last written symbol is text and the first is a note
\newskip\gre@interwordspacetextnotes

% space after at the end of a word when the last written symbol is text and the first is text
\newskip\gre@interwordspacetext

% space between notes of a bivirga or trivirga
\newskip\gre@bitrivirspace

% space between notes of a bistropha or tristropha
\newskip\gre@bitristrospace

% space between two punctum inclinatum
\newskip\gre@punctuminclinatumshift

% space before puncta inclinata
\newskip\gre@beforepunctainclinatashift

% space between a punctum inclinatum and a punctum inclinatum deminutus
\newskip\gre@punctuminclinatumanddebilisshift

% space between two punctum inclinatum deminutus
\newskip\gre@punctuminclinatumdebilisshift

% space between puncta inclinata, larger ambitus (range=3rd)
\newskip\gre@punctuminclinatumbigshift %

% space between puncta inclinata, larger ambitus (range=4th -or more?-)
\newskip\gre@punctuminclinatummaxshift %

% space for the bars (inside syllables)
%first for virgula and divisio minima
\newskip\gre@spacearoundsmallbar

%then divisio minor
\newskip\gre@spacearoundminor

%divisio major
\newskip\gre@spacearoundmaior

%divisio finalis
\newskip\gre@spacearoundfinalis

%a special space for finalis, for when it is the last glyph
\newskip\gre@spacebeforefinalfinalis

% the space that will appear around bars that are preceded by a custo and followed by a key.
% well... actually it's the difference between the normal space around bars and the space described previously.
\newskip\gre@spacearoundclefbars

% space between the text and the text of the bar
\newskip\gre@textbartextspace

% minimal space between a note and a bar
\newskip\gre@notebarspace

% maximal space between two syllables for which we consider a dash is not needed
\newdimen\gre@maximumspacewithoutdash

% an extensible space for the beginning of lines
\newskip\gre@afterclefnospace

% width of the additional lines, used only for the custos (the width of the custo is added to it)
% the width is the one for the custos at end of lines, the line for custos in the middle of a score is the same
% multiplied by 2.
\newdimen\gre@additionalcustoslineswidth

% space between the initial and the beginning of the score
\newskip\gre@afterinitialshift

% space before the initial and the beginning of the score
\newskip\gre@beforeinitialshift

% forced size of the box for the initial.  Initial is centered in box.  Ignored if 0.
\newskip\gre@manualinitialwidth

% space before the initial and the beginning of the score
\newdimen\gre@minimalspaceatlinebeginning

% this space is the one between the bottom of the first anotation line and the top
% of the second anotation line (above the initial)
\newdimen\gre@aboveinitialseparation

\newdimen\gre@clivisalignmentmin

%%%%%%%%%%%%%%%%%%
%% vertical spaces
%%%%%%%%%%%%%%%%%%
\newdimen\gre@abovesignsspace
\newdimen\gre@belowsignsspace

% the shift for the low choral sign
\newdimen\gre@lowchoralsignshift

% the shift for the high choral sign
\newdimen\gre@highchoralsignshift

% the space for the translation
\newdimen\gre@translationheight

%the space above the lines
\newskip\gre@spaceabovelines

% shift of the text above the note line
\newdimen\gre@abovelinestextraise 

% additional space in case of text above the lines
\newdimen\gre@abovelinestextheight

%the space between the lines and the bottom of the text
\newskip\gre@spacelinestext

%the space beneath the text
\newskip\gre@spacebeneathtext

%%% the following values are computed from the others, after some calculus

% staffheight is the total height of the staff : that is to say the four written lines
\newdimen\gre@staffheight 

% space for the clef changes
\newskip\gre@clefchangespace

% space at the beginning of the lines when there is no clef
\newdimen\gre@noclefspace

% an additional shift you can give to the brace above the bars if you don't like it
\newskip\gre@braceshift

% a shift you can give to the accentus above the curly brace
\newskip\gre@curlybraceaccentusshift

%%%%%%%%%%%%%%%%%%%%%%%%%%%%%%%%%%%%%%%%%
%%%%%%%%%%%%%%%%%%%%%%%%%%%%%%%%%%%%%%%%%
%%%%%%%%%%%%%%%%%%%%%%%%%%%%%%%%%%%%%%%%%


%%%%%%%%%%%%%%%%
%% Global distances
%%%%%%%%%%%%%%%%

% textlower is the height of the separation between the bottom line (which is invisible : for the notes which are very low) and the bottom of the text
\newdimen\gre@textlower
\def\gre@calculate@textlower{%
	\gre@textlower=\gre@spacebeneathtext
	%\advance\gre@textlower by \translationheight
}

% stafflinewidth is the width of a line of staff, this can vary, for example at the first line
\newdimen\gre@stafflinewidth
\def\gre@calculate@stafflinewidth{%
	\gre@stafflinewidth=\gre@linewidth 
}


% linewidth is the width of a line of a score (including the initial)
\newdimen\gre@linewidth
\def\gre@calculate@linewidth{%
	\gre@linewidth=\hsize 
}

% Messing with the staff line thickness directly is messy, so we provide the following interface to make life easier on the user:
% stafflineheight is the height of a staff line (in arbitrary units)
% = 1500 * stafflinefactor
\newdimen\gre@stafflineheight
\def\gre@calculate@stafflineheight{%
	\global\gre@stafflineheight=1500 sp
	\global\multiply\gre@stafflineheight by \gre@stafflinefactor\relax %
}

% interstafflinespace is the space between two lines of staff
% = (30000 - (stafflineheight/grefactor - 1500)) * grefactor = 31500 * grefactor - stafflineheight
\newdimen\gre@interstafflinespace
\def\gre@calculate@interstafflinespace{
	\global\gre@interstafflinespace=31500sp%
	\global\multiply\gre@interstafflinespace by \grefactor
	\global\advance\gre@interstafflinespace by -\gre@stafflineheight %
}

% a distance to help place glyphs when the lines are not their default thickness
% = (stafflineheight/grefactor - 1500sp)/2 * grefactor
\newdimen\gre@stafflinediff
\def\gre@calculate@stafflinediff{%
	\global\gre@stafflinediff = \gre@stafflineheight %
	\global\divide\gre@stafflinediff by \grefactor\relax
	\global\advance\gre@stafflinediff by -1500sp%
	\global\divide\gre@stafflinediff by 2\relax %
	\global\multiply\gre@stafflinediff by \the\grefactor %
}

% the default factor
% the stafflinefactor follows the same scale as the grefactor, i.e. a stafflinefactor corresponds to the default staff line thickness for grefactor 17, stafflinefactor 34 corresponds to the default staff line thickness for grefactor 34, etc.
\xdef\gre@stafflinefactor{17}%
% flag for whether the stafflinefactor should scale with changes of the grefactor
\xdef\gre@scale@stafflinefactor{1}

% a macro for setting the thickness of the staff lines.  This changes the stafflinefactor and then adjusts the spaces that are affected by the thicker staff lines.
\ifdefined\setstafflinethickness%
	\greerror{\protect\setstafflinethickness\space is already defined.  Check for package conflicts.}
\else
	\def\setstafflinethickness#1{%
		\xdef\gre@stafflinefactor{#1}%
		\gre@computespaces %
		%% TODO: make smarter
		\relax %
	}
\fi

\def\gresetstafflinefactor#1{
	\gre@warning{\protect\gresetstafflinefactor\space is deprecated.\MessageBreak Use \protect\setstafflinethickness\space instead.}
	\setstafflinethickness{#1}	
}

%constantglyphraise is the space between the 0 of the gragorian fonts and the effective 0 of the TeX score
\newdimen\gre@constantglyphraise 
% to calculate that, we take the bottom of the third line : it is at 200 in the fonts, and it must be at grespacelinestext + grespacebeneathtext + 2*greinterstafflinespace + 2*grestafflineheight + translationheight
\def\gre@calculate@constantglyphraise{%
  \global\gre@constantglyphraise = -22000 sp%
  \global\multiply\gre@constantglyphraise by \the\grefactor %
  \global\advance\gre@constantglyphraise by \gre@additionalbottomspace %
  \global\advance\gre@constantglyphraise by \gre@spacebeneathtext %
  \global\advance\gre@constantglyphraise by \gre@spacelinestext %
  \global\advance\gre@constantglyphraise by \gre@interstafflinespace %
  \global\advance\gre@constantglyphraise by \gre@interstafflinespace %
  \global\advance\gre@constantglyphraise by \gre@stafflineheight %
  \global\advance\gre@constantglyphraise by \gre@stafflineheight %
  \global\advance\gre@constantglyphraise by \gre@currenttranslationheight %
  % an adjustment in the case of big lines
  \global\advance\gre@constantglyphraise by \gre@stafflinediff %
  \relax %
}

%% Here is the function to compute some more vertical spaces from the basic values
\newdimen\gre@staffheight
\def\gre@calculate@staffheight{%
  \global\gre@staffheight = 4\gre@stafflineheight %
  \global\advance\gre@staffheight by 3\gre@interstafflinespace %
  %\global\multiply\gre@spacebeneathtext by \grefactor % uncomment it if you want
  % something else than 0
  \relax %
}

% A routine that simply aggregates the above global space calculating routines so we can easily update all when needed.
%% Note: It used to be that some distance calculating functions called others.  Since this can create problems with circularity if one is not careful, this is no longer the case.  Now all distance calculating functions simply calculate their respective distance.  This means that dependent distances are not necessarily recalculated when an individual distance is recalculated.  This function updates all global calculated distances and in the order needed for the dependencies.
%% Dependencies:
%% textlower: spacebeneathtext
%% linewidth: hsize
%% stafflinewidth: linewidth
%% stafflineheight: stafflinefactor & grefactor
%% interstafflinespace: stafflineheight & grefactor
%% stafflinediff: stafflineheight & grefactor
%% staffheight: stafflineheight & interstafflinespace
%% constantglyphraise: grefactor, additionalbottomspace, spacebeneathtext, spacelinestext, interstafflinespace, stafflineheight, currenttranslationheight, stafflinediff
\def\gre@computespaces{
	\gre@calculate@textlower
	\gre@calculate@linewidth
	\gre@calculate@stafflinewidth
	\gre@calculate@stafflineheight
	\gre@calculate@interstafflinespace
	\gre@calculate@stafflinediff
	\gre@calculate@staffheight
	\gre@calculate@constantglyphraise
}


%%%%%%%%%%%%%%%%%%
%% Local Distances (computed as needed)
%%%%%%%%%%%%%%%%%%

% glyphraisevalue is the value of which we must raise one glyph (that will vary with every glyph)
\newdimen\gre@glyphraisevalue 

% addedraisevalue is for the vertical episema and the puncta
\newdimen\gre@addedraisevalue

% a very useful macro : it determines the good height of a glyph : the argument is the "number" where the glyph should be : 4 for the first line, 6 for the second, etc.
% the second argument is for the cases of signs: for example if the note is on a line, the punctummora will be above, and the auctus duplex beneath. the possible values are:
%% 0: no modification
%% 1: puts the value on the interline just above if it is on a line
%% 2: puts the value on the interline just beneath if it is on a line
%% 3: case of the vertical episemus, which is not placed at the same place if the corresponding note is on a line or not
%% 4: case of the punctum mora, for the same reason
%% 5: case of the horizontal episemus under a note, that must be placed a bit lower if the note is on a line
%% 6: case of the signs above (accentus, etc.)
%% 8: case of the punctum mora of the first note of a podatus or the 2nd note of a porrectus, etc.
%% 9: case of the horizontal episemus, that must be placed a bit lower if the note is on a line
%% 10: case of the choral sign
\def\gre@calculate@glyphraisevalue#1#2{%
\gre@glyphraisevalue=0pt%
  \global\greisonaline=\number 0%
  % z is the very special case of vertical episemus on the lowest note
  \if z\grefirstcar#1\endgrefirstchar %
    \newcount\gretempdimcount
    \gretempdimcount=\number 0%
  \fi%
  \if a\grefirstcar#1\endgrefirstchar %
    \newcount\gretempdimcount
    \gretempdimcount=\number 1%
    \fi%
  \if b\grefirstcar#1\endgrefirstchar %
    \newcount\gretempdimcount
    \gretempdimcount=\number 2%
    \ifnum#2=3\relax %
    \else %
      \global\greisonaline=1 % if it is a vertical episemus, we don't care if it is on a line or not... which may cause some problems...
    \fi %
  \fi%
  \if c\grefirstcar#1\endgrefirstchar %
    \newcount\gretempdimcount
    \gretempdimcount=\number 3%
  \fi%
  \if d\grefirstcar#1\endgrefirstchar %
    \newcount\gretempdimcount
    \gretempdimcount=\number 4%
    \global\greisonaline=1 %
  \fi%
  \if e\grefirstcar#1\endgrefirstchar %
    \newcount\gretempdimcount
    \gretempdimcount=\number 5%
  \fi%
  \if f\grefirstcar#1\endgrefirstchar %
    \newcount\gretempdimcount
    \gretempdimcount=\number 6%
    \global\greisonaline=1 %
  \fi%
  \if g\grefirstcar#1\endgrefirstchar %
    \newcount\gretempdimcount
    \gretempdimcount=\number 7%
  \fi%
  \if h\grefirstcar#1\endgrefirstchar %
    \newcount\gretempdimcount
    \gretempdimcount=\number 8 %
    \global\greisonaline=1 %
  \fi%
  \if i\grefirstcar#1\endgrefirstchar %
    \newcount\gretempdimcount
    \gretempdimcount=\number 9%
  \fi%
  \if j\grefirstcar#1\endgrefirstchar %
    \newcount\gretempdimcount
    \gretempdimcount=\number 10%
    \global\greisonaline=1 %
  \fi%
  \if k\grefirstcar#1\endgrefirstchar %
    \newcount\gretempdimcount
    \gretempdimcount=\number 11%
  \fi%
  \if l\grefirstcar#1\endgrefirstchar %
    \newcount\gretempdimcount
    \gretempdimcount=\number 12%
    \global\greisonaline=1 %
  \fi%
  \if m\grefirstcar#1\endgrefirstchar %
    \newcount\gretempdimcount
    \gretempdimcount=\number 13%
  \fi%
  % n is only useful for horizontal episemus and rare signs (signs below k have m as first argument, and above have n)
  \if n\grefirstcar#1\endgrefirstchar %
    \newcount\gretempdimcount
    \gretempdimcount=\number 14%
  \fi%
  % if there is not line... we don't consider notes are on lines
  \ifnum\greremovelinescount=1\relax %
    \global\greisonaline=0 %
  \fi %
  % if the note is on a line, we change its height if necessary
  \ifcase\greisonaline\or% isonaline = 1
    \ifcase#2 %
    \or% 1
      \global\advance\gretempdimcount by 1%
    \or% 2
      \global\advance\gretempdimcount by -1%
    \or% 3
      \global\advance\gretempdimcount by -1%
    \or% 4
      \global\advance\gretempdimcount by 1%
    \or% 5
      \global\advance\gretempdimcount by -1%
    \or\or\or % 8
      \global\advance\gretempdimcount by -1%
    \or % 9
      \global\advance\gretempdimcount by 1%
    \or % 10
      \global\advance\gretempdimcount by 1%
    \or % 11
      \global\advance\gretempdimcount by 1%
    \or % 12
      \global\advance\gretempdimcount by -1%
    \fi%
  \fi%
  \global\advance\gretempdimcount by -7 %
  \global\gre@glyphraisevalue = 15750 sp %
  \global\multiply\gre@glyphraisevalue by \the\grefactor %
  \global\multiply\gre@glyphraisevalue by \the\gretempdimcount %
  \gre@addedraisevalue= 0 sp%
  \ifcase#2 % 
  \or\or\or%3: if it is a vertical episemus on a line, we shift it a bit higher, so that it's more beautiful
    \ifnum\greisonaline=1%
    \gre@addedraisevalue=7250 sp%
    \multiply\gre@addedraisevalue by \the\grefactor %
    \global\advance\gre@glyphraisevalue by \the\gre@addedraisevalue %
    \else % if it is not on a line, we shift it a bit lower
    \gre@addedraisevalue=-1380 sp%
    \multiply\gre@addedraisevalue by \the\grefactor %
    \global\advance\gre@glyphraisevalue by \the\gre@addedraisevalue %
    \fi %
  \or% 4: if it is a punctum mora on a line, we shift it a bit lower, for the same reason
    \ifnum\greisonaline=1%
      \gre@addedraisevalue=-6900 sp%
      \multiply\gre@addedraisevalue by \the\grefactor %
      \global\advance\gre@glyphraisevalue by \the\gre@addedraisevalue %
    \else % 
      \gre@addedraisevalue=-2200 sp%
      \multiply\gre@addedraisevalue by \the\grefactor %
      \global\advance\gre@glyphraisevalue by \the\gre@addedraisevalue %
    \fi%
  \or% 5: if it is a horizontal episemus under a note which is on a line, we shift it lower
    \ifnum\greisonaline=0%
      \gre@addedraisevalue=-4980 sp%
      \multiply\gre@addedraisevalue by \the\grefactor %
      \global\advance\gre@glyphraisevalue by \the\gre@addedraisevalue %
    \else % if it is under a note between two lines, we shift it higher
      \gre@addedraisevalue=4000 sp%
      \multiply\gre@addedraisevalue by \the\grefactor %
      \global\advance\gre@glyphraisevalue by \the\gre@addedraisevalue %
    \fi %
  \or% 6: if it is a sign, we put it at an arbitrary height
    \gre@addedraisevalue=20000 sp%
    \multiply\gre@addedraisevalue by \the\grefactor %
    \global\advance\gre@glyphraisevalue by \the\gre@addedraisevalue %
  \or\or% 8: if it is a punctum mora on a line, we shift it a bit lower, for the same reason
    \ifnum\greisonaline=1%
      \gre@addedraisevalue=5000 sp%
      \multiply\gre@addedraisevalue by \the\grefactor %
      \global\advance\gre@glyphraisevalue by \the\gre@addedraisevalue %
    \fi %
  \or% 9: if it is an horizontal episemus not on a line, we put it a bit lower
    \ifnum\greisonaline=1%
      \gre@addedraisevalue=-5500 sp%
    \else %
      \gre@addedraisevalue=3000 sp%
    \fi %
    \multiply\gre@addedraisevalue by \the\grefactor %
    \global\advance\gre@glyphraisevalue by \the\gre@addedraisevalue %
  \or% 10: if it is a low choral sign, we shift it a bit lower, of a user-defined value
    \global\advance\gre@glyphraisevalue by -\the\gre@lowchoralsignshift %
  \or% 11: if it is a high choral sign, we shift it a bit lower, of a user-defined value
    \ifnum\greisonaline=1%
      \global\advance\gre@glyphraisevalue by -\the\gre@highchoralsignshift %
    \else %
      \global\advance\gre@glyphraisevalue by -\the\gre@lowchoralsignshift %
    \fi %
  \or% 12: if it is a low choral sign that is lower than the note, we shift it a bit higher
    \ifnum\greisonaline=1%
      \global\advance\gre@glyphraisevalue by -\the\gre@highchoralsignshift %
    \else %
      \global\advance\gre@glyphraisevalue by -\the\gre@lowchoralsignshift %
    \fi %
  \or% 12: if it is the brace above the bars, we shift it to a user-defined value
      \global\advance\gre@glyphraisevalue by -\the\gre@braceshift %
  \fi%
  \global\advance\gre@glyphraisevalue by \the\gre@constantglyphraise %
}

% two dimensions for the additionalspaces
\newdimen\gre@additionalbottomspace
\newdimen\gre@additionaltopspace

% same arguments as grenewlinewithspace
\def\gre@calculate@additionalspaces#1#2{%
  \ifcase#1\relax %
    \global\gre@additionalbottomspace=0 sp%
  \or % case 1
    \global\gre@additionalbottomspace=0 sp%
    % here we don't add any space... it's just in case...
  \or % case 2
    \global\gre@additionaltopspace=15000 sp%
    \global\multiply\gre@additionaltopspace by \the\grefactor %
  \or % case 3
    \global\gre@additionaltopspace=30000 sp%
    \global\multiply\gre@additionaltopspace by \the\grefactor %
  \fi %
  \ifcase#2\relax %
    % case 0
    \global\gre@additionalbottomspace=0 sp%
  \or % case 1
    \global\gre@additionalbottomspace=0 sp%
  \or % case 2
    \global\gre@additionalbottomspace=15000 sp%
    \global\multiply\gre@additionalbottomspace by \the\grefactor %
  \or % case 3
    \global\gre@additionalbottomspace=30000 sp%
    \global\multiply\gre@additionalbottomspace by \the\grefactor %
  \or % case 4
    \global\gre@additionalbottomspace=45000 sp%
    \global\multiply\gre@additionalbottomspace by \the\grefactor %
  \fi %
  \gregeneratelines %
  \gre@calculate@constantglyphraise %
  \relax %
}

%% macros for additional bottom space for the first line

% #1 is 1, 2 or 3, with the same signification as in grenewlinewithspace
\def\grefirstlinebottomspace#1#2{%
  \ifcase#1\relax %
    % case 0
    \global\gre@additionalbottomspace=0 sp%
  \or % case 1
    \global\gre@additionalbottomspace=0 sp%
  \or % case 2
    \global\gre@additionalbottomspace=15000 sp%
    \global\multiply\gre@additionalbottomspace by \the\grefactor %
  \or % case 3
    \global\gre@additionalbottomspace=30000 sp%
    \global\multiply\gre@additionalbottomspace by \the\grefactor %
  \fi %
  \ifnum#2=1\relax %
    \greaddtranslationspace %
  \else %
    \greremovetranslationspace %
  \fi %
  \gregeneratelines %
  \gre@calculate@constantglyphraise %
  \relax %
}

%% macro that typesets the text of the syllable, and sets textaligncenter to the middle of the middle letters, it is needed because we align the note (often the middle of the note) with the middle of the middle letters
%% third argument is 0 if it's the current syllable, 1 if it's the alignment of the following one
%% warning: gretextaligncenter is the width from the beginning of the letters to the middle of the middle letters
%% warning: value is approximative when a ligature appears

\newdimen\gre@textaligncenter

\def\gre@calculate@textaligncenter#1#2#3{%
  \ifnum#3=0\relax%
    \grewidthof{\grefixedtextformat{#1#2}}%
  \else %
    \grewidthof{\grefixednexttextformat{#1#2}}%
  \fi %
  \global\gre@textaligncenter=\the\gre@tempwidth %
  \ifnum#3=0\relax%
    \grewidthof{\grefixedtextformat{#2}}%
  \else %
    \grewidthof{\grefixednexttextformat{#2}}%
  \fi %
  \divide\gre@tempwidth by 2 %
  \global\advance\gre@textaligncenter by -\the\gre@tempwidth%
  \relax%
}

% a dimen that will contain the difference between the end of the text and the end of the notes for the previous syllable (if we are in the same word) : positive if notes go further than text. We will use it for space adjustment between syllables of the same word
\newdimen\gre@enddifference

% a dimen that will contain the enddifference of the previous glyph
\newdimen\gre@previousenddifference

% macro to set enddifference (defined above) to \wd\GreSyllablenotes - (\wd\GreSyllabletext - textaligncenter) - notesaligncenter
% enddifference will be positive if text go further than the notes, and negative in the other case
% arguments are :
% #1: \wd\GreSyllablenotes : the total width of the notes
% #2: \wd\GreSyllabletext : the total width of the text
% #3: textaligncenter (defined above)
% #4: notesaligncenter (defined above too)
% #5: if we have to set previousenddifference or not
\def\gre@calculate@enddifference#1#2#3#4#5{%
  \ifcase#5\or %
    \global\gre@previousenddifference=\the\gre@enddifference %
  \fi %
  \global\gre@enddifference=#1%
  \global\advance\gre@enddifference by -#2%
  \global\advance\gre@enddifference by #3%
  \global\advance\gre@enddifference by -#4%
  \relax%
}

% temporary value for space for the translation, beneath the text
\newdimen\gre@currenttranslationheight

% macro to tell gregorio to set space for the translation
\def\greaddtranslationspace{%
  \global\gre@currenttranslationheight=\gre@translationheight %
  \global\gre@textlower=\gre@spacebeneathtext %
  \global\advance\gre@textlower by \gre@translationheight %
  \gregeneratelines %
  \gre@calculate@constantglyphraise %
  \relax %
}

\def\greremovetranslationspace{%
  \global\gre@currenttranslationheight=0 sp%
  \global\gre@textlower=\gre@spacebeneathtext %
  \gregeneratelines %
  \gre@calculate@constantglyphraise %
  \relax %
}

%nextbegindifference is the begindifference of the next syllable
\newskip\gre@nextbegindifference

% macro to set nextbegindifference
%% 1 : the first letters of the next syllable
%% 2 : the middle letters of the next syllable
%% 3 : the end letters of the next syllable
%% 4 : the type of notes alignment
\def\gre@calculate@nextbegindifference#1#2#3#4{%
  %to prevent the pollution of the normal values, we stock them into a temp value
  \newskip\gre@tempdimskip
  \gre@tempdimskip=\gre@textaligncenter %
  \gre@calculate@textaligncenter{#1}{#2}{1}%
  \global\gre@nextbegindifference=-\gre@textaligncenter %
  \global\gre@textaligncenter=\gre@tempdimskip %
  \newskip\gre@tempdimskip
  \gre@tempdimskip=\gre@notesaligncenter %
  \grefindnextnotesaligncenter{#4}% idem
  \global\advance\gre@nextbegindifference by \the\gre@notesaligncenter %
  \global\gre@notesaligncenter=\gre@tempdimskip %
  \relax %
}

%The distance from the baseline of the line to the baseline of the annotations
\newdimen\gre@aboveinitialfirstraise
\newdimen\gre@aboveinitialsecondraise
% When text is placed in the annotation boxes these dimensions are initialized with values based on the contents and the user parameters
%This function sets the true raises of the two lines above the inital (it has to be called just as the boxes are placed in order to make sure that the values are all correct)
\def\gre@calculate@aboveinitialraise{%
  \global\advance\gre@aboveinitialfirstraise by \gre@staffheight %
  \global\advance\gre@aboveinitialfirstraise by \gre@spacebeneathtext %
  \global\advance\gre@aboveinitialfirstraise by \gre@currenttranslationheight %
  \global\advance\gre@aboveinitialfirstraise by \gre@spacelinestext %
  \global\advance\gre@aboveinitialfirstraise by \gre@additionalbottomspace %
  \global\advance\gre@aboveinitialsecondraise by \gre@aboveinitialfirstraise %
  \relax %
}



%%%%%%%%%%%%%%%%%%
%% other spaces calculated elsewhere
%%%%%%%%%%%%%%%%%%

% These distances don't have independent functions which calculate their value, generally because their calculation is distributed over multiple events.

% begindifference is the difference between the begginning of the text and the beginning of the notes. Warning : it can be negative.
\newdimen\gre@begindifference

% the width of the clef
\newdimen\gre@clefwidth

% the width of the last glyph
\newdimen\gre@lastglyphwidth

% notes align center is the point of alignment for the notes
\newdimen\gre@notesaligncenter

%this dimention is the additional space that we have to add to the localleftbox sometimes. For now it is used only for the initials on two lines
\newdimen\gre@additionalleftspace

% the calculated width of the initial (may be actual width of letter or be forced wider under certain conditions)
\newdimen\gre@initialwidth
\gre@initialwidth= 0 pt

\newdimen\gre@currentabovelinestextheight
\gre@currentabovelinestextheight = 0pt

%% TODO: try localizing all temporary variables
% temporary spaces used in calculations
%\newdimen\gre@tempwidth
%\newdimen\gre@tempdimsignwidth
%\newdimen\gre@tempdimtwo
%\newdimen\gre@tempdim
%\newskip\gre@tempdimskip % couldn't we use another existing temp* ? maybe not
%\newskip\gre@skipone

%%%%%%%%%%%%%%%%%%%%%%%%%%%%
%% dimension changing macros
%%%%%%%%%%%%%%%%%%%%%%%%%%%%

%% This macro changes one dim (#1) to the value #2, with the current factor.
%% or use the next macro.
\def\grechangedim#1#2{%
  \grechangedimatfactor{#1}{#2}{\grefactor}%
  \relax %
}

%% This macro changes one dimen (#1) to the value #2, at the factor specified
%% in #3.
\def\grechangedimatfactor#1#2#3{%
  #1=#2%
  \ifnum #3=\grefactor\else %
    \divide #1 by \number #3%
    \multiply #1 by \grefactor %
  \fi %
  \gre@computespaces %
  \relax %
}

%%%%%%%%%%%%%%%%%%%%%%%%%%%%%
% space configuration loading
%%%%%%%%%%%%%%%%%%%%%%%%%%%%%

\def\GreLoadSpaceConf#1{%
  \input gsp-#1.tex\relax %
  \gre@changedimenfactor{\greconffactor}{\grefactor} %
  \gre@computespaces %
  \relax %
}


% We input the default config, for everything to work fine it has to be done after the gregoriotex package is completely loaded (so that all functions are well defined).  As a result we delay the loading to the beginning of the document.
\AtBeginDocument{\GreLoadSpaceConf{default}}


%%%%%%%%%%%%%
%% Rescaling dimensions (for when \grefactor changes)
%%%%%%%%%%%%%

%% an aux function adapting the value #1 from the factor #2 to the factor #3
\def\gre@changeonedimenfactor#1#2#3{%
  \global\divide #1 by \number #2%
  \global\multiply #1 by \number #3%
  \relax %
}
%% this function changes all the values of the spaces (vertical and horizontal) from one factor to another
%% simply by dividing them by the old factor, and multiplying them by the new one.
% #1 is the old grefactor, #2 is the new one
\def\gre@changedimenfactor#1#2{%
	\gre@changeonedimenfactor{\gre@additionallineswidth}{#1}{#2}%
	\gre@changeonedimenfactor{\gre@additionalcustoslineswidth}{#1}{#2}%
	\gre@changeonedimenfactor{\gre@zerowidthspace}{#1}{#2}%
	\gre@changeonedimenfactor{\gre@interglyphspace}{#1}{#2}%
	\gre@changeonedimenfactor{\gre@alterationspace}{#1}{#2}%
	\gre@changeonedimenfactor{\gre@clefflatspace}{#1}{#2}%
	\gre@changeonedimenfactor{\gre@beforechoralsignspace}{#1}{#2}%
	\gre@changeonedimenfactor{\gre@beforealterationspace}{#1}{#2}%
	\gre@changeonedimenfactor{\gre@interelementspace}{#1}{#2}%
	\gre@changeonedimenfactor{\gre@largerspace}{#1}{#2}%
	\gre@changeonedimenfactor{\gre@glyphspace}{#1}{#2}%
	\gre@changeonedimenfactor{\gre@intersyllablespace}{#1}{#2}%
	\gre@changeonedimenfactor{\gre@spacebeforecusto}{#1}{#2}%
	\gre@changeonedimenfactor{\gre@spacebeforesigns}{#1}{#2}%
	\gre@changeonedimenfactor{\gre@spaceaftersigns}{#1}{#2}%
	\gre@changeonedimenfactor{\gre@spaceafterlineclef}{#1}{#2}%
	\gre@changeonedimenfactor{\gre@interwordspacenotes}{#1}{#2}%
	\gre@changeonedimenfactor{\gre@interwordspacenotestext}{#1}{#2}%
	\gre@changeonedimenfactor{\gre@interwordspacetextnotes}{#1}{#2}%
	\gre@changeonedimenfactor{\gre@interwordspacetext}{#1}{#2}%
	\gre@changeonedimenfactor{\gre@bitrivirspace}{#1}{#2}%
	\gre@changeonedimenfactor{\gre@bitristrospace}{#1}{#2}%
	\gre@changeonedimenfactor{\gre@punctuminclinatumshift}{#1}{#2}%
	\gre@changeonedimenfactor{\gre@beforepunctainclinatashift}{#1}{#2}%
	\gre@changeonedimenfactor{\gre@punctuminclinatumanddebilisshift}{#1}{#2}%
	\gre@changeonedimenfactor{\gre@punctuminclinatumdebilisshift}{#1}{#2}%
	\gre@changeonedimenfactor{\gre@punctuminclinatumbigshift}{#1}{#2}%
	\gre@changeonedimenfactor{\gre@punctuminclinatummaxshift}{#1}{#2}%
	\gre@changeonedimenfactor{\gre@spacearoundsmallbar}{#1}{#2}%
	\gre@changeonedimenfactor{\gre@spacearoundminor}{#1}{#2}%
	\gre@changeonedimenfactor{\gre@spacearoundmaior}{#1}{#2}%
	\gre@changeonedimenfactor{\gre@spacearoundfinalis}{#1}{#2}%
	\gre@changeonedimenfactor{\gre@spacebeforefinalfinalis}{#1}{#2}%
	\gre@changeonedimenfactor{\gre@spacearoundclefbars}{#1}{#2}%
	\gre@changeonedimenfactor{\gre@textbartextspace}{#1}{#2}%
	\gre@changeonedimenfactor{\gre@notebarspace}{#1}{#2}%
	\gre@changeonedimenfactor{\gre@maximumspacewithoutdash}{#1}{#2}%
	\gre@changeonedimenfactor{\gre@afterclefnospace}{#1}{#2}%
	\gre@changeonedimenfactor{\gre@afterinitialshift}{#1}{#2}%
	\gre@changeonedimenfactor{\gre@beforeinitialshift}{#1}{#2}%
	\gre@changeonedimenfactor{\gre@minimalspaceatlinebeginning}{#1}{#2}%
	\gre@changeonedimenfactor{\gre@manualinitialwidth}{#1}{#2}%
	\gre@changeonedimenfactor{\gre@aboveinitialseparation}{#1}{#2}%
	\gre@changeonedimenfactor{\gre@noclefspace}{#1}{#2}%
	\gre@changeonedimenfactor{\gre@clefchangespace}{#1}{#2}%
	\gre@changeonedimenfactor{\gre@clivisalignmentmin}{#1}{#2}%
	\gre@changeonedimenfactor{\gre@abovesignsspace}{#1}{#2}%
	\gre@changeonedimenfactor{\gre@belowsignsspace}{#1}{#2}%
	\gre@changeonedimenfactor{\gre@lowchoralsignshift}{#1}{#2}%
	\gre@changeonedimenfactor{\gre@highchoralsignshift}{#1}{#2}%
	\gre@changeonedimenfactor{\gre@translationheight}{#1}{#2}%
	\gre@changeonedimenfactor{\gre@spaceabovelines}{#1}{#2}%
	\gre@changeonedimenfactor{\gre@spacelinestext}{#1}{#2}%
	\gre@changeonedimenfactor{\gre@spacebeneathtext}{#1}{#2}%
	\gre@changeonedimenfactor{\gre@abovelinestextraise}{#1}{#2}%
	\gre@changeonedimenfactor{\gre@abovelinestextheight}{#1}{#2}%
	\gre@changeonedimenfactor{\gre@braceshift}{#1}{#2}%
	\gre@changeonedimenfactor{\gre@curlybraceaccentusshift}{#1}{#2}%
	\ifnum\gre@scale@stafflinefactor=1\relax%
		\newcount\tempcount
		\tempcount = \gre@stafflinefactor
		\multiply\tempcount by #2\relax%
		\divide\tempcount by #1\relax
		\xdef\gre@stafflinefactor{\the\tempcount}
	\fi
	\gre@computespaces
	\relax %
}

%%%%%%%%%%%%%%%%%%%%%%%%%%%%%%%
%  Some Macros for changing the spacing around the initial
%%%%%%%%%%%%%%%%%%%%%%%%%%%%%%%

%Seeing as these are the distances that people will want to change the most often,
%we give them their own set of macros to make that easier.


\def\GreSetAboveInitialSeparation#1{
  \grechangedim{\gre@aboveinitialseparation}{#1}%
  \relax %
}

\let\setaboveinitialseparation\GreSetAboveInitialSeparation

\def\GreSetSpaceAfterInitial#1{%
  \ifnum\grebiginitial=0\relax%
    \greinitialformat{\global\grechangedim{\gre@afterinitialshift}{#1}}%
  \else%
    \grebiginitialformat{\global\grechangedim{\gre@afterinitialshift}{#1}}%
  \fi%
  \relax %
}

\let\setspaceafterinitial\GreSetSpaceAfterInitial

\def\GreSetSpaceBeforeInitial#1{%
  \ifnum\grebiginitial=0\relax%
    \greinitialformat{\global\grechangedim{\gre@beforeinitialshift}{#1}}%
  \else%
    \grebiginitialformat{\global\grechangedim{\gre@beforeinitialshift}{#1}}%
  \fi%
  \relax %
}

\let\setspacebeforeinitial\GreSetSpaceBeforeInitial

\def\setinitialspacing#1#2#3{
  \grechangedim{\gre@beforeinitialshift}{#1}%
  \ifnum\grebiginitial=0\relax %
    \greinitialformat{\global\grechangedim{\gre@manualinitialwidth}{#2}}%
  \else%
    \grebiginitialformat{\global\grechangedim{\gre@manualinitialwidth}{#2}}%
  \fi%
  \grechangedim{\gre@afterinitialshift}{#3}%
  \relax%
}

