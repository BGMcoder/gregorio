%GregorioTeX file.
%Copyright (C) 2007-2009 Elie Roux <elie.roux@telecom-bretagne.eu>
%
%This program is free software: you can redistribute it and/or modify
%it under the terms of the GNU General Public License as published by
%the Free Software Foundation, either version 3 of the License, or
%(at your option) any later version.
%
%This program is distributed in the hope that it will be useful,
%but WITHOUT ANY WARRANTY; without even the implied warranty of
%MERCHANTABILITY or FITNESS FOR A PARTICULAR PURPOSE.  See the
%GNU General Public License for more details.
%
%You should have received a copy of the GNU General Public License
%along with this program.  If not, see <http://www.gnu.org/licenses/>.

% this file contains definitions of spaces

%%%%%%%%%%%%%%%%%%%%%%%%%%%%%%%%%%%%%%%
%% macros for the typesetting of spaces
%%%%%%%%%%%%%%%%%%%%%%%%%%%%%%%%%%%%%%%

%% All the following values correspond to a grefactor of 1,
%% so in most cases it will be much smaller that what you will finally
%% get with an average factor (17 is the default value for the factor).
%% We multiply all the columns by the factor when the factor changes,
%% and it changes at the beginning of the first score by default.

%%%%%%%%%%%%%%%%%%%%
%% horizontal spaces
%%%%%%%%%%%%%%%%%%%%

% null space
\newskip\zerowidthspace
\zerowidthspace=0pt plus 0pt minus 0pt

% space between glyphs in the same element
\newskip\interglyphspace
\interglyphspace = 7600 sp plus 400 sp minus 400 sp%

% space between an alteration (flat or natural) and the next glyph
\newskip\alterationspace
\alterationspace = 8500 sp plus 1400 sp minus 500 sp%

% space between a clef and a flat (for clefs with flat)
\newskip\clefflatspace
\clefflatspace = 6000 sp plus 700 sp minus 700 sp%

% negative space, difference between the normal space between two notes and the space between a note and a flat
\newskip\beforealterationspace
\beforealterationspace = -36000 sp plus 1200 sp minus 1200 sp%

% space between elements
\newskip\interelementspace
\interelementspace = 7600 sp plus 200 sp minus 400 sp%

% larger space between elements
\newskip\largerspace
\largerspace = 12000 sp plus 2000 sp minus 1000 sp%

% space between elements which has the size of a note
\newskip\glyphspace
\glyphspace = 24000 sp plus 2000 sp minus 2000 sp%

% minimum space between two notes of different syllables
\newskip\intersyllablespace
\intersyllablespace= 28000 sp plus 35000 sp minus 0 sp%

% space before custo
\newskip\spacebeforecusto
\spacebeforecusto = 20000 sp plus 35000 sp minus 7000 sp%

% space before punctum mora and augmentum duplex
\newskip\spacebeforesigns
\spacebeforesigns= 6000 sp plus 500 sp minus 500 sp%

% space after punctum mora and augmentum duplex
\newskip\spaceaftersigns
\spaceaftersigns= 9000 sp plus 900 sp minus 900 sp%

% space after a clef at the beginning of a line
\newskip\spaceafterlineclef
\spaceafterlineclef = 30000 sp plus 16000 sp minus 1500 sp%

% space after at the end of a word when the last written symbol is a note and the first is a note
\newskip\interwordspacenotes
\interwordspacenotes = 32000 sp plus 9600 sp minus 6000 sp%

% space after at the end of a word when the last written symbol is a note and the first is text
\newskip\interwordspacenotestext
\interwordspacenotestext = 30000 sp plus 30000 sp minus 8000 sp%

% space after at the end of a word when the last written symbol is text and the first is a note
\newskip\interwordspacetextnotes
\interwordspacetextnotes = 30000 sp plus 30000 sp minus 8000 sp%

% space after at the end of a word when the last written symbol is text and the first is text
\newskip\interwordspacetext
\interwordspacetext = 25000 sp plus 45000 sp minus 8000 sp%

% space between notes of a bivirga or trivirga
\newskip\bitrivirspace
\bitrivirspace = 7600 sp plus 200 sp minus 600 sp%

% space between notes of a bistropha or tristropha
\newskip\bitristrospace
\bitristrospace = 7600 sp plus 200 sp minus 600 sp%

% space between two punctum inclinatum
\newskip\punctuminclinatumshift
\punctuminclinatumshift= -2350 sp plus 100 sp minus 100 sp%

% space before puncta inclinata
\newskip\beforepunctainclinatashift
\beforepunctainclinatashift= 5800 sp plus 800 sp minus 500 sp%

% space between a punctum inclinatum and a punctum inclinatum deminutus
\newskip\punctuminclinatumanddebilisshift
\punctuminclinatumanddebilisshift= -1700 sp plus 100 sp minus 100 sp%

% space between two punctum inclinatum deminutus
\newskip\punctuminclinatumdebilisshift
\punctuminclinatumdebilisshift= -100 sp plus 100 sp minus 100 sp%

% space between puncta inclinata, larger ambitus (range=3rd)
\newskip\punctuminclinatumbigshift %
\punctuminclinatumbigshift= 9800 sp plus 100 sp minus 100 sp%

% space between puncta inclinata, larger ambitus (range=4th -or more?-)
\newskip\punctuminclinatummaxshift %
\punctuminclinatummaxshift= 19600 sp plus 100 sp minus 100 sp%

% space for the bars (inside syllables)
%first for virgula and divisio minima
\newskip\spacearoundsmallbar
\spacearoundsmallbar = 20000 sp plus 10000 sp minus 6000 sp%

%then divisio minor
\newskip\spacearoundminor
\spacearoundminor = 20000 sp plus 15000 sp minus 6000 sp%

%divisio major
\newskip\spacearoundmaior
\spacearoundmaior = 20000 sp plus 15000 sp minus 6000 sp%

%divisio finalis
\newskip\spacearoundfinalis
\spacearoundfinalis = 20000 sp plus 10000 sp minus 6000 sp%

%a special space for finalis, for when it is the last glyph
\newskip\spacebeforefinalfinalis
\spacebeforefinalfinalis= 32000 sp plus 8000 sp minus 16000 sp%

% the space that will appear around bars that are preceded by a custo and followed by a key.
% well... actually it's the difference between the normal space around bars and the space described previously.
\newskip\spacearoundclefbars
\spacearoundclefbars= 4000 sp plus 500 sp minus 100 sp%

% space between the text and the text of the bar
\newskip\textbartextspace
\textbartextspace = 27000 sp plus 15000 sp minus 5400 sp%

% minimal space between a note and a bar
\newskip\notebarspace
\notebarspace = 35000 sp plus 30000 sp minus 3100 sp%

% maximal space between two syllables for which we consider a dash is not needed
\newdimen\maximumspacewithoutdash
\maximumspacewithoutdash = 2200 sp%

% an extensible space for the beginning of lines
\newskip\afterclefnospace
\afterclefnospace = 0 pt plus 30000 sp minus 0 pt%

% width of the additional lines, used only for the custos (maybe should depend on the width of the custo...)
% the width is the one for the custos at end of lines, the line for custos in the middle of a score is the same
% multiplied by 2.
\newdimen\additionallineswidth
\additionallineswidth = 17000 sp%

% space between the initial and the beginning of the score
\newskip\afterinitialshift
\afterinitialshift=0.6 em plus 0em minus 0em

% space before the initial and the beginning of the score
\newskip\beforeinitialshift
\afterinitialshift=0.6 em plus 0em minus 0em

% this space is the one between the bottom of the first anotation line and the top
% of the second anotation line (above the initial)

\newdimen\aboveinitialseparation
\aboveinitialseparation = 0.5mm

\def\setaboveinitialseparation#1{
\aboveinitialseparation=#1%
\relax %
}

\def\setspaceafterinitial#1{%
\afterinitialshift=#1 %
\relax %
}

\def\setspacebeforeinitial#1{%
\beforeinitialshift=#1 %
\relax %
}

%%%%%%%%%%%%%%%%%%
%% vertical spaces
%%%%%%%%%%%%%%%%%%

% \stafflineheight is the height of a staff line
\newdimen\stafflineheight
\stafflineheight=1500 sp%

% \interstafflinespace is the space between two lines of staff
\newdimen\interstafflinespace
\interstafflinespace=30000 sp%

%the space above the lines
\newskip\spaceabovelines
\spaceabovelines = 50000 sp plus 40000 sp minus 10000 sp%

%the space between the lines and the bottom of the text
\newskip\spacelinestext
\spacelinestext = 66500 sp plus 0 sp minus 0 sp%

%the space beneath the text
\newskip\spacebeneathtext
\global\spacebeneathtext = 0 sp plus 0 sp minus 0 sp%

%%% the following values are computed from the others, after some calculus

%constantglyphraise is the space between the 0 of the gragorian fonts and the effective 0 of the TeX score
\newdimen\constantglyphraise %

% \staffheight is the total height of the staff : that is to say the four written lines
\newdimen\staffheight 

% \stafflinewidth is the width of a line of staff, this can vary, for example at the first line
\newdimen\stafflinewidth

% space for the clef changes
\newskip\clefchangespace

%%%%%%%%%%%%%%%%%%%%%%%%%%%%
%% dimension changing macros
%%%%%%%%%%%%%%%%%%%%%%%%%%%%

%% This macro changes one dim (#1) to the value #2, with the current factor.
%% If the factor is 0, it takes the default value (17), but warning: this
%% means that you have to set your personal grefactor before changing the values,
%% or use the next macro.
\def\changegredim#1#2{%
\ifnum\grefactor=0\relax %
\changegredimatfactor{#1}{#2}{17}%
\else %
\changegredimatfactor{#1}{#2}{\grefactor}%
\fi %
\relax %
}

%% This macro changes one dimen (#1) to the value #2, at the factor specified
%% in #3.
\def\changegredimatfactor#1#2#3{%
#1=#2%
\ifnum #3=\grefactor\else %
\divide #1 by \number #3%
\ifnum\grefactor=0\else % 
% if grefactor = 0 it means that we must consider it's 1,
% as we are before the beginning of the first score.
\multiply #1 by \grefactor %
\fi %
\fi %
\setgreverticalspaces %
\relax %
}

%% Here is the function to compute some more vertical spaces from the basic values
\def\setgreverticalspaces{%
\staffheight=4\stafflineheight %
\advance\staffheight by 3\interstafflinespace %
%\global\multiply\spacebeneathtext by \grefactor % uncomment it if you want
% something else than 0
\global\textlower=\spacebeneathtext %
\global\calculateconstantglyphraise %
\relax %
}

%% an aux function adapting the value #1 from the factor #2 to the factor #3
\def\changeonedimenfactor#1#2#3{%
\global\divide #1 by \number #2%
\global\multiply #1 by \number #3%
\relax %
}
%% this function changes all the values of the spaces (vertical and horizontal) from one factor to another
%% simply by dividing them by the old factor, and multiplying them by the new one.
% #1 is the old grefactor, #2 is the new one
\def\changedimenfactor#1#2{%
\changeonedimenfactor{\stafflineheight}{#1}{#2}%
\changeonedimenfactor{\interstafflinespace}{#1}{#2}%
\changeonedimenfactor{\spaceabovelines}{#1}{#2}%
\changeonedimenfactor{\spacelinestext}{#1}{#2}%
\changeonedimenfactor{\spacebeneathtext}{#1}{#2}%
\changeonedimenfactor{\interglyphspace}{#1}{#2}%
\changeonedimenfactor{\alterationspace}{#1}{#2}%
\changeonedimenfactor{\clefflatspace}{#1}{#2}%
\changeonedimenfactor{\beforealterationspace}{#1}{#2}%
\changeonedimenfactor{\interelementspace}{#1}{#2}%
\changeonedimenfactor{\largerspace}{#1}{#2}%
\changeonedimenfactor{\glyphspace}{#1}{#2}%
\changeonedimenfactor{\intersyllablespace}{#1}{#2}%
\changeonedimenfactor{\spacebeforecusto}{#1}{#2}%
\changeonedimenfactor{\spacebeforesigns}{#1}{#2}%
\changeonedimenfactor{\spaceaftersigns}{#1}{#2}%
\changeonedimenfactor{\spaceafterlineclef}{#1}{#2}%
\changeonedimenfactor{\interwordspacenotes}{#1}{#2}%
\changeonedimenfactor{\interwordspacenotestext}{#1}{#2}%
\changeonedimenfactor{\interwordspacetextnotes}{#1}{#2}%
\changeonedimenfactor{\interwordspacetext}{#1}{#2}%
\changeonedimenfactor{\bitrivirspace}{#1}{#2}%
\changeonedimenfactor{\bitristrospace}{#1}{#2}%
\changeonedimenfactor{\punctuminclinatumshift}{#1}{#2}%
\changeonedimenfactor{\beforepunctainclinatashift}{#1}{#2}%
\changeonedimenfactor{\punctuminclinatumanddebilisshift}{#1}{#2}%
\changeonedimenfactor{\punctuminclinatumdebilisshift}{#1}{#2}%
\changeonedimenfactor{\punctuminclinatumbigshift}{#1}{#2}%
\changeonedimenfactor{\punctuminclinatummaxshift}{#1}{#2}%
\changeonedimenfactor{\spacearoundsmallbar}{#1}{#2}%
\changeonedimenfactor{\spacearoundminor}{#1}{#2}%
\changeonedimenfactor{\spacearoundmaior}{#1}{#2}%
\changeonedimenfactor{\spacearoundfinalis}{#1}{#2}%
\changeonedimenfactor{\spacebeforefinalfinalis}{#1}{#2}%
\changeonedimenfactor{\spacearoundclefbars}{#1}{#2}%
\changeonedimenfactor{\textbartextspace}{#1}{#2}%
\changeonedimenfactor{\notebarspace}{#1}{#2}%
\changeonedimenfactor{\maximumspacewithoutdash}{#1}{#2}%
\changeonedimenfactor{\afterclefnospace}{#1}{#2}%
\changeonedimenfactor{\additionallineswidth}{#1}{#2}%
\relax %
}
