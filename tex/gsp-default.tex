%GregorioTeX file.
%Copyright (C) 2007-2010 Elie Roux <elie.roux@telecom-bretagne.eu>
%
%This program is free software: you can redistribute it and/or modify
%it under the terms of the GNU General Public License as published by
%the Free Software Foundation, either version 3 of the License, or
%(at your option) any later version.
%
%This program is distributed in the hope that it will be useful,
%but WITHOUT ANY WARRANTY; without even the implied warranty of
%MERCHANTABILITY or FITNESS FOR A PARTICULAR PURPOSE.  See the
%GNU General Public License for more details.
%
%You should have received a copy of the GNU General Public License
%along with this program.  If not, see <http://www.gnu.org/licenses/>.

% this file contains definitions of spaces and penalties

%% First, the penalties

% penalty at the end of a syllable which is the end of a word
\xdef\greendofwordpenalty{-100}
% penalty at the end of a syllable which is not the end of a word
\xdef\greendofsyllablepenalty{-50}
% penalty at the end of a syllable which is just a bar, with something printed
% under it
\xdef\greendafterbarpenalty{-200}
% penalty right after a bar with nothing printed
\xdef\greendafterbaraltpenalty{-200}
% penalty at the end of a breakable neumatic element (typically at a space
% between elements)
\xdef\greendofelementpenalty{-50}
% hyphenpenalty will be used in discretionaries, in Gregorio this is used for
% a bar with clef change for example. It also set \exhyphenpenalty. It should
% be close to \greendafterbarpenalty
\xdef\grehyphenpenalty{-200}

% broken penalty is the vertical penalty inserted after a break on a clef change
% I'm not sure it should be set, but it might be useful...
\xdef\grebrokenpenalty{0}

%% Then, some fine tunings
\def\grelooseness{\looseness}
\def\gretolerance{9000}
% Workaround for bug 842 (http://tracker.luatex.org/view.php?id=842)
% see http://tug.org/pipermail/luatex/2013-July/004516.html
\ifnum\the\luatexversion < 78\relax %
  \global\def\grepretolerance{-1}
\else %
  \global\def\grepretolerance{\pretolerance}
\fi %
\def\greemergencystretch{\emergencystretch}
% By default, we don't care if a line of score is alone on the page,
% if you think it is bad, you can modify the two following values. Assigning
% 10000 to them will prevent all orphaned lines (this will certainly improve
% some spacings a lot).
\def\grewidowpenalty{0}
\def\greclubpenalty{0}

%% And finally, all the spacings:

%% All the following values correspond to a grefactor of 1,
%% so in most cases it will be much smaller that what you will finally
%% get with an average factor (17 is the default value for the factor).
%% We multiply all the columns by the factor when the factor changes,
%% and it changes at the beginning of the first score by default.


% the additional width of the additional lines (compared to the width of the glyph they're associated with)
\gre@additionallineswidth = 16000sp
% null space
\gre@zerowidthspace=0pt plus 0pt minus 0pt
% space between glyphs in the same element
\gre@interglyphspace = 7600 sp plus 400 sp minus 400 sp
% space between an alteration (flat or natural) and the next glyph
\gre@alterationspace = 8500 sp plus 1400 sp minus 500 sp
% space between a clef and a flat (for clefs with flat)
\gre@clefflatspace = 6000 sp plus 700 sp minus 700 sp
% space before a choral sign
\gre@beforechoralsignspace = 5000 sp plus 700 sp minus 700 sp
% negative space, difference between the normal space between two notes and the space between a note and a flat
\gre@beforealterationspace = -36000 sp plus 1200 sp minus 1200 sp
% space between elements
\gre@interelementspace = 7600 sp plus 200 sp minus 400 sp
% larger space between elements
\gre@largerspace = 12000 sp plus 2000 sp minus 1000 sp
% space between elements which has the size of a note
\gre@glyphspace = 24000 sp plus 2000 sp minus 2000 sp
% minimum space between two notes of different syllables
\gre@intersyllablespace= 28000 sp plus 35000 sp minus 0 sp
% space before custo
\gre@spacebeforecusto = 20000 sp plus 35000 sp minus 7000 sp
% space before punctum mora and augmentum duplex
\gre@spacebeforesigns= 6000 sp plus 500 sp minus 500 sp
% space after punctum mora and augmentum duplex
\gre@spaceaftersigns= 9000 sp plus 900 sp minus 900 sp
% space after a clef at the beginning of a line
\gre@spaceafterlineclef = 30000 sp plus 16000 sp minus 1500 sp
% space after at the end of a word when the last written symbol is a note and the first is a note
\gre@interwordspacenotes = 32000 sp plus 9600 sp minus 6000 sp
% space after at the end of a word when the last written symbol is a note and the first is text
\gre@interwordspacenotestext = 30000 sp plus 30000 sp minus 8000 sp
% space after at the end of a word when the last written symbol is text and the first is a note
\gre@interwordspacetextnotes = 30000 sp plus 30000 sp minus 8000 sp
% space after at the end of a word when the last written symbol is text and the first is text
\gre@interwordspacetext = 25000 sp plus 45000 sp minus 8000 sp
% space between notes of a bivirga or trivirga
\gre@bitrivirspace = 7600 sp plus 200 sp minus 600 sp
% space between notes of a bistropha or tristrophae
\gre@bitristrospace = 7600 sp plus 200 sp minus 600 sp
% space between two punctum inclinatum
\gre@punctuminclinatumshift= -4300 sp plus 100 sp minus 100 sp
% space before puncta inclinata
\gre@beforepunctainclinatashift= 5800 sp plus 800 sp minus 500 sp
% space between a punctum inclinatum and a punctum inclinatum deminutus
\gre@punctuminclinatumanddebilisshift= -2500 sp plus 100 sp minus 100 sp
% space between two punctum inclinatum deminutus
\gre@punctuminclinatumdebilisshift= -800 sp plus 100 sp minus 100 sp
% space between puncta inclinata, larger ambitus (range=3rd)
\gre@punctuminclinatumbigshift= 8300 sp plus 100 sp minus 100 sp
% space between puncta inclinata, larger ambitus (range=4th -or more?-)
\gre@punctuminclinatummaxshift= 19600 sp plus 100 sp minus 100 sp
% space for the bars (inside syllables)
%first for virgula and divisio minima
\gre@spacearoundsmallbar = 20000 sp plus 25000 sp minus 6000 sp
%then divisio minor
\gre@spacearoundminor = 20000 sp plus 25000 sp minus 6000 sp
%divisio major
\gre@spacearoundmaior = 20000 sp plus 25000 sp minus 6000 sp
%divisio finalis
\gre@spacearoundfinalis = 20000 sp plus 20000 sp minus 6000 sp
%a special space for finalis, for when it is the last glyph
\gre@spacebeforefinalfinalis= 32000 sp plus 8000 sp minus 30000 sp
% the space that will appear around bars that are preceded by a custo and followed by a key.
% well... actually it's the difference between the normal space around bars and the space described previously.
\gre@spacearoundclefbars= 4000 sp plus 500 sp minus 100 sp
% space between the text and the text of the bar
\gre@textbartextspace = 27000 sp plus 15000 sp minus 5400 sp
% minimal space between a note and a bar
\gre@notebarspace = 35000 sp plus 30000 sp minus 3100 sp
% maximal space between two syllables for which we consider a dash is not needed
\gre@maximumspacewithoutdash = 2200 sp
% an extensible space for the beginning of lines
\gre@afterclefnospace = 0 pt plus 30000 sp minus 0 pt
% width of the additional lines, used only for the custos (maybe should depend on the width of the custo...)
% the width is the one for the custos at end of lines, the line for custos in the middle of a score is the same
% multiplied by 2.
\gre@additionalcustoslineswidth = 10000 sp
% space between the initial and the beginning of the score
\gre@afterinitialshift=0.035 em plus 0em minus 0em
% space before the initial and the beginning of the score
\gre@minimalspaceatlinebeginning=1mm
% space to force the initial width to.  Ignored when 0.
\gre@manualinitialwidth=0pt
% this space is the one between the bottom of the first anotation line and the top
% of the second anotation line (above the initial)
\gre@aboveinitialseparation = 0.5mm
% space at the beginning of the lines if there is no clef
\gre@noclefspace = 1mm

%%%%%%%%%%%%%%%%%
% vertical spaces
%%%%%%%%%%%%%%%%%

% first, we have two spaces for the chironomic signs
\gre@abovesignsspace = 8mm
\gre@belowsignsspace = 0mm
% the shift for the low choral sign and the high choral signs on totally free interlines
\gre@lowchoralsignshift = 1000sp
% the shift for the high choral sign
\gre@highchoralsignshift = -5000sp
% the space for the translation
\gre@translationheight = 0.5cm
%the space above the lines
\gre@spaceabovelines = 50000 sp plus 40000 sp minus 10000 sp
%the space between the lines and the bottom of the text
\gre@spacelinestext = 66500 sp plus 0 sp minus 0 sp
%the space beneath the text
\gre@spacebeneathtext = 0 sp plus 0 sp minus 0 sp
% height of the text above the note line
\gre@abovelinestextraise=-1mm%
% height that is added at the top of the lines if there is text above the lines (it must be bigger than the text for it to be taken into consideration)
\gre@abovelinestextheight=3mm%
% an additional shift you can give to the brace above the bars if you don't like it
\gre@braceshift = 0pt
% a shift you can give to the accentus above the curly brace
\gre@curlybraceaccentusshift = -0.5mm

