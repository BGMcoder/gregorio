%GregorioTeX file.
%Copyright (C) 2007-2009 Elie Roux <elie.roux@telecom-bretagne.eu>
%
%This program is free software: you can redistribute it and/or modify
%it under the terms of the GNU General Public License as published by
%the Free Software Foundation, either version 3 of the License, or
%(at your option) any later version.
%
%This program is distributed in the hope that it will be useful,
%but WITHOUT ANY WARRANTY; without even the implied warranty of
%MERCHANTABILITY or FITNESS FOR A PARTICULAR PURPOSE.  See the
%GNU General Public License for more details.
%
%You should have received a copy of the GNU General Public License
%along with this program.  If not, see <http://www.gnu.org/licenses/>.


% this file contains definitions of signs (bar, episemus, punctum, alterations)

%%%%%%%%%%%%%%%%%%%%%%%%%%%%%%%%%%%%%%%%%%%%%%%%%%%%%%%%%%%%%%%%%%%%%%%%%%%%%%%%%%
%% macros for the typesetting of the clefs of the beginning of lines and custos
%%%%%%%%%%%%%%%%%%%%%%%%%%%%%%%%%%%%%%%%%%%%%%%%%%%%%%%%%%%%%%%%%%%%%%%%%%%%%%%%%%

% a count describing the clef line and pitch : 1 for c on the first (bottom) line, 2 for c on the second line, 5 for f on the first, etc.
\newcount\greclefnum

% the width of the clef
\newdimen\greclefwidth

%% marcro to define the clef that will appear at the beginning of the lines
% the first argument is the type : f or c, and the second is the height
% the third argument is whether we must type a space after or not (0 if not, 1 if yes)
% if the fourth argument is a, it means that we must not put a flat after the key, otherwise it's the height of the flat
\def\gresetlinesclef#1#2#3#4{%
  \grelocalleftbox{%
    \kern\greadditionalleftspace %
    \copy\GreLines% draws the lines
    \unkern %
    \hbox{\gretypekey{#1}{#2}{0}{#3}{#4}\hskip\greafterclefnospace}%
  }%
  \xdef\greclefflat{#4}%
  \relax%
}

%% macro calculating the \greclefnum from the letter and number
% #1 is the letter, and #2 the line number, #3 is a if not flated
% and otherwise the height of the flat
\def\grecalculateclefnum#1#2#3{%
  \global\greclefnum=#2\relax %
  \ifx f#1%
    \global\advance\greclefnum by 4\relax %
  \fi %
  \relax %
}

%% macro redrawing a key from clefnum, useful for vertical space changes
\def\greupdatelinesclef{%
  \ifnum\greclefnum > 5\relax%
    \gretempdimcount=\greclefnum %
    \advance\gretempdimcount by -4\relax %
    \gresetlinesclef{f}{\gretempdimcount}{1}{\greclefflat}%
  \else %
    \gresetlinesclef{c}{\greclefnum}{1}{\greclefflat}%
  \fi %
  \relax %
}

% macro that typesets the key
% arguments are : 
%% #1: the type of the key : c or f
%% #2: the line of the key (1 is the lowest)
%% #3: if we must use small key characters (inside a line) or not 0: if not inside, 1 if inside
%% #4: if we must type a space after or not
%% #5: if a, it means that we must not put a flat after the key, otherwise it's the height of the flat
\def\gretypekey#1#2#3#4#5{%
  \ifcase#2 %
  \or%
    \grecalculateglyphraisevalue{c}{0}%
  \or%
    \grecalculateglyphraisevalue{e}{0}%
  \or%
    \grecalculateglyphraisevalue{g}{0}%
  \or%
    \grecalculateglyphraisevalue{i}{0}%
  \fi%
  \gretempdimskip=\grespaceafterlineclef %
  \ifnum#4=0\relax %
    \gretempdimskip=\greafterclefnospace %
  \fi %
  \ifx c#1% we check if it is a c key
    \ifcase#3%
      \raise\greglyphraisevalue\hbox{\grecclefchar}%\hskip\gretempdimskip}%
      \setbox\GreTempwidth=\hbox{\grecclefchar}%
      \global\greclefwidth=\wd\GreTempwidth %
    \or%
      \raise\greglyphraisevalue\hbox{\greincclefchar}%\hskip\gretempdimskip}%
    \fi%
  \else % we consider that it is a f key
    \ifcase#3%
      \raise\greglyphraisevalue\hbox{\grefclefchar}%\hskip\gretempdimskip}%
      \setbox\GreTempwidth=\hbox{\grefclefchar}%
      \global\greclefwidth=\wd\GreTempwidth %
    \or%
      \raise\greglyphraisevalue\hbox{\greinfclefchar}%\hskip\gretempdimskip}%
    \fi%
  \fi%
  \if a#5%
    \hskip\gretempdimskip %
  \else %
    \hskip\greclefflatspace %
    \greflat{#5}{1}%
    \hskip\gretempdimskip %
  \fi %
  \relax %
}

% macro that writes the initial key, and sets the next keys to the same value
% if #3 is a, it means that we must not put a flat after the key, otherwise it's the height
% of the flat
\def\gresetinitialclef#1#2#3{%
  \grecalculateclefnum{#1}{#2}{#3}%
  \ifnum\grelastoflinecount=2\relax % we must not type a space if there is no initial
    \gretypekey{#1}{#2}{0}{0}{#3}%
  \else %
    \gretypekey{#1}{#2}{0}{1}{#3}%
  \fi %
  \gresetlinesclef{#1}{#2}{1}{#3}%
  % if the initial is big, then we adjust the second line
  \ifnum\grebiginitial=0\relax %
  \else %
    \greadjustsecondline %
  \fi %
  \relax%
}

% macro called when the key changes
% #1 and #2 are the type and line of the clef
% #3 is 1 or 0 according to the need of a space before the clef. Useful for clefs after bars for example
% if #4 is a, it means that we must not put a flat after the key, otherwise it's the height
% of the flat
\def\grechangeclef#1#2#3#4{%
  \grecalculateclefnum{#1}{#2}{#4}%
  \greclefchangespace = 16500 sp plus 1650 sp minus 16500 sp%
  \multiply\greclefchangespace by \the\grefactor %
  \ifnum#3=1\relax %
    \hskip\greclefchangespace %
  \else %
    % here it means that there is a bar before the clef, so we skip the difference between the normal space and the space around bars with clef changes
    \hskip -\grespacearoundclefbars %
  \fi %
  \gretypekey{#1}{#2}{1}{0}{#4}%
  \hskip\greclefchangespace %
  \gresetlinesclef{#1}{#2}{1}{#4}%
  \relax%
}

% macro called when the key changes inside a syllable

\def\greinchangeclef#1#2#3{%
  % to see why we reset \grelastoflinecount, see comments of \greglyph.
  \ifnum\grelastoflinecount=2\relax %
    \global\grelastoflinecount=0\relax %
  \fi %
  \greclefchangespace = 16500 sp plus 1650 sp minus 16500 sp%
  \multiply\greclefchangespace by \grefactor %
  \hskip\greclefchangespace %
  \gretypekey{#1}{#2}{1}{0}{#3}%
  \hskip\greclefchangespace %
  \gresetlinesclef{#1}{#2}{0}{#3}%
  \relax%
}

% custo just typesets a custo, useful for before the key changes for example
\def\grecusto#1{%
  \grecalculateglyphraisevalue{#1}{0}%
  %here we need some tricks to draw the line before the custo (for the color)
  \setbox\GreTempwidth=\hbox{\grecustochar#1}%
  \global\gretempwidth=\wd\GreTempwidth %
  \ifx l#1%
    \greadditionaltopcustolinemiddle %
  \fi %
  \ifx m#1%
    \greadditionaltopcustolinemiddle %
  \fi %
  \ifx a#1%
    \greadditionalbottomcustolinemiddle %
  \fi %
  \ifx b#1%
    \greadditionalbottomcustolinemiddle %
  \fi %
  \raise \greglyphraisevalue%
  \copy\GreTempwidth %
  % for now we consider we always have a bar after the custo
  % we don't want to end the line here
  \penalty 10001\relax %
  \hskip -\grespacearoundclefbars %
  \penalty 10001\relax %
  \relax %
}

% the argument is the height
\def\gresetcusto#1{%
  \ifnum\greblockcusto=0\relax %
    \grecalculateglyphraisevalue{#1}{0}%
    %here we need some tricks to draw the line before the custo (for the color)
    \setbox\GreTempwidth=\hbox{%
    % we type a hskip and the we type the custo
    \hskip\grespacebeforecusto %
    \grecustochar#1\relax %
    }%
    \global\gretempwidth=\wd\GreTempwidth %
    % we make \wd\GreTempsign contain the width of a custo
    \setbox\GreTempsign=\hbox{%
    \grecustochar#1\relax %
    }%
    \grelocalrightbox{%
    \ifx l#1%
      \greadditionaltopcustolineend %
    \fi %
    \ifx m#1%
      \greadditionaltopcustolineend %
    \fi %
    \ifx a#1%
      \greadditionalbottomcustolineend %
    \fi %
    \ifx b#1%
      \greadditionalbottomcustolineend %
    \fi %
    \raise \greglyphraisevalue%
    \copy\GreTempwidth %
    }%
  \fi %
  \relax%
}

% macro that typesets an additional line at the top for custos at end of line

\def\greadditionaltopcustolineend{%
  \gretempdim=\grestaffheight %
  \advance\gretempdim by \grespacebeneathtext %
  \advance\gretempdim by \grespacelinestext %
  \advance\gretempdim by \greinterstafflinespace %
  \advance\gretempdim by \greadditionalbottomspace %
  \advance\gretempdim by \grecurrenttranslationheight %
  \raise\gretempdim %
  \hbox to 0pt{%
    \greadditionalstafflinesformat %
    \kern\gretempwidth %
    \gretempdim=\wd\GreTempsign %
    \advance\gretempdim by \greadditionallineswidth %
    \kern -\gretempdim %
    \vrule width \gretempdim height \grestafflineheight%
    \hss %
  }%
  \relax %
}

\def\greadditionalbottomcustolineend{%
  \gretempdim=\grespacebeneathtext %
  \advance\gretempdim by \grespacelinestext %
  \advance\gretempdim by \greadditionalbottomspace %
  \advance\gretempdim by \grecurrenttranslationheight %
  \advance\gretempdim by -\greinterstafflinespace %
  \advance\gretempdim by -\grestafflineheight %
  \raise\gretempdim %
  \hbox to 0pt{%
    \greadditionalstafflinesformat %
    \kern\gretempwidth %
    \gretempdim=\wd\GreTempsign %
    \advance\gretempdim by \greadditionallineswidth %
    \kern -\gretempdim %
    \vrule width \gretempdim height \grestafflineheight%
    \hss %
  }%
  \relax %
}

% same macros, but for a custo in the middle

\def\greadditionaltopcustolinemiddle{%
  \gretempdim=\grestaffheight %
  \advance\gretempdim by \grespacebeneathtext %
  \advance\gretempdim by \grespacelinestext %
  \advance\gretempdim by \greinterstafflinespace %
  \advance\gretempdim by \greadditionalbottomspace %
  \advance\gretempdim by \grecurrenttranslationheight %
  \raise\gretempdim %
  \hbox to 0pt{%
    \greadditionalstafflinesformat %
    \hss %
    \kern\gretempwidth %
    \gretempdim=\greadditionallineswidth %
    \multiply\gretempdim by 2%
    \advance\gretempdim by \wd\GreTempsign %
    \vrule width \gretempdim height \grestafflineheight%
    \hss %
  }%
  \relax %
}

\def\greadditionalbottomcustolinemiddle{%
  \gretempdim=\grespacebeneathtext %
  \advance\gretempdim by \grespacelinestext %
  \advance\gretempdim by \greadditionalbottomspace %
  \advance\gretempdim by \grecurrenttranslationheight %
  \advance\gretempdim by -\greinterstafflinespace %
  \advance\gretempdim by -\grestafflineheight %
  \raise\gretempdim %
  \hbox to 0pt{%
    \greadditionalstafflinesformat %
    \hss %
    \kern\gretempwidth %
    \gretempdim=\greadditionallineswidth %
    \multiply\gretempdim by 2%
    \advance\gretempdim by \wd\GreTempsign %
    \vrule width \gretempdim height \grestafflineheight%
    \hss %
  }%
  \relax %
}

\def\grecustochar#1{%
  \ifx a#1%
    \grecustotopmiddlechar %
  \fi%
  \ifx b#1%
    \grecustotoplongchar %
  \fi%
  \ifx c#1%
    \grecustotopshortchar %
  \fi%
  \ifx d#1%
    \grecustotoplongchar %
  \fi%
  \ifx e#1%
    \grecustotopshortchar %
  \fi%
  \ifx f#1%
    \grecustotoplongchar %
  \fi%
  \ifx g#1%
    \grecustotoplongchar %
  \fi%
  \ifx h#1%
    \grecustotoplongchar %
  \fi%
  \ifx i#1%
    \grecustotoplongchar %
  \fi%
  \ifx j#1%
    \grecustobottomlongchar %
  \fi%
  \ifx k#1%
    \grecustobottomshortchar %
  \fi%
  \ifx l#1%
    \grecustobottomlongchar %
  \fi%
  \ifx m#1%
    \grecustobottommiddlechar %
  \fi%
}

\def\removecusto{%
  \grelocalrightbox{}%
  \relax%
}


%%%%%%%%%%%%%%%%%%%%%%%%%%%%%%%%%%%%%%%%%%%%%%%%%%%%%%%%%%%%%%%%%%%%%%%%%%%%%
%% macros for the typesetting of punctum mora, auctum duplex and choral signs
%%%%%%%%%%%%%%%%%%%%%%%%%%%%%%%%%%%%%%%%%%%%%%%%%%%%%%%%%%%%%%%%%%%%%%%%%%%%%

% a function to typeset a punctum mora, the first argument is the letter of the height of the punctum mora
% if the second argument is one, we the go back to the end of the punctum
% if the second argument is two, it means that we must shift the width of one punctum to the left
% if it is three, it means the same as when it is two, but with ambitus of one
% #3 is 1 in case of a punctommora in the note before the last note of a podatus, porrectus or torculus resupinus, 0 otherwise.
% #4 is 1 if we are at a punctum inclinatum, 0 otherwise
\def\grepunctummora#1#2#3#4{%
  \penalty 10001\relax %
  \ifcase#2\relax %
    \hskip\grespacebeforesigns%
  \or %
    \kern\grespacebeforesigns %
  \or %
    % to get the widht of a punctum minus a line, we calculate the width of a flexus (with ambitus of two) minus the width of a punctum
    \setbox\GreTempwidth=\hbox{\gregoriofont \char 2051}%
    \gretempdim=\wd\GreTempwidth %
    \setbox\GreTempwidth=\hbox{\gregoriofont \char 17}%
    \advance\gretempdim by -\wd\GreTempwidth %
    \kern -\gretempdim %
    \kern \grespacebeforesigns %
  \or %
    \setbox\GreTempwidth=\hbox{\gregoriofont \char 17}%
    \gretempdim=\wd\GreTempwidth %
    \kern -\gretempdim %
    \kern \grespacebeforesigns %
  \fi %
  \ifnum #3=0 %
    \grecalculateglyphraisevalue{#1}{4}%
  \else %
    \grecalculateglyphraisevalue{#1}{8}%
  \fi %
  % here we shift a bit left in the case where we have a punctum inclinatum on a line
  \ifnum#4=1\relax %
    \ifnum\greisonaline=1\relax %
      \gretempwidth=3700sp%
      \multiply\gretempwidth by \the\grefactor %
      \kern -\gretempwidth %
      \gretempwidth =4500sp%
      \multiply\gretempwidth by \the\grefactor %
      \advance\greglyphraisevalue by -\gretempwidth %
    \else %
      \gretempwidth =2500sp%
      \multiply\gretempwidth by \the\grefactor %
      \advance\greglyphraisevalue by -\gretempwidth %
    \fi %
  \fi %
  \penalty 10001\relax %
  \raise \greglyphraisevalue \hbox{\grepunctummorachar}%
  \penalty 10001\relax %
  \ifcase#2\relax\or %
    \setbox\GreTempwidth=\hbox{\grepunctummorachar}%
    \kern -\wd\GreTempwidth %
    \kern -\grespacebeforesigns %
  \or %
    \setbox\GreTempwidth=\hbox{\grepunctummorachar}%
    \kern -\wd\GreTempwidth %
    \kern -\grespacebeforesigns %
    \kern \gretempdim %
  \or %
    \setbox\GreTempwidth=\hbox{\grepunctummorachar}%
    \kern -\wd\GreTempwidth %
    \kern -\grespacebeforesigns %
    \kern \gretempdim %
  \fi %
  \penalty 10001\relax %
  \relax%
}

% a function to typeset an augmentum duplex, easy enough to be understood...
\def\greaugmentumduplex#1#2#3{%
  \grepunctummora{#1}{1}{#3}{0}%
  \grepunctummora{#2}{0}{0}{0}%
  \relax %
}

\gdef\grelowchoralsignstyle#1{#1}
\gdef\grehighchoralsignstyle#1{#1}

% quite simple function: #1 is the height, #2 is the string, #3 is #2 of punctum mora, #4 is #3 of punctum mora
% #3 is 1 if it must be a bit higher
\def\grelowchoralsign#1#2#3{%
  \penalty 10001\relax %
  \hskip\grebeforechoralsignspace %
  \penalty 10001\relax %
  \ifnum#3=1\relax %
    \grecalculateglyphraisevalue{#1}{12}%
  \else %
    \grecalculateglyphraisevalue{#1}{10}%
  \fi %
  \raise\greglyphraisevalue\hbox{\grelowchoralsignstyle{#2}}%
  \relax %
}

\def\grehighchoralsign#1#2#3{%
  \penalty 10001\relax %
  \grevepisemusorrare{#1}{#3}{4}{3}{\grehighchoralsignstyle{#2}}%
  %\grehepisorline{#1}{#3}{0}{4}{\grehighchoralsignstyle{#2}}%
  \relax %
}

%%%%%%%%%%%%%%%%%%%%%%%%%%%%%%%%%%%%%%%%%%%%%%%%%%
%% macros for the typesetting of linea
%%%%%%%%%%%%%%%%%%%%%%%%%%%%%%%%%%%%%%%%%%%%%%%%%%

\def\grelinea#1#2#3{%
  \greglyph{\char 87}{#1}{#2}{#3}{}{}%
  \relax %
}

%%%%%%%%%%%%%%%%%%%%%%%%%%%%%%%%%%%%%%%%%%%%%%%%%%
%% macros for the typesetting of vertical episemus
%%%%%%%%%%%%%%%%%%%%%%%%%%%%%%%%%%%%%%%%%%%%%%%%%%

\newbox\GreTempsign
\newdimen\gretempdimsignwidth

% a macro to help typesetting vertical episemus. The third argument is 0 when we go back to the beginning of the glyph. If it is 2, it means that we must go back first of width #1, and then forward of #2. If it is 1, it means that we only need to go back of #2. if it is 3, we go to the beginning of the glyph, then forward of #1 then back of #2
% #4 is a shift that we want to get applied, useful for punctum inclinatum for example
% #5 is the glyph number. If it is 1 it is an ictus arsicus (?) and 2 for an ictus theticus
% #6 is the type of sign (1: vertical episemus, 2: rare sign, 3: choral sign)
% #7 is the choral sign if relevant
\def\grevepisemusorrareaux#1#2#3#4#5#6#7{%
  % first we set \gretempwidth to the width of the last glyph
  \gretempwidth=\grelastglyphwidth %
  \setbox\GreTempsign=\hbox{\gregoriofont #2}%
  \gretempdimsignwidth=\wd\GreTempsign%
  \divide\gretempdimsignwidth by 2 %
  \ifcase#3%
  % tempwidth is the width of the last glyph
    \advance\gretempwidth by -\gretempdimsignwidth %
  \or%
    \gretempwidth=\gretempdimsignwidth %
  \or%
    \setbox\GreTempsign=\hbox{\gregoriofont #1}%
    \gretempwidth=\wd\GreTempsign %
    \advance\gretempwidth by -\gretempdimsignwidth %
  \or %
    \setbox\GreTempsign=\hbox{\gregoriofont #1}%
    \advance\gretempwidth by -\wd\GreTempsign %
    \advance\gretempwidth by \gretempdimsignwidth %
  \fi%
  \kern-\gretempwidth % we do it here because of the ictus
  % here we save the position of the ictus or we draw the glyph
  \ifnum#5<3\relax %
    \greictus{#5}%
  \else %
  % then we draw the sign
    \ifcase#6\or %
      % vertical episemus
      \setbox\GreTempsign=\hbox{\greverticalepisemuschar}%
    \or % rare sign
      \setbox\GreTempsign=\hbox{\gregoriofont \char #5}%
    \or % choral sign
      \setbox\GreTempsign=\hbox{#7}%
    \fi %
    % we set tempwidth to half a punctum malus half the sign width, so that the centers are aligned
    \gretempdimsignwidth=\wd\GreTempsign %
    \divide\gretempdimsignwidth by 2 %
    \advance\gretempwidth by \gretempdimsignwidth %
    \kern -\gretempdimsignwidth%
    \kern #4sp%
    \raise \greglyphraisevalue \copy\GreTempsign %
    \kern -#4sp%
    % and finally we go back to the end of the glyph, where we were first
    \advance\gretempwidth by -2\gretempdimsignwidth %
  \fi %
  \kern \gretempwidth%
  \relax%
}

% here are the common values for both hepisemus (and consequently also for additional lines) and vepisemus
% this indicates the note
%% 0: last note, which is a standard punctum (works with pes)
%% 1: same, but the last note is a deminutus
%% 2: the note before the last note, which is a standard punctum
%% 3: idem, but the note is the note preceding a deminutus
%% 4: the note before the note before the last note (for porrectus flexus)
%% 5: idem, but when the two last notes are a deminutus
%% 6: the first note, if it is a standard punctum
%% 7: the first note, if it is an initio debilis
%% 8: the first note, if it is a porrectus
%%% the three next arguments make no sense for a vepisemus
%% 9: the two first notes, if it is a porrectus
%% 10: the two first notes, if it is a porrectus flexus
%% 11: the notes two and three of a torculus resupinus
%% 12: the last note, if it is a punctum inclinatum
%% 13: idem, if it is a punctum inclinatum deminutus
%% 14: idem, if it is a stropha
%% 15: idem, with a quilisma
%% 16: idem, with an oriscus
%% 17: same of 2 but for ambitus of one
%% 18: same of 0, but the last note is a smaller punctum (concerning simple podatus, podatus, and torculus resupinus)
%% 19: the first note, if it is an oriscus
%% 20: the first note, if it is a quilisma
%% 21: the second note of a torculus resupinus with first ambitus of at least two
%% 22: idem with ambitus of one
%% 23: idem with initio debilis
%% 24: the last note, if it is a linea punctum (or linea punctum cavum)
%% 25: the last note, if it is a bar
%% 26: the last note, if it is a virgula
%% 27: the last note, if it is a divisio finalis

% a function to typeset a vertical episemus or a rare accent (like accentus, circulus, etc.). The firts argument is the letter of the height of the episemus (not the height of the note it corresponds to. This function must be called after a call to \greglyph. The second argument is the type of glyph it was, more precisely the kind of space there is between the end (or in special cases the beginning) of the glyph and the place where we will typeset the episemus. The values are described in the commentary just above.
% the third argument is the glyph number in the font
% #4 is type (1: vertical episemus, 2: rare sign, 3: choral sign)
% #5 is the choral sign if relevant
\def\grevepisemusorrare#1#2#3#4#5{%
  \ifcase#4\or %
    % if it is a vertical episemus, we call the normal calculateglyphvalue
    \grecalculateglyphraisevalue{#1}{3}%
  \or %
    % if it is not, we call it with 6 as second argument, it will give us the height of the rare signs (accentus, etc.) the first argument is m if the pitch is < k, otherwise it's n.
    \edef\greisabovek{\number 0}%
    \ifx m#1%
      \edef\greisabovek{\number 1}%
    \fi%
    \ifx l#1%
      \edef\greisabovek{\number 2}%
    \fi%
    \ifx k#1%
      \edef\greisabovek{\number 3}%
    \fi%
    \ifnum\greisabovek=1\relax %
      \grecalculateglyphraisevalue{n}{6}%
    \else %
      \ifnum\greisabovek=2\relax %
        \grecalculateglyphraisevalue{m}{6}%
      \else %
        \ifnum\greisabovek=3\relax %
          \grecalculateglyphraisevalue{l}{6}%
        \else %
          \grecalculateglyphraisevalue{k}{6}%
        \fi %
      \fi %
    \fi %
  \or % if it's a choral sign
    \grecalculateglyphraisevalue{#1}{11}%
  \fi %
  \ifcase#2 %
    %case 0
    \grevepisemusorrareaux{0}{\char 17}{1}{0}{#3}{#4}{#5}%
  \or%
    %case 1
    \grevepisemusorrareaux{0}{\char 13}{1}{0}{#3}{#4}{#5}%
  \or%
    %case 2
    % a kind of flexus, it has the good width
    \grevepisemusorrareaux{\char 5634}{\char 17}{2}{0}{#3}{#4}{#5}%
  \or%
    %case 3
    % in order to go to the good place, we first make a kern of - the glyph before deminutus, which has the same width as a standard flexus deminutus
    \grevepisemusorrareaux{0}{\char 5250}{1}{0}{#3}{#4}{#5}%
  \or%
    %case 4
    % is a torculus, it has the good width
    \grevepisemusorrareaux{\char 15371}{\char 17}{2}{0}{#3}{#4}{#5}%
  \or%
    %case 5
    % is a torculus deminutus, it has the good width
    \grevepisemusorrareaux{\char 15884}{\char 17}{2}{0}{#3}{#4}{#5}%
  \or%
    %case 6
    \grevepisemusorrareaux{0}{\char 17}{0}{0}{#3}{#4}{#5}%
  \or%
    %case 7
    \grevepisemusorrareaux{0}{\char 13}{0}{0}{#3}{#4}{#5}%
  \or%
    %case 8, in which we do (for now) the same as case 6
    \grevepisemusorrareaux{0}{\char 17}{0}{0}{#3}{#4}{#5}%
  \or% case 9, 10 and 11
  \or\or\or %
    %case 12
    \grevepisemusorrareaux{0}{\char 19}{0}{30\the\grefactor }{#3}{#4}{#5}%
  \or%
    %case 13
    \grevepisemusorrareaux{0}{\char 32}{0}{0}{#3}{#4}{#5}%
  \or%
    %case 14
    \grevepisemusorrareaux{0}{\char 20}{0}{0}{#3}{#4}{#5}%
  \or%
    %case 15
    \grevepisemusorrareaux{0}{\char 26}{0}{0}{#3}{#4}{#5}%
  \or%
    %case 16
    \grevepisemusorrareaux{0}{\char 27}{0}{0}{#3}{#4}{#5}%
  \or%
    %case 17
    \grevepisemusorrareaux{\char 5633}{\char 17}{2}{0}{#3}{#4}{#5}%
  \or%
    %case 18
    \grevepisemusorrareaux{0}{\char 74}{1}{0}{#3}{#4}{#5}%
  \or%
    %case 19
    \grevepisemusorrareaux{0}{\char 28}{0}{0}{#3}{#4}{#5}%
  \or%
    %case 20
    \grevepisemusorrareaux{0}{\char 26}{0}{0}{#3}{#4}{#5}%
  \or%
    %case 21
    % 5634 is a flexus without bar with ambitus of 2
    \grevepisemusorrareaux{\char 5634}{\char 17}{3}{0}{#3}{#4}{#5}%
  \or%
    %case 22 %
    % 5633 is a flexus without bar with ambitus of 1
    \grevepisemusorrareaux{\char 5633}{\char 17}{3}{0}{#3}{#4}{#5}%
  \or%
    %case 23
    \grevepisemusorrareaux{\char 1986}{\char 17}{3}{0}{#3}{#4}{#5}%
  \or%
    %case 24
    \grevepisemusorrareaux{0}{\char 35}{1}{0}{#3}{#4}{#5}%
  \or%
    %case 25
    \grevepisemusorrareaux{0}{\char 9}{1}{0}{#3}{#4}{#5}%
  \or%
    %case 26
    \grevepisemusorrareaux{0}{\char 8}{1}{0}{#3}{#4}{#5}%
  \or%
    %case 27
    \grevepisemusorrareaux{0}{\gredivisiofinalissymbol}{1}{0}{#3}{#4}{#5}%
  \fi%
  \relax%
}

\def\grevepisemus#1#2{%
  \grevepisemusorrare{#1}{#2}{33}{1}{}%
  \relax %
}

\def\grebarvepisemus#1{%
  \grevepisemusorrare{c}{#1}{33}{1}{}%
  \relax %
}

\def\greictusa#1{%
  \grevepisemusorrare{a}{#1}{1}{2}{}%
  \relax %
}

\def\greictust#1{%
  \grevepisemusorrare{a}{#1}{2}{2}{}%
  \relax %
}

% maybe these four should be optimized
\def\grevepisemusictusa#1#2{%
  \grevepisemusorrare{#1}{#2}{33}{1}{}%
  \greictusa{#2}%
  \relax %
}

\def\grebarvepisemusictusa#1{%
  \grevepisemusorrare{c}{#1}{33}{1}{}%
  \greictusa{#1}%
  \relax %
}

\def\grevepisemusictust#1#2{%
  \grevepisemusorrare{#1}{#2}{33}{1}{}%
  \greictust{#2}%
  \relax %
}

\def\grebarvepisemusictust#1{%
  \grevepisemusorrare{c}{#1}{33}{1}{}%
  \greictust{#1}%
  \relax %
}

\def\greaccentus#1#2{%
  \grevepisemusorrare{#1}{#2}{39}{2}{}%
  \relax %
}

\def\gresemicirculus#1#2{%
  \grevepisemusorrare{#1}{#2}{38}{2}{}%
  \relax %
}

\def\grecirculus#1#2{%
  \grevepisemusorrare{#1}{#2}{37}{2}{}%
  \relax %
}

\def\grereversedaccentus#1#2{%
  \grevepisemusorrare{#1}{#2}{69}{2}{}%
  \relax %
}

\def\grereversedsemicirculus#1#2{%
  \grevepisemusorrare{#1}{#2}{70}{2}{}%
  \relax %
}

%% the macros to save the absolute positions of ictus a and t in the aux file

% macro called at the place of an ictus
% #1 is 1 if ictus a, 2 if ictus t
\def\greictus#1{%
  \pdfsavepos %
  \ifcase#1\or %
    \grewriteaux{ia:\number\pdflastxpos}%
  \or %
    \grewriteaux{it:\number\pdflastxpos}%
  \fi %
  \gregorioattr=4\relax %
  \hbox to 0pt{}%
  \gregorioattr=0\relax %
  \relax %
}

%%%%%%%%%%%%%%%%%%%%%%%%%%%%%%%%%%%%%%%%%%%%%%%%%
%% macros for the typesetting horizontal episemus
%%%%%%%%%%%%%%%%%%%%%%%%%%%%%%%%%%%%%%%%%%%%%%%%%


% a macro that will help in the typesetting of a horizontal episemus and additional lines, the first argument is a glyph that have the same width as the width between the end of the glyph and the beginning of the episemus, and the second argument is the character of the episemus. If the third argument is 0, we go directly to the beginning of the glyph, else we don't change anything
% 4th argument is the same as in hepisorline
% #5 is the choral sign if relevant
\def\grehepisorlineaux#1#2#3#4{%
  \ifnum#3=0%
  % first we set \gretempwidth to the width of the last glyph
    \gretempwidth=\grelastglyphwidth %
  \else%
    \setbox\GreTempsign=\hbox{\gregoriofont #1}%
    \gretempwidth=\wd\GreTempsign%
  \fi%
  \kern -\gretempwidth %
  % then we draw the sign, and go back to the beginning of the sign
  \setbox\GreTempsign=\hbox{\gregoriofont\char #2}%
  % we set tempwidth to half a punctum malus half the sign width, so that the centers are aligned
  \gretempdimsignwidth=\wd\GreTempsign %
  \ifnum#4<2\relax % case of the lines
  \else %
    \gretempdim=\greadditionallineswidth %
    \kern -\gretempdim %
    \advance\gretempdimsignwidth by 2\gretempdim %
  \fi %
  \ifcase#4%
    %case of hepisemus
    \raise \greglyphraisevalue \copy\GreTempsign %
  \or %
    %case of hepisemus at the bottom
    \raise \greglyphraisevalue \copy\GreTempsign %
  \or % case of a line at the top
    \greglyphraisevalue=\greadditionalbottomspace %
    \advance\greglyphraisevalue by \grespacebeneathtext %
    \advance\greglyphraisevalue by \grespacelinestext %
    \advance\greglyphraisevalue by \grecurrenttranslationheight %
    \advance\greglyphraisevalue by 4\greinterstafflinespace %
    \advance\greglyphraisevalue by 4\grestafflineheight %
    \raise\greglyphraisevalue\hbox{\vrule height \grestafflineheight width \gretempdimsignwidth}%
    \kern \gretempdim %
  \or % case of a line at the bottom
    \greglyphraisevalue=\greadditionalbottomspace %
    \advance\greglyphraisevalue by \grespacebeneathtext %
    \advance\greglyphraisevalue by \grecurrenttranslationheight %
    \advance\greglyphraisevalue by \grespacelinestext %
    \advance\greglyphraisevalue by -\greinterstafflinespace %
    \advance\greglyphraisevalue by -\grestafflineheight %
    \raise\greglyphraisevalue\hbox{\vrule height \grestafflineheight width \gretempdimsignwidth}%
    \kern \gretempdim %
  \or %
    %case of choral sign
    \raise \greglyphraisevalue \copy\GreTempsign %
  \or %
  \fi %
  % and finally we go back to the end of the glyph, where we were first
  \advance\gretempwidth by -\gretempdimsignwidth %
  \kern \gretempwidth %
  \relax%
}

% a function to typeset a horizontal line (additional line or episemus). The firts argument is the letter of the height of the episemus (not the height of the note it corresponds to. This function must be called after a call to \greglyph. The second argument is the type of glyph it was, more precisely the kind of space there is between the end (or in special cases the beginning) of the glyph and the place where we will typeset the episemus. The possible values are the common ones
% the third argument is a bit particular, it is the ambitus of the porrectus or porrectus flexus if the second argument is 8 or 9, otherwise
% #4 is 0 for an horizontal episemus, 1 for an horizontal episemus under a note, 3 for a line at the bottom, 2 for a line at the top
\def\grehepisorline#1#2#3#4{%
  \ifcase#4 %
    \grecalculateglyphraisevalue{#1}{9}%
  \or %
    \grecalculateglyphraisevalue{#1}{5}%
  \or %
    % the glyphraisevalue is ignored anyway... but it's just in case...
    \grecalculateglyphraisevalue{l}{0}%
  \or %
    \grecalculateglyphraisevalue{b}{0}%
  \or %
    \grecalculateglyphraisevalue{#1}{11}%
  \fi %
  \ifcase#2 %
    %case 0
    \grehepisorlineaux{\char 17}{40}{1}{#4}%
  \or%
    %case 1
    \grehepisorlineaux{\char 13}{42}{1}{#4}%
  \or%
    %case 2
    % a kind of flexus, it has the good width
    \grehepisorlineaux{\char 5634}{40}{1}{#4}%
  \or%
    %case 3
    % in order to go to the good place, we first make a kern of - the glyph before deminutus, which has the same width as a standard flexus deminutus
    \grehepisorlineaux{\char 5250}{40}{1}{#4}%
  \or%
    %case 4
    % a torculus, it has the good width
    \grehepisorlineaux{\char 15371}{40}{1}{#4}%
  \or%
    %case 5
    % \char 29190 is a torculus deminutus, it has the good width
    \grehepisorlineaux{\char 15884}{40}{1}{#4}%
  \or%
    %case 6
    \grehepisorlineaux{0}{40}{0}{#4}%
  \or%
    %case 7
    %we assume that the initio-debilis has the same width as a punctum deminutus
    \grehepisorlineaux{0}{41}{0}{#4}%
  \or%
    %case 8
    \grehepisorlineaux{0}{40}{0}{#4}%
  \or%
    %case 9
    \ifcase#3%
    \or%
      \grehepisorlineaux{0}{46}{0}{#4}%
    \or%
      \grehepisorlineaux{0}{47}{0}{#4}%
    \or%
      \grehepisorlineaux{0}{48}{0}{#4}%
    \or%
      \grehepisorlineaux{0}{49}{0}{#4}%
    \or%
      \grehepisorlineaux{0}{50}{0}{#4}%
    \fi%
  \or%
    %case 10
    \ifcase#3%
    \or%
      \grehepisorlineaux{0}{51}{0}{#4}%
    \or%
      \grehepisorlineaux{0}{52}{0}{#4}%
    \or%
      \grehepisorlineaux{0}{53}{0}{#4}%
    \or%
      \grehepisorlineaux{0}{54}{0}{#4}%
    \or%
      \grehepisorlineaux{0}{55}{0}{#4}%
    \fi%
  \or %
    %case 11
    \ifcase#3%
    \or%
      \grehepisorlineaux{\char 11270}{51}{1}{#4}%
    \or%
      \grehepisorlineaux{\char 11271}{52}{1}{#4}%
    \or%
      \grehepisorlineaux{\char 11272}{53}{1}{#4}%
    \or%
      \grehepisorlineaux{\char 11273}{54}{1}{#4}%
    \or%
      \grehepisorlineaux{\char 11274}{55}{1}{#4}%
    \fi%
  \or%
    %case 12
    \grehepisorlineaux{\char 19}{43}{1}{#4}%
  \or%
    %case 13
    \grehepisorlineaux{\char 32}{44}{1}{#4}%
  \or%
    %case 14
    \grehepisorlineaux{\char 20}{45}{1}{#4}%
  \or%
    %case 15
    \grehepisorlineaux{\char 26}{56}{1}{#4}%
  \or%
    %case 16
    \grehepisorlineaux{\char 27}{47}{1}{#4}%
  \or%
    %case 17
    \grehepisorlineaux{\char 5633}{40}{1}{#4}%
  \or %
    %case 18
    \grehepisorlineaux{\char 74}{58}{1}{#4}%
  \or %
    %case 19
    \grehepisorlineaux{0}{57}{0}{#4}%
  \or %
    %case 20
    \grehepisorlineaux{0}{56}{0}{#4}%
  \or %
    %case 21
    \ifcase#3%
    \or%
      \grehepisorlineaux{\char 11270}{40}{1}{#4}%
    \or%
      \grehepisorlineaux{\char 11271}{40}{1}{#4}%
    \or%
      \grehepisorlineaux{\char 11272}{40}{1}{#4}%
    \or%
      \grehepisorlineaux{\char 11273}{40}{1}{#4}%
    \or%
      \grehepisorlineaux{\char 11274}{40}{1}{#4}%
    \fi%
  \or %
    %case 22
    \ifcase#3%
    \or%
      \grehepisorlineaux{\char 11270}{40}{1}{#4}%
    \or%
      \grehepisorlineaux{\char 11271}{40}{1}{#4}%
    \or%
      \grehepisorlineaux{\char 11272}{40}{1}{#4}%
    \or%
      \grehepisorlineaux{\char 11273}{40}{1}{#4}%
    \or%
      \grehepisorlineaux{\char 11274}{40}{1}{#4}%
    \fi%
  \or %
    %case 23
    \ifcase#3%
    \or%
      \grehepisorlineaux{\char 11270}{40}{1}{#4}%
    \or%
      \grehepisorlineaux{\char 11271}{40}{1}{#4}%
    \or%
      \grehepisorlineaux{\char 11272}{40}{1}{#4}%
    \or%
      \grehepisorlineaux{\char 11273}{40}{1}{#4}%
    \or%
      \grehepisorlineaux{\char 11274}{40}{1}{#4}%
    \fi%
  \or%
    %case 24
    \grehepisorlineaux{\char 1985}{40}{1}{#4}%the episemus is not quite long enough so I assumed a different width for now...
  \fi%
  \relax%
}

% dumb top function
% #4 is a trick for bridges: if we must use a different height because of a bridge, it's #4, otherwise #4=#1
\def\grehepisemus#1#2#3#4{%
  \ifnum\greaddhepisemusbridges=1\relax %
    \grehepisorline{#4}{#2}{#3}{0}%
  \else %
    \grehepisorline{#1}{#2}{#3}{0}%
  \fi %
  \relax %
}

% same but for episema at the bottom of a note
\def\grehepisemusbottom#1#2#3{%
  \grehepisorline{#1}{#2}{#3}{1}%
  \relax %
}

\def\AddHEpisemusBridges{%
\xdef\greaddhepisemusbridges{1}%
\relax %
}

\def\RemoveHEpisemusBridges{%
\xdef\greaddhepisemusbridges{0}%
\relax %
}

\AddHEpisemusBridges

% same but for a "bridge episemus" after the last note of a glyph (element, syllable) if the next episemus is at the same height
\def\grehepisemusbridge#1#2#3{%
  \ifnum\greaddhepisemusbridges=1\relax %
    \grecalculateglyphraisevalue{#1}{9}%
    \raise\greglyphraisevalue\hbox to 0pt{\gregoriofont\char 40\hss}%
  \fi %
  \relax %
}

% another dumb top function
\def\greadditionalline#1#2#3{%
  \xdef\gresavedglyphraise{\the\greglyphraisevalue}%
  {\greadditionalstafflinesformat %
  \grehepisorline{a}{#1}{#2}{#3}%
  }%
  \greglyphraisevalue=\gresavedglyphraise %
  \relax %
}


%%%%%%%%%%%%%%%%%%%%%%%%%%%%%%%%%%%%%
%% macros for the typesetting of bars
%%%%%%%%%%%%%%%%%%%%%%%%%%%%%%%%%%%%%

% we define two types of macro for each four bar : when it is inside a syllable, and when it is not

\def\greinvirgula#1{%
  \grewritebar{0}{1}{#1}%
  \relax%
}

\def\grevirgula#1{%
  \grewritebar{0}{0}{#1}%
  \relax%
}

\def\greindivisiominima#1{%
  \grewritebar{1}{1}{#1}%
  \relax%
}

\def\gredivisiominima#1{%
  \grewritebar{1}{0}{#1}%
  \relax%
}

\def\greindivisiominor#1{%
  \grewritebar{2}{1}{#1}%
  \relax%
}

\def\gredivisiominor#1{%
  \grewritebar{2}{0}{#1}%
  \relax%
}

\def\greindivisiomaior#1{%
  \grewritebar{3}{1}{#1}%
  \relax%
}

\def\gredivisiomaior#1{%
  \grewritebar{3}{0}{#1}%
  \relax%
}

\def\gredominica#1#2{%
  \ifcase#1\or %
    \grewritebar{6}{0}{#2}%
  \or %
    \grewritebar{7}{0}{#2}%
  \or %
    \grewritebar{8}{0}{#2}%
  \or %
    \grewritebar{9}{0}{#2}%
  \or %
    \grewritebar{10}{0}{#2}%
  \or %
    \grewritebar{11}{0}{#2}%
  \fi %
  \relax%
}

\def\greindominica#1#2{%
  \ifcase#1\or %
    \grewritebar{6}{1}{#2}%
  \or %
    \grewritebar{7}{1}{#2}%
  \or %
    \grewritebar{8}{1}{#2}%
  \or %
    \grewritebar{9}{1}{#2}%
  \or %
    \grewritebar{10}{1}{#2}%
  \or %
    \grewritebar{11}{1}{#2}%
  \fi %
  \relax%
}

\newdimen\gretempdimtwo

\def\greindivisiofinalis#1{%
  \ifcase\greendofscore %
    \grewritebar{4}{1}{#1}%
  \or %
    \grewritebar{5}{1}{#1}%
  \fi %
  \relax%
}

\def\gredivisiofinalis#1{%
  \ifcase\greendofscore %
    \grewritebar{4}{0}{#1}%
  \or %
    \grewritebar{5}{0}{#1}%
  \fi %
  \relax%
}

%a macro to write a bar
%% 1: the type of the bar : 0 for virgula, 1 for minima 2 for minor, 3 for major, 4 for finalis and 5 for the last finalis
%% 2: is % for now we don't use it
%%% 0 if it is outside a syllable
%%% 1 if it is in a syllable
%% 3: macros that may happen before the skip after the bar (typically grevepisemus)
\def\grewritebar#1#2#3{%
  \grenewglyphcommon %
  \grecalculateglyphraisevalue{g}{0}% bar glyphs are made to be at this height
  \penalty 10001 %
  \ifcase#1 % 0 : virgula
    \ifnum#2=1\relax %
      \hskip\grespacearoundsmallbar %
      \penalty 10001 %
    \fi %
    \setbox\GreTempwidth=\hbox{\gregoriofont \char 8}%
    \raise\greglyphraisevalue\hbox{\gregoriofont \char 8}%
    #3\relax %
    \ifnum#2=1\relax %
      \hskip\grespacearoundsmallbar %
    \fi %
  \or % 1 : minima
    \ifnum#2=1\relax %
      \hskip\grespacearoundsmallbar %
      \penalty 10001%
    \fi %
    \setbox\GreTempwidth=\hbox{\gregoriofont \char 9}%
    \raise\greglyphraisevalue\hbox{\gregoriofont \char 9}%
    #3\relax %
    \ifnum#2=1\relax %
      \hskip\grespacearoundsmallbar %
    \fi %
  \or % 2 : minor
    \ifnum#2=1\relax %
      \hskip\grespacearoundminor %
      \penalty 10001 %
    \fi %
    \setbox\GreTempwidth=\hbox{\gregoriofont \char 10}%
    \raise\greglyphraisevalue\hbox{\gregoriofont \char 10}%
    #3\relax %
    \ifnum#2=1\relax %
      \hskip\grespacearoundminor %
    \fi %
  \or % 3 : maior
    \ifnum#2=1\relax %
      \hskip\grespacearoundmaior %
      \penalty 10001 %
    \fi %
    \setbox\GreTempwidth=\hbox{\gregoriofont \char 11}%
    \gredivisiomaiorsymbol %
    #3\relax %
    \ifnum#2=1\relax %
      \hskip\grespacearoundmaior %
    \fi %
  \or % 4 : finalis
    \ifnum#2=1\relax %
      \hskip\grespacearoundfinalis %
      \penalty 10001 %
    \fi %
    \setbox\GreTempwidth=\hbox{\gredivisiofinalissymbol}%
    #3\relax %
    \gredivisiofinalissymbol%
    \ifnum#2=1\relax %
      \hskip\grespacearoundfinalis %
    \fi %
  \or % 5 : finalis
    \ifnum#2=1\relax %
      \hskip\grespacebeforefinalfinalis %
      \penalty 10001 %
    \fi %
    \setbox\GreTempwidth=\hbox{\gredivisiofinalissymbol}%
    #3\relax %
    \gredivisiofinalissymbol%
    \ifnum#2=1\relax %
      \hskip\grespacearoundfinalis %
    \fi %
  \or % 6 : dominican bar 1
    \grecalculateglyphraisevalue{e}{0}%
    % we need to adjust the height of the bar a little so that it is perfectly aligned with the bottom (or the top for some bars) of the staff line, which is not the case by default if \grestafflinefactor is not 10.
    \advance\greglyphraisevalue by -\grestafflinediff %
    \ifnum#2=1\relax %
      \hskip\grespacearoundmaior %
      \penalty 10001 %
    \fi %
    \setbox\GreTempwidth=\hbox{\gregoriofont \char 83}%
    \raise\greglyphraisevalue\hbox{\gregoriofont \char 83}%
    #3\relax %
    \ifnum#2=1\relax %
      \hskip\grespacearoundmaior %
    \fi %
  \or % 7 : dominican bar 2
    \grecalculateglyphraisevalue{e}{0}%
    \advance\greglyphraisevalue by \grestafflinediff %
    \ifnum#2=1\relax %
      \hskip\grespacearoundmaior %
      \penalty 10001 %
    \fi %
    \setbox\GreTempwidth=\hbox{\gregoriofont \char 84}%
    \raise\greglyphraisevalue\hbox{\gregoriofont \char 84}%
    #3\relax %
    \ifnum#2=1\relax %
      \hskip\grespacearoundmaior %
    \fi %
  \or % 8 : dominican bar 3
    \ifnum#2=1\relax %
      \hskip\grespacearoundmaior %
      \penalty 10001 %
    \fi %
    \advance\greglyphraisevalue by -\grestafflinediff %
    \setbox\GreTempwidth=\hbox{\gregoriofont \char 83}%
    \raise\greglyphraisevalue\hbox{\gregoriofont \char 83}%
    #3\relax %
    \ifnum#2=1\relax %
      \hskip\grespacearoundmaior %
    \fi %
  \or % 9 : dominican bar 4
    \ifnum#2=1\relax %
      \hskip\grespacearoundmaior %
      \penalty 10001 %
    \fi %
    \advance\greglyphraisevalue by \grestafflinediff %
    \setbox\GreTempwidth=\hbox{\gregoriofont \char 84}%
    \raise\greglyphraisevalue\hbox{\gregoriofont \char 84}%
    #3\relax %
    \ifnum#2=1\relax %
      \hskip\grespacearoundmaior %
    \fi %
  \or % 10 : dominican bar 5
    \grecalculateglyphraisevalue{i}{0}%
    \advance\greglyphraisevalue by -\grestafflinediff %
    \ifnum#2=1\relax %
      \hskip\grespacearoundmaior %
      \penalty 10001 %
    \fi %
    \setbox\GreTempwidth=\hbox{\gregoriofont \char 83}%
    \raise\greglyphraisevalue\hbox{\gregoriofont \char 83}%
    #3\relax %
    \ifnum#2=1\relax %
      \hskip\grespacearoundmaior %
    \fi %
  \or % 11 : dominican bar 6
    \grecalculateglyphraisevalue{i}{0}%
    \advance\greglyphraisevalue by \grestafflinediff %
    \ifnum#2=1\relax %
      \hskip\grespacearoundmaior %
      \penalty 10001 %
    \fi %
    \setbox\GreTempwidth=\hbox{\gregoriofont \char 84}%
    \raise\greglyphraisevalue\hbox{\gregoriofont \char 84}%
    #3\relax %
    \ifnum#2=1\relax %
      \hskip\grespacearoundmaior %
    \fi %
  \fi %
  \global\grelastglyphwidth=\wd\GreTempwidth %
  \relax%
}

\def\gredivisiomaiorsymbol{%
  \ifnum\grestafflinefactor=10\relax %
    %\grecalculateglyphraisevalue{g}{0}% bar glyphs are made to be at this height
    \raise\greglyphraisevalue\hbox{\gregoriofont \char 11}%
  \else %
    \setbox\GreTempwidth=\hbox{\gregoriofont \char 11}%
    % we calculate the raise of the bar
    \gretempdim=\greadditionalbottomspace %
    \advance\gretempdim by \grespacebeneathtext %
    \advance\gretempdim by \grespacelinestext %
    \advance\gretempdim by \grecurrenttranslationheight %
    % we calculate the height of the bar
    \gretempwidth=4\grestafflineheight %
    \advance\gretempwidth by 3\greinterstafflinespace %
    \raise\gretempdim\hbox{\vrule height \gretempwidth width \wd\GreTempwidth}%
  \fi %
  \relax %
}

\def\gredivisiofinalissymbol{%
  \grecalculateglyphraisevalue{g}{0}% bar glyphs are made to be at this height
  \gredivisiomaiorsymbol %
  \gretempdimtwo = 12000 sp%
  \multiply\gretempdimtwo by \the\grefactor%
  \kern \gretempdimtwo%
  \penalty 10001%
  \gredivisiomaiorsymbol %
}

%a count to tell if we have to keep the localrightbox until the end
\newcount\keeprightbox

%macro to end a line with a divisio finalis
\def\grefinaldivisiofinalis#1{%
  \ifcase#1% case 0
    \grelocalrightbox{}%
    \grelocalleftbox{}%
    \penalty 10001\relax %
    \hskip\grespacebeforefinalfinalis %
    \penalty 10001\relax %
    \grebarsyllable{}{}{}{1}{}{}{0}{}{%
      \gredivisiofinalis{}%
    }%
  \or % case 1
    \global\keeprightbox=1 %
    \grelocalrightbox{%
      \hbox{%
        \hskip\grespacebeforefinalfinalis %
      }%
      \gredivisiofinalissymbol %
    }%
  \fi %
  \relax%
}

%macro to end a line with a divisio maior
\def\grefinaldivisiomaior#1{%
  \ifcase#1% case 0
    \grelocalrightbox{}%
    \grelocalleftbox{}%
      \grebarsyllable{}{}{}{1}{}{}{0}{}{%
      \gredivisiomaior %
    }%
  \or % case 1
    \hskip\grespacebeforefinalfinalis %
    \global\keeprightbox=1 %
    \grelocalrightbox{%
      \grecalculateglyphraisevalue{g}{0}% bar glyphs are made to be at this height
      \raise\greglyphraisevalue\hbox{\gregoriofont \char 11}%
    }%
  \fi %
  \relax%
}

%%%%%%%%%%%%%%%%%%%%%%%%%%%%%%%%%%%%%%%%%%
%% macros for filling holes of empty notes
%%%%%%%%%%%%%%%%%%%%%%%%%%%%%%%%%%%%%%%%%%

\def\GreHidePCLines{%
\xdef\grehidepclines{1}%
}

\def\GreDontHidePCLines{%
\xdef\grehidepclines{0}%
}

\def\GreHideAltLines{%
\xdef\grehidealtlines{1}%
}

\def\GreDontHideAltLines{%
\xdef\grehidealtlines{0}%
}

\GreDontHideAltLines
\GreHidePCLines


% the argument is the character with which we fill the hole, and we suppose that
% isonaline and glyphraisevalue are correctly set.
\def\grefillhole#1{%
  \setbox\GreTempwidth=\hbox{#1}%
  \hbox to 0pt{%
    \color{grebackgroundcolor}%
    \raise \greglyphraisevalue %
    \copy\GreTempwidth %
    \pdfliteral{}% this is a ugly hack for old versions of LuaTeX to work
    \hss %
  }%
  \penalty 10001\relax %
\relax %
}

%%%%%%%%%%%%%%%%%%%%%%%%%%%%%%%%%%%%%
%% macros for typesetting alterations
%%%%%%%%%%%%%%%%%%%%%%%%%%%%%%%%%%%%%

% a count saying if the first glyph is an alteration
\newcount\grefirstisalteration

% the top level macro:
% #1 is the height
% #2 is the char of the alteration
% #3 is the char of the alteration hole
% #4 is 1 in the case of a flat for a key change, 0 otherwise
\def\grealteration#1#2#3#4{%
  % to see why we reset \grelastoflinecount, see comments of \greglyph.
  \ifnum#4=0\relax %
    \grenewglyphcommon %
    \ifnum\the\grefirstglyph=1\relax %
      \global\grefirstisalteration=1\relax %
    \fi %
  \fi %
  \grecalculateglyphraisevalue{#1}{0}%
  \ifnum\grehidealtlines=1\relax %
    \grefillhole{#3}%
  \fi %
  \setbox\GreTempwidth=\hbox{#2}%
  \gretempwidth=\wd\GreTempwidth %
  %\kern\gretempwidth %
  %#3\relax %
  %\kern-\gretempwidth %
  \raise \greglyphraisevalue%
  \copy\GreTempwidth%
  \ifnum#4=0\relax %
    % we try to avoid line breaking after a flat or a natural
    \penalty 10001\relax %
    \ifnum\the\grefirstglyph=1\relax %
      \gretempdim=\wd\GreTempwidth %
      \advance\gretempdim by \grealterationspace %
      \global\advance\grenotesaligncenter by \gretempdim %
      \kern\grealterationspace %
    \else %
      \hskip\grealterationspace %
    \fi %
    \penalty 10001\relax %
  \fi %
  %#4\relax %
  %\penalty 10001\relax %
  \relax %
}
  
% This macro typesets a flat on the height provided by #1. #2 is not used yet, but it
% will determine if the flat has zero width or not. #3 and #4 should be the same as greglyph's #5 and #6, but it's not really done nor useful...

\def\greflat#1#2{%
  \grealteration{#1}{\greflatchar}{\greflatholechar}{#2}%
  \relax%
}

% Same as the one before, but for naturals.

\def\grenatural#1#2{%
  \grealteration{#1}{\grenaturalchar}{\grenaturalholechar}{#2}%
  \relax%
}

% Same as the one before, but for sharps.

\def\gresharp#1#2{%
  \grealteration{#1}{\gresharpchar}{\gresharpholechar}{#2}%
  \relax%
}

%%%%%%%%%%%%%%%%%%%%%%%%%%%%%%%%%%%%%%%
%% macros for typesetting punctum cavum
%%%%%%%%%%%%%%%%%%%%%%%%%%%%%%%%%%%%%%%

\def\UseAlternatePunctumCavum{%
\gdef\punctumcavumchar{\gregoriofont\char 75}%
\gdef\lineapunctumcavumchar{\gregoriofont\char 76}%
\gdef\punctumcavumholechar{\gregoriofont\char 78}%
\gdef\lineapunctumcavumholechar{\gregoriofont\char 80}%
\relax %
}

\def\UseNormalPunctumCavum{%
\gdef\grepunctumcavumchar{\gregoriofont\char 34}%
\gdef\grelineapunctumcavumchar{\gregoriofont\char 36}%
\gdef\grepunctumcavumholechar{\gregoriofont\char 77}%
\gdef\grelineapunctumcavumholechar{\gregoriofont\char 79}%
\relax %
}

\UseNormalPunctumCavum

\def\grepunctumcavum#1#2#3#4#5{%
  \grecalculateglyphraisevalue{#1}{0}%
  \setbox\GreTempwidth=\hbox{\gregoriofont\grepunctumcavumchar}%
  \global\grelastglyphwidth=\wd\GreTempwidth %
  \kern\grelastglyphwidth %
  #4\relax %
  \kern-\grelastglyphwidth %
  \ifnum\grehidepclines=1\relax %
    \grefillhole{\grepunctumcavumholechar}%
  \fi %
  \greglyph{\grepunctumcavumchar}{#1}{#2}{#3}{}{#5}%
  \relax %
}

\def\grelineapunctumcavum#1#2#3#4#5{%
  \grecalculateglyphraisevalue{#1}{0}%
  \setbox\GreTempwidth=\hbox{\gregoriofont\grelineapunctumcavumchar}%
  \global\grelastglyphwidth=\wd\GreTempwidth %
  \kern\grelastglyphwidth %
  #4\relax %
  \kern-\grelastglyphwidth %
  \ifnum\grehidepclines=1\relax %
    \grefillhole{\grelineapunctumcavumholechar}%
  \fi %
  \greglyph{\grelineapunctumcavumchar}{#1}{#2}{#3}{}{#5}%
  \relax %
}

%%%%%%%%%%%%%%%%%%%%%%%%%%%%%%%%%%
%% macros for the different styles
%%%%%%%%%%%%%%%%%%%%%%%%%%%%%%%%%%

\def\greusedefaultstyle{%
\gdef\greflatchar{\gregoriofont\char 6}%
\gdef\greflatholechar{\gregoriofont\char 81}%
\gdef\grenaturalchar{\gregoriofont\char 7}%
\gdef\grenaturalholechar{\gregoriofont\char 82}%
\gdef\gresharpchar{\gregoriofont\char 85}%
\gdef\gresharpholechar{\gregoriofont\char 86}%
\gdef\grecclefchar{\gregoriofont\char 1}%
\gdef\grefclefchar{\gregoriofont\char 2}%
\gdef\greincclefchar{\gregoriofont\char 3}%
\gdef\greinfclefchar{\gregoriofont\char 4}%
\gdef\grepunctummorachar{\gregoriofont\char 14}%
\gdef\greverticalepisemuschar{\gregoriofont\char 33}%
\gdef\grecustotoplongchar{\gregoriofont\char 60}%
\gdef\grecustotopshortchar{\gregoriofont\char 61}%
\gdef\grecustotopmiddlechar{\gregoriofont\char 62}%
\gdef\grecustobottomlongchar{\gregoriofont\char 63}%
\gdef\grecustobottomshortchar{\gregoriofont\char 64}%
\gdef\grecustobottommiddlechar{\gregoriofont\char 65}%
\relax %
}

\def\greusemedicaeastyle{%
\greusestylecommon %
\gdef\grecustotoplongchar{\gregoriostylefont\char 8}%
\gdef\grecustotopshortchar{\gregoriostylefont\char 7}%
\gdef\grecustotopmiddlechar{\gregoriostylefont\char 9}%
\gdef\grecustobottomlongchar{\gregoriostylefont\char 11}%
\gdef\grecustobottomshortchar{\gregoriostylefont\char 10}%
\gdef\grecustobottommiddlechar{\gregoriostylefont\char 12}%
\gdef\grecclefchar{\gregoriostylefont\char 21}%
\gdef\grefclefchar{\gregoriostylefont\char 23}%
\gdef\greincclefchar{\gregoriostylefont\char 22}%
\gdef\greinfclefchar{\gregoriostylefont\char 24}%
\gdef\greflatchar{\gregoriostylefont\char 0}%
\gdef\greflatholechar{\gregoriostylefont\char 33}%
\relax %
}

\def\greusehufnagelstyle{%
\greusestylecommon %
\gdef\grecustotoplongchar{\gregoriostylefont\char 2}%
\gdef\grecustotopshortchar{\gregoriostylefont\char 1}%
\gdef\grecustotopmiddlechar{\gregoriostylefont\char 3}%
\gdef\grecustobottomlongchar{\gregoriostylefont\char 5}%
\gdef\grecustobottomshortchar{\gregoriostylefont\char 4}%
\gdef\grecustobottommiddlechar{\gregoriostylefont\char 6}%
\gdef\grecclefchar{\gregoriostylefont\char 25}%
\gdef\grefclefchar{\gregoriostylefont\char 26}%
\gdef\greincclefchar{\gregoriostylefont\char 27}%
\gdef\greinfclefchar{\gregoriostylefont\char 28}%
\gdef\greflatchar{\gregoriostylefont\char 20}%
\gdef\greflatholechar{\gregoriostylefont\char 32}%
\relax %
}

\def\greusemensuralstyle{%
\greusestylecommon %
\gdef\grecustotoplongchar{\gregoriostylefont\char 14}%
\gdef\grecustotopshortchar{\gregoriostylefont\char 13}%
\gdef\grecustotopmiddlechar{\gregoriostylefont\char 15}%
\gdef\grecustobottomlongchar{\gregoriostylefont\char 17}%
\gdef\grecustobottomshortchar{\gregoriostylefont\char 16}%
\gdef\grecustobottommiddlechar{\gregoriostylefont\char 18}%
\gdef\greflatchar{\gregoriostylefont\char 19}%
\gdef\greflatholechar{\gregoriostylefont\char 31}%
\relax %
}

\def\greusestylecommon{%
  \ifnum\greusestylefont=0\relax %
  \xdef\greusestylefont{1}%
  \gresetstylefont %
  \relax %
}

\greusedefaultstyle
