\section{User Controls}

These functions are available to the user to customize elements of the score which cannot be controlled from the gabc file.

\subsection{Colors}

Colors are strictly a \LaTeX\ phenomena as currently implemented.  All commands which change the color of text simply print the text without alteration in PlainTeX.

All colors can be redefined using \verb=\definecolor=.  See \verb=xcolor= package for documentation.

\verb=grebackgroundcolor=%tex.sty
?? color behind staves?
\verb={RGB}{255,255,255}= (white)

\verb=gregoriocolor=%tex.sty
Color of elements formatted by \verb=\colored=
\verb={RGB}{229,53,44} (red similar to what is found in liturgical documents)


\subsection{Environments}

Environments are used to apply standard formatting to elements of the score.  Redefining the environment (via a \verb=\renewenvironment= command) allows the user to change how these elements appear in the score.

While environments are technically a \LaTeX\ phenomena, for users of PlainTeX each environment has a pseudo-environment equivalent (defined in \verb=gregoriotex.tex=) which is used in place of the environment.  The pseudo-enviroments are called with a \verb=\start*= and \verb=\stop*= command (where ``*'' is the name of the environment).  To change the format of the text in the pseudo-environment, simply redefine these commands (usually you only need to redefine the \verb=\start*= command).  Just make sure that any new definition for \verb=\start*= has \verb=\begingroup= as the first line and for \verb=\stop*= has \verb=\endgroup= as the last line.

\verb=initialformat=%tex.sty, PlainTeX version in tex.tex
	Defines how the first letter of a score appears when using leading initial.  Defaults to \verb=huge= (in PlainTeX).
	Deprecated version: \verb=\greinitialformat=

\verb=biginitialformat=%tex.sty, PlainTeX version in tex.tex
	Defines how the first letter of a score appears when using a 2-line leading initial.  Defaults to \verb=Huge= (in PlainTeX)
	Deprecated version: \verb=\grebiginitialformat=

\verb=abovelinetextstyle=%tex.sty, PlainTeX version in tex.tex
	Defines how the text placed above the staff lines appears.  Defaults to \verb=small= and \verb=italic= (\verb=italic= only in PlainTeX).
	Deprecated version: \verb=\greabovelinestextstyle=

\verb=translationformat=%tex.sty, PlainTeX version in ???
	Defines how the translation text appears.  Defaults to \verb=italic=.


\subsection{Commands}

In general, commands should not be modified.  Exceptions are noted below.

\verb=\colored{<text>}=%tex.sty
	Colors its argument with \verb=gregoriocolor=
	Modify color by changing \verb=gregoriocolor=.

\verb=\coloredlines{<color-name>}=%tex.sty
	Colors the staff lines.  Color argument must be a named color defined using \verb=\definecolor=
	Deprecated version: \verb=\grecoloredlines=

\verb=\redlines=%tex.sty
	Short cut for coloring the staff lines \verb=gregoriocolor=.  Equivalent to \verb=\coloredlines{gregoriocolor}=
	Modify color by changing \verb=gregoriocolor=.
	Deprecated version: \verb=\greredlines=

\verb=\normallines=%tex.sty
	Removes all formatting from staff lines.
	Deprecated version: \verb=\grenormallines=

\verb=\setstaffsize{<int>}=%tex.tex
	Changes the size of the staff (and the neumes placed on the staves).  Larger values for larger staves, smaller values for smaller staves.  Default value is 17.  \emph{Note:} This does not change the size of the accompanying text (lyrics and/or translations).
	Deprecated version: \verb=\setgrefactor=

\verb=\addtranslationspace=%Not sure this needs to be a user function%tex.tex
	Add space below the stave for a translation.  Has no effect if space is already allocated for a translation.

\verb=\removetranslationspace=%Not sure this needs to be a user function%tex.tex
	Removes the space below the stave for the translation (un-does \verb=\addtranslationspace=).  Has no effect if space is currently not allocated to a translation.

\verb=\setfirstannotation{<string>}=%tex.tex
	Macro to set the first (top) annotation above the initial.  This macro automatically aligns the top of the annotation with the 4th bar line.  This is equivalent to \verb=\setfirstlineaboveinitial{<string>}{<string>}=.
	Deprecated version: \verb=\gresetfirstannotation=
	Deprecated version: \verb=\writemode= (\nb this applied small caps and bold automatically while \verb=\setfirstannotation= does not.)%tex.sty, PlainTeX version in tex.tex

\verb=\setfirstlineaboveinitial{<string>}{<object>}=%tex.tex
	Macro to set the first (top} annotation above the initial.  This macros allows you to control how far below the top of the staff <string> appears via the second argument.  When its height is 0, the baseline of the annotation aligns with the 4th bar line.  Positive values for the height push the annotation down.  The argument cannot have a negative height.  <object> can be pretty much anything with a visible representation.  \verb=\newline= and \verb=\\= are not respected in either <string> or <object>.  Note: large annotations which stick out above the staff will push the commentary up.
	Deprecated version: \verb=\gresetfirstlineaboveinitial=

\verb=\setsecondannotation{<string>}=%tex.tex
	Macro to set the second (bottom) annotation above the initial.
	Deprecated version: \verb=\gresetsecondannotation=

\verb=\scorereference=%tex.tex
	Does nothing.

\verb=\commentary=%tex.tex
	Marco to place the commentary (usually the scriptural reference) in the top right-hand corner of the score.  While individual calls do not support multiple lines, the macro can be called multiple times; each call will typeset a new line.
	
\verb=\removelines=%tex.tex
	Macro to remove the staff lines.
	Deprecated version: \verb=\greremovelines=

\verb=\donotremovelines=%tex.tex
	Macro to force staff lines (undoes the effects of \verb=\removelines=).
	Deprecated version: \verb=\gredonotremovelines=

\verb=\settranslationcenteringscheme{<int>}=%tex.tex
	Macro to change the centering scheme for the translation.
	Possible values: 0=translation is left aligned with the corresponding text; 1=translation is centered below the corresponding text
	Deprecated version: \verb=\setgretranslationcenteringscheme=

\verb=\setnlbintranslation{<int>}=%tex.tex
	Macro to change whether line breaks are allowed in the translations.
	Possible values: 0=line breaks are allowed; 1=line breaks are prohibited

\verb=\blockcustos=%tex.tex
	Macro to block custom.  Applies to all subsequent scores in group.
	Deprecated version: \verb=\greblockcustos=


	
\section{Gregorio Controls}

These functions are the ones written by Gregorio to the gtex file.  While one could, in theory, use/change them to alter the appearance of elements of the score, it is far better to make your changes in the gabc file and let Gregorio make the changes to the gtex file.


\verb=\begingregorioscore=%tex.tex
	Macro to start a score.
\verb=\endgregorioscore=%tex.tex
	Macro to end a score.

\verb=\greaccentus=
\verb=\greactiveatechironomy=%tex.tex
	Macro called at the beginning of the score to enable chironomy.
\verb=\greadditionalline=
\verb=\greadjustsecondline=%didn't actually find this one in gregoriotex-write.c%tex.tex
	Macro to call before first syllable, but after \verb=\gresetinitialclef=
\verb=\greadjustthirdline=%tex.tex
	Macro to call during the second line.
\verb=\greaugmentumduplex=

\verb=\grebarbrace=
\verb=\grebarsyllable=
\verb=\grebarvepisemus=
\verb=\grebarvepisemusictusa=
\verb=\grebarvepisemusictust=
\verb=\grebeginnlbarea{<int1>}{<int2>}=%tex.tex
	Macro called at beginning of a no line break area.
	<int1>: 0 = not in the neumes; 1 = in the neumes
	<int2>: 0 = call didn't come from translation centering; 1 = call came from translation centering.
\verb=\grebeginnotes=%tex.tex
	Macro to draw the staff lines.  Comes after the initial but before the clef.
\verb=\greboldfont{<string>}=%tex.sty, PlainTeX version in tex.tex
	Makes <string> bold.  Accesses \LaTeX \verb=\textbf= or PlainTeX \verb=\bf= as appropriate.  Corresponds to ``<b></b>'' tags in gabc.

\verb=\grechangeclef
\verb=\grecirculus=
\verb=\grecolored{<string>}=%tex.sty, PlainTeX version in tex.tex
	Colors <string> in \verb=gregoriocolor.=  Corresponds to ``<c></c>'' tags in gabc.  Does nothing in PlainTeX
\verb=\grecusto=

\verb=\gredagger=
\verb=\grediscretionary{}=
\verb=\gredivisiofinalis=
\verb=\gredivisiomaior=
\verb=\gredivisiominima=
\verb=\gredivisiominor=
\verb=\gredominica=

\verb=\greendnlbarea=%tex.tex
	Macro called at beginning of a no line break area.
	<int1>: 0 = not in the neumes; 1 = in the neumes
	<int2>: 0 = call didn't come from translation centering; 1 = call came from translation centering.

\verb=\greendofelement{<int1>}{<int2>}=%tex.tex
	Macro to end elements.
	<int1>: 0 = default space; 1 = larger space; 2 = glyph space; 3 = zero-width space
	<int2>: 0 = space is breakable; 1 = space is unbreakable
\verb=\greendofglyph{<int>}=%tex.tex
	Macro to end a glyph without ending the element.
	<int>: 0 = default space; 1 = zero-width space; 2 = space between flat or natural and a note; 3 = space between two puncta inclinata; 4 = space between bivirga or trivirga; 5 = space between bistropha or tristropha; 6 = space after a punctum mora XXX: not used yet, not so sure it is a good idea...; 7 = space between a punctum inclinatum and a punctum inclinatum debilis; 8 = space between two puncta inclinata debilis; 9 = space before a punctum (or something else) and a punctum inclinatum; 10 = space between puncta inclinata (also debilis for now), larger ambitus (range=3rd).; 11 = space between puncta inclinata (also debilis for now), larger ambitus (range=4th or more)

\verb=\grefinaldivisiofinalis{0}=
\verb=\grefinaldivisiomaior{0}=
\verb=\grefirstlinebottomspace{<int1>}{<int2>}=%tex.tex
	Macro for additional bottom space for the first line
	<int1>: 0 = no note below staff; 1 = note below 1st line (position c); 2 = note on 0th line (position b); 3 = note below 0th line (position a); 4 = note below 0th line (position a) with vertical episemus
	<int2>: 0 = no translation; 1 = translation present
\verb=\greflat=

\verb=\greglyph=
\verb=\gregorianmode{<int>}=%tex.tex
	If the gabc file contains a mode in the header, then this function places said mode as the first (top) annotation.  This function effectively disables \verb=\setfirstlineaboveinitial=.  <int> is an arabic number from 1 to 8.  Other values are ignored (and \verb=\setfirstlineaboveinitial= should still work).
	Bug: This macro needs to appear before \verb=\greinitial= in the gtex file but \verb=gregoriotex-write.c= places it after.
\verb=\gregorioapiverstion{<int>}=%tex.tex
	Checks to see if GregorioTeX API is version <int> (and therefore compatible with the score.  <int> is date in format: yyyymmdd

\verb=\grehepisemus=
\verb=\grehepisemusbottom=
\verb=\grehepisemusbridge=
\verb=\grehighchoralsign=
\verb=\grehyph=

\verb=\greictusa=
\verb=\greictust=
\verb=\grein=
\verb=\greinitial{<char>}=%tex.tex
	Macro to set the initial (<char>) in the score.
\verb=\greinsertchiroline=%tex.tex
	Macro called at the beginning of a line to insert the chironomic signs.
\verb=\greitalic{<string>}=%tex.sty, PlainTeX version in tex.tex
	Makes <string> italic.  Accesses \LaTeX \verb=\textit= or PlainTeX \verb=\it= as appropriate.  Corresponds to ``<i></i>'' tags in gabc.

\verb=\grelastofline=%tex.tex
	Macro to set \verb=\gre@lastoflinecount= to 1 (i.e. mark that this syllable is the last of the line).
\verb=\grelastofscore=%tex.tex
	Macro to mark the syllable as the last of the score.
\verb=\grelinea=
\verb=\grelineapunctumcavum=
\verb=\grelowchoralsign=

\verb=\grenatural=
\verb=\grenewline=%tex.tex
	Macro to call if you want to go to the next line simply.
\verb=\grenewlinewithspace{<int1>}{<int2>}{<int3>}{<int4>}=%tex.tex
	Macro called to go to the next line but when there are additional vertical spaces to add
	<int1>: 0 = no note above staff; 1 = note above 4th line (position k); 2 = note on 5th line (position l); 3 = note above 5th line (position m)
	<int2>: 0 = no note below staff; 1 note below 1st line (position c); 2 = note on 0th line (position b); 3 = note below 0th line (position a); 4 = note below 0th line (position a) with vertical episemus
	<int3>: 0 = no translation; 1 = translation present
	<int4>: 0 = no extra space above staff; 1 = extra space above staff
\verb=\grenewparline=%tex.tex
	Same as \verb=\grenewline= except line is not justified.
\verb=\grenewparlinewithspace{<int1>}{<int2>}{<int3>}{<int4>}=%tex.tex
	Same as \verb=\grenewlinewithspace= except line is not justified.
\verb=\grenoinitial=%tex.tex
	Macro called when no initial is being set.
\verb=\grenormalinitial=%not actually found in gregoriotex-write.c%tex.tex
	Macro to cancel a 2-line initial.

\verb=\grepunctumcavum=
\verb=\grepunctummora=

\verb=\grereversedaccentus=
\verb=\grereversedsemicirculus=

\verb=\grescorereference=%tex.tex
	Currently does nothing.
\verb=\gresemicirculus=
\verb=\gresetbiginitial=%tex.tex
	Macro which indicates that a 2-line initial is desired.
\verb=\gresetfixednexttextformat=
\verb=\gresetfixedtextformat=
\verb=\gresetinitialclef{<char>}{<int>}{<boolean>}=%tex.tex
	possible values for <char>: c, f
\verb=\gresetlinesclef{<char>}{<int1>}{<int2>}{<int3>}=%tex.tex
	possible values for <char>: c, f
	possible values for <int2>: 0, 1 (1 = space before clef, 0 = no space)
\verb=\gresettextabovelines<{string>}=
	Macro to place <string> above the lines and empty \verb=\gre@currenttextabovelines= when done.
\verb=\gresharp=
\verb=\gresmallcaps{<string>}=%tex.sty, PlainTeX version in tex.tex
	Makes <string> small capitals.  Accesses \LaTeX \verb=\textsc= or PlainTeX \verb=\sc= as appropriate  Corresponds to ``<sc></sc>'' tags in gabc.
\verb=\grestar=
\verb=\gresyllable=

\verb=\gretilde=%tex.tex
	Macro to print $\sim$.
\verb=\gretranslationcenterend=%tex.tex
	Macro to set \verb=\gre@mustdotranslationcenerend= to 1.
\verb=\grett{<string>}=%tex.sty, PlainTeX version in tex.tex
	Makes <string> typewriter font.  Accesses \LaTeX \verb=\texttt= or PlainTeX \verb=\tt= as appropriate.

\verb=\greul{<string>}=%tex.sty, PlainTeX version in tex.tex
	Makes <string> underlined under \LaTeX using \verb=\underline=.  Does nothing in PlainTeX

\verb=\grevepisemus=
\verb=\grevepisemusictusa=
\verb=\grevirgula=

\verb=\grewritetranslation{<string>}=%tex.tex
	Macro to typeset <string> in the translation position.
\verb=\grewritetranslationwithcenterbeginning{<string>}=%tex.tex
	Macro to typeset <string> in the translation position (at the beginning of a line?).

\verb=\grezerhyph=

\verb=\setgregoriofont{<string>}=%tex.tex
	Macro to set the font used for the glyphs.
	<string>: gregorio; parmesan; greciliae; gregoria


\section{Gregoriotex Controls}

These functions are the ones used by Gregoriotex internally as it process the commands listed above.  They should not appear in any user document and are listed here for programmer documentation purposes only.

\verb=\gre@coloredlines{<string>}=%tex.sty
	Changes the color of the staff lines to <string>.  <string> must be a named color defined using \verb=\definecolor=

\verb=\gre@redlines=%tex.sty
	Changes the color of the staff lines to \verb=gregoriocolor=.

\verb=\gre@normallines%tex.sty
	Resets the formatting of the staff lines.

\verb=\gre@error{<string>}=%tex.tex
	Raises an error which is identified as coming from GregorioTeX.  Uses \LaTeX \verb=\PackageError= or PlainTeX \verb=\errmessage= as appropriate.  <string> is the accompanying message.
	
\verb=\gre@warn{<string>}=%tex.tex
	Raises a warning which is identified as coming from GregorioTeX.  Uses \LaTeX \verb=\PackageWarning= or PlainTeX \verb=\message= as appropriate.  <string is the accompanying message.
	
\verb=\gre@localleftbox=%tex.tex
	Alias for \verb=\luatexlocalleftbox=.  Used to make propagating changes in latex easier.

\verb=\gre@localrightbox=%tex.tex
	Alias for \verb=\luatexlocalrightbox=.  Used to make propagating changes in latex easier.
	
\verb=\gre@unsetattribute=%tex.tex
	Alias for \verb=\unsetlutexattribute{\gregorioattr}=

\verb=\gregorioattr=%tex.tex
	A luatex attribute we put on the text nodes.
	If it is 1, it means that there may be a dash here if this syllable is at the end of a line.
	If it is 2, it means that it's never useful to typeset a dash.
	If it is 0, it just means that we are in a score.

\verb=\gregoriocenterattr=%tex.tex
	A luatex attribute used for translation centering.

\verb=\gregoriotexversion{<int>}=%tex.tex
	Defines the current version of GregorioTeX API.  <int> is a date in format: yyyymmdd

\verb=\gre@internalversion{<int>}=%tex.tex
	The version of GregorioTeX.  <int> is a date in format: yyyymmdd

\verb=\gre@declarefileversion{<string>}{<int>}=%tex.tex
	Checks to see if GregorioTeX component <string> with version <int> is compatible with \verb=\gre@internalversion=

\verb=\gre@factor=%tex.tex
	Count representing the size of the staff.  Initialized to 0, but changed to 17 by \verb=\begingregorianscore= if user hasn't changed it.
	
\verb=\gre@stafflinewidth=%tex.tex
	Dimension representing the width of a line of staff.  Can vary, for example, at the first line.

\verb=\gre@linewidth=%tex.tex
	Dimension representing the width of the score (including initial).

\verb=\gre@calculateconstantglyphraise=%tex.tex
	Macro to caluclate \verb=\gre@constantglyphraise=

\verb=\gre@constantglyphraise=%tex.tex
	Dimension representing ??

\verb=\gre@currenttranslationheight=%tex.tex
	Dimension representing the space for the translation beneath the text.

\verb=\gre@addtranslationspace=%tex.tex
	Macro to tell Gregorio to set space for the translation.

\verb=\gre@removetranslationspace=%tex.tex
	Macro to tell Gregorio to remove the space allocated to the translation.

\verb=\gre@kernbeforeeol=%tex.tex
	Macro describing a kern to make before ending the line, which we sometimes want (see \verb=\gresyllable=)

\verb=\gre@newlinecommon{<int1>}{<int2>}{<int3>}{<int4>}{<int5>}=%tex.tex
	Macro we call each time we force a changing of line, it automatically sets \verb=\greknownline=, and adjusts left spaces.
	<int1>: 0 = no note above staff; 1 = note above 4th line (position k); 2 = note on 5th line (position l); 3 = note above 5th line (position m)
	<int2>: 0 = no note below staff; 1 note below 1st line (position c); 2 = note on 0th line (position b); 3 = note below 0th line (position a); 4 = note below 0th line (position a) with vertical episemus
	<int3>: 0 = no translation; 1 = translation present
	<int4>: 0 = justify line; 1 = do not justify line
	<int5>: 0 = no extra space above staff; 1 = extra space above staff

\verb=\gre@additionalbottomspace=%tex.tex
	Dimension representing extra space below the staff needed for low notes.

\verb=\gre@additionaltopspace=%tex.tex
	Dimension representing extra space above the staff needed for high notes.

\verb=\gre@updateadditionalspaces{<int1>}{<int2>}=%tex.tex
	Macro which updates \verb=\gre@additionalbottomspace= and \verb=\gre@additionaltopspace=
	<int1>: 0 = no note above staff; 1 = note above 4th line (position k); 2 = note on 5th line (position l); 3 = note above 5th line (position m)
	<int2>: 0 = no note below staff; 1 note below 1st line (position c); 2 = note on 0th line (position b); 3 = note below 0th line (position a); 4 = note below 0th line (position a) with vertical episemus

\verb=\gre@textlower=%tex.tex
	Dimension representing the height of the separation between the 0th line (which is invisible except for notes in the a or b position) and the bottom of the text.

\verb=\gre@Tempwidth=%tex.tex
	Box used to calculate \verb=\gre@tempwidth=
	
\verb=\gre@tempwidth=%tex.tex
	Dimension representing width of some element.

\verb=\gre@widthof{<string>}=%tex.tex
	Macro which calculates \verb=\gre@tempwidth= as width of <string>.

\verb=\gre@textaligncenter=%tex.tex
	Dimension representing the width from the beginning of the letters in a syllable to the middle of the middle letters.  Used for lining up neumes and syllables.
	
\verb=\gre@findtextaligncenter{<string1>}{<string2>}{<int>}=%tex.tex
	Macro for calculating \verb=\gre@textaligncenter=.
	<string1>: The first part of the syllable (any preceding consonants in Latin)
	<string2>: The middle part of the syllable (the vowel in Latin)
	<int>: syllable the calculation is being performed for; 0 = current syllable; 1 = next syllable

\verb=\gre@additionalleftspace=%tex.tex
	Dimension representing the additional space that has to be added to the localleftbox for a big initial (one taking two lines).

\verb=\Gre@Initial=%tex.tex
	Box containing the initial.

\verb=\gre@initialwidth=%tex.tex
	Dimension representing the width of the initial (and the space after).

\verb=\gre@biginitial=%tex.tex
	Count indicating whether initial takes 2 lines: 0 = it doesn't; 1 = it does.

\verb=\gre@knowline=%tex.tex
	Line gregoriotex thinks is being set.

\verb=\gre@updateleftbox=%tex.tex
	Macro which adjusts size of line and clef placement after initial.

\verb=\gre@updatelinewidth=%tex.tex
	Macro which adjusts the width of the line.

\verb=\gre@initialformat{<char>}=%tex.tex
	Macro which applies formatting in \verb=initialformat= environment to initial (<char>).

\verb=\gre@biginitialformat{<char>}=%tex.tex
	Macro which applies formatting in \verb=biginitialformat= environment to initial (<char>).

\verb=\Gre@Aboveinitialfirstbox=%tex.tex
	The box which contains the first (top) annotation above the initial.

\verb=\gre@aboveinitialfirstraise=%tex.tex
	Dimension representing the space allocated to the first annotation.

\verb=\Gre@Aboveinitialsecondbox=%tex.tex
	The box which contains the second (bottom) annotation above the initial.

\verb=\gre@aboveinitialsecondraise=%tex.tex
	Dimension representing the space allocated to the second annotation.

\verb=\gre@setaboveinitialrais=%tex.tex
	Macro to give \verb=\gre@aboveinitialfirstraise= and \verb=\gre@aboveinitialsecondraise= their working values.
	
\verb=\gre@currentabovelinestextheight=%tex.tex
	Dimension representing the space allocated above the lines for text.	

\verb=\gre@abovelinestextstyle{<string>}=%tex.tex
	Macro to apply the above line text style to <string>.

\verb=\gre@addspaceabove=%tex.tex
	Macro to allocate space for text placed above the staff lines.

\verb=\gre@removespaceabove=%tex.tex
	Macro to unallocate space for text above the staff lines.

\verb=\gre@currenttextabovelines{<string>}=%tex.tex
	Macro containing the text (<string>) which is currently being placed above the staff lines.

\verb=\gre@typsettextabovelines{<string>}=%tex.tex
	Macro to typeset <string> above the staff lines.

\verb=\gre@removelinescount=%tex.tex
	Boolean indicating whether staff lines should be printed.  Has value 0 when lines are printed, 1 when they are not.

\verb=\gre@drawfirstlines=%tex.tex
	Macro to draw the staff lines for the first line of the score (\ie it accounts for the space taken up by the initial).

\verb=\Gre@Lines=%tex.tex
	Box to contain the staff lines for lines other than the first line.

\verb=\gre@generatelines=%tex.tex
	Macro to fill \verb=\Gre@Lines=.

\verb=\gre@smallsecondline=%tex.tex
	Macro called when the initial is big to make the second set of staff lines the same length as the first.

\verb=\gre@normallines=%tex.tex
	Marco called after the second set of staff lines when the initial is big to go back to normal width lines.

\verb=\gre@translationcenteringscheme=%tex.tex
	Boolean to indicate the centering scheme used for the translation.
	Possible values: 0=translation is left aligned with corresponding text; 1=translation is centered with the corresponding text

\verb=\gre@nlbintranslation=%tex.tex
	Variable used to indicate whether line breaks are allowed in the translation.
	Possible values: 0=line breaks are allowed; 1=line breaks are prohibited

\verb=\gre@translationformat{<string>}=%tex.tex (is this defined somewhere else too?)
	Macro to apply the translation format to string.

\verb=\gre@mustdotranslationcenterend=%tex.tex
	Boolean to indicate if the translation is at the end of a line?
	Possible values: 0=not at the end of a line; 1=at the end of a line

\verb=\gre@dotranslationcenterend=%tex.tex
	Macro to set things up for a translation at the end of a line.

\verb=\gre@printchirovbars=%tex.tex
	Count that is 1 or 0 if we need to print the small vertical bars in the chironomic line.

\verb=\gre@nolastline=%tex.tex
	Macro to call when there is just a little thing that will go to the last line, when it is not necessary
	It doesn't seem to be used, so it's a good candidate for deprecation!

\verb=\gre@endofword{<int>}=%tex.tex
	Macro called at the end of each word.  Does extra stuff when <int> is 1?

\verb=\gre@endbeforebar{<int>}=%tex.tex
	Macro called at end of word when next element is a bar.  <int> has same function as in \verb=\gre@endofword=.

\verb=\gre@endafterbar{<int>}=%tex.tex
	Macro called after a bar.  <int> has same function as in \verb=\gre@endofword=.

\verb=\gre@lastoflinecount=%tex.tex
	Count which marks the last syllable of the line.
	Possible values: 0=nothing; 1=last syllable of line; 2=first syllable of line

\verb=\gre@disableeolshifts=%tex.tex
	Boolean which indicates that the syllable should be shifted left a bit.
	Possible values: 0=shift happens; 1=shift doesn't happen.

\verb=\Gre@DisableEOLShifts=%tex.tex
	Macro to set \verb=\gre@disableeolshifts= to 1.

\verb=\Gre@EnableEOLShifts=%tex.tex
	Macro to set \verb=\gre@disableeolshifts= to 0.

\verb=\gre@blockcusto=%tex.tex
	Count to indicate if the custo should be blocked.
	Possible values: 0=do not block custo; 1=block custo

\verb=\gre@endofsyllable=%tex.tex
	Macro called at the end of a syllable which does not end a word.

\verb=\gre@nlbstate=%tex.tex
	Count to indicate the no line break areas.
	Possible values: 0 = not a no line break area; 1 = no line break due to translation centering; 2 = no line break due to <nlba> tag

\verb=\gregoriofontname=%tex.tex
	Macro which holds the default font name (greciliae).

\verb=\gre@usestylefont=%tex.tex
	Count to indicate if gregoriostylefont should be used.
	Possible values: 0 = do not use; 1 = use

\verb=\gre@setstylefont=%tex.tex
	Macro to load greextra as gregoriostylefont at correct size.