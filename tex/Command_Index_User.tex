\section{User Controls}

These functions are available to the user to customize elements of the score which cannot be controlled from the gabc file.

\subsection{Colors}

Colors are strictly a \LaTeX\ phenomena as currently implemented.  All commands which change the color of text simply print the text without alteration in PlainTeX.

All colors can be redefined using \verb=\definecolor=.  See \verb=xcolor= package for documentation.

\verb=grebackgroundcolor=%tex.sty
?? color behind staves?
\verb={RGB}{255,255,255}= (white)

\verb=gregoriocolor=%tex.sty
Color of elements formatted by \verb=\colored=
\verb={RGB}{229,53,44} (red similar to what is found in liturgical documents)


\subsection{Environments}

Environments are used to apply standard formatting to elements of the score.  Redefining the environment (via a \verb=\renewenvironment= command) allows the user to change how these elements appear in the score.

While environments are technically a \LaTeX\ phenomena, for users of PlainTeX each environment has a pseudo-environment equivalent which is used in place of the environment.  The pseudo-enviroments are called with a \verb=\start*= and \verb=\stop*= command (where ``*'' is the name of the environment).  To change the format of the text in the pseudo-environment, simply redefine these commands (usually you only need to redefine the \verb=\start*= command).  Just make sure that any new definition for \verb=\start*= has \verb=\begingroup= as the first line and for \verb=\stop*= has \verb=\endgroup= as the last line.

\verb=initialformat=%tex.sty, PlainTeX version in tex.tex
	Defines how the first letter of a score appears when using leading initial.  Defaults to \verb=huge= (New Century Roman at 40pt in PlainTeX).
	Deprecated version: \verb=\greinitialformat=

\verb=biginitialformat=%tex.sty, PlainTeX version in tex.tex
	Defines how the first letter of a score appears when using a 2-line leading initial.  Defaults to \verb=Huge= (New Century Roman at 80pt in PlainTeX)
	Deprecated version: \verb=\grebiginitialformat=

\verb=abovelinetextstyle=%tex.sty, PlainTeX version in tex.tex
	Defines how the text placed above the staff lines appears.  Defaults to \verb=small= and \verb=italic= (\verb=italic= only in PlainTeX).
	Deprecated version: \verb=\greabovelinestextstyle=

\verb=translationformat=%tex.sty, PlainTeX version in tex.tex
	Defines how the translation text appears.  Defaults to \verb=italic=.

\verb=normalstafflinesformat=%tex.sty, PlainTeX version in tex.tex
	Defines how the staff lines will appear.  Note that the staff lines are drawn with \verb=\hrule= and so very few things will actually affect their appearance (color being the most prominent one that does).  Empty default.
	Deprecated version: \verb=\grenormalstafflinesformat

\verb=additionalstafflinesformat=%tex.sty, PlainTeX version in tex.tex
	Defines how the additional staff lines (the little ones which are drawn for notes above or below the normal staff lines) appear.  Defaults to \verb=normalstafflinesformat=.
	Deprecated version: \verb=\greadditionalstafflinesformat

\verb=lowchoralsignstyle=%tex.sty, PlainTeX version in tex-signs.tex
	Defines how the low choral signs should appear.  Empty default.
	Deprecated version: \verb=\grelowchoralsignstyle=

\verb=highchoralsignstyle=%tex.sty, PlainTeX version in tex-signs.tex
	Defines how the high choral signs should appear.  Empty default.
	Deprecated version: \verb=\grelowchoralsignstyle=

\subsection{Commands}

In general, commands should not be modified.  Exceptions are noted below.

\verb=\colored#1=%tex.sty
	Colors its argument with \verb=gregoriocolor=
	Modify color by changing \verb=gregoriocolor=.

\verb=\coloredlines#1=%tex.sty
	Colors the staff lines.
	#1 must be a named color defined using \verb=\definecolor=
	Deprecated version: \verb=\grecoloredlines=

\verb=\redlines=%tex.sty
	Short cut for coloring the staff lines \verb=gregoriocolor=.  Equivalent to \verb=\coloredlines{gregoriocolor}=
	Modify color by changing \verb=gregoriocolor=.
	Deprecated version: \verb=\greredlines=

\verb=\normallines=%tex.sty
	Removes all formatting from staff lines.
	Deprecated version: \verb=\grenormallines=

\verb=\setstaffsize#1=%tex.tex
	Changes the size of the staff (and the neumes placed on the staves).  #1 is an integer. Larger values for larger staves, smaller values for smaller staves.  Default value is 17.  \emph{Note:} This does not change the size of the accompanying text (lyrics and/or translations).
	Deprecated version: \verb=\setgrefactor=

\verb=\addtranslationspace=%Not sure this needs to be a user function%tex.tex
	Add space below the stave for a translation.  Has no effect if space is already allocated for a translation.

\verb=\removetranslationspace=%Not sure this needs to be a user function%tex.tex
	Removes the space below the stave for the translation (un-does \verb=\addtranslationspace=).  Has no effect if space is currently not allocated to a translation.

\verb=\setfirstannotation#1=%tex.tex
	Macro to set the first (top) annotation above the initial.  Argument is string (the contents of the annotation).  This macro automatically aligns the top of the annotation with the 4th bar line.  This is equivalent to \verb=\setfirstlineaboveinitial{#1}{#1}=.
	Deprecated version: \verb=\gresetfirstannotation=
	Deprecated version: \verb=\writemode= (\nb this applied small caps and bold automatically while \verb=\setfirstannotation= does not.)%tex.sty, PlainTeX version in tex.tex

\verb=\setfirstlineaboveinitial#1#2=%tex.tex
	Macro to set the first (top} annotation above the initial.  This macros allows you to control how far below the top of the staff #1 (string containing contents of annotation) appears via the second argument.  When its height is 0, the baseline of the annotation aligns with the 4th bar line.  Positive values for the height push the annotation down.  The argument cannot have a negative height.  #2 can be pretty much anything with a visible representation.  \verb=\newline= and \verb=\\= are not respected in either #1 or #2.  Note: large annotations which stick out above the staff will push the commentary up.
	Deprecated version: \verb=\gresetfirstlineaboveinitial=

\verb=\setsecondannotation#1=%tex.tex
	Macro to set the second (bottom) annotation above the initial.  Argument is string containing contents of annotation.
	Deprecated version: \verb=\gresetsecondannotation=

\verb=\scorereference=%tex.tex
	Does nothing.

\verb=\commentary=%tex.tex
	Marco to place the commentary (usually the scriptural reference) in the top right-hand corner of the score.  While individual calls do not support multiple lines, the macro can be called multiple times; each call will typeset a new line.
	
\verb=\removelines=%tex.tex
	Macro to remove the staff lines.
	Deprecated version: \verb=\greremovelines=

\verb=\donotremovelines=%tex.tex
	Macro to force staff lines (undoes the effects of \verb=\removelines=).
	Deprecated version: \verb=\gredonotremovelines=

\verb=\settranslationcenteringscheme#1=%tex.tex
	Macro to change the centering scheme for the translation.
	#1 --- integer: 0=translation is left aligned with the corresponding text; 1=translation is centered below the corresponding text
	Deprecated version: \verb=\setgretranslationcenteringscheme=

\verb=\setnlbintranslation#1=%tex.tex
	Macro to change whether line breaks are allowed in the translations.
	#1 --- integer: 0=line breaks are allowed; 1=line breaks are prohibited

\verb=\blockcustos=%tex.tex
	Macro to block custom.  Applies to all subsequent scores in group.
	Deprecated version: \verb=\greblockcustos=

\verb=\GreSetStaffLinesFormat#1=%tex.tex
	Deprecated.  Used to set the format for the staff lines.  See above notes on the environments for updated documentation.

\verb=\includescore#1=%tex.tex
	Macro for including scores.  Works on both gabc and tex files.  #1 is string specifying the filename and location of the score (relative or absolute).

\verb=\includetexscore#1=%tex.tex
	Macro for including scores which have already been run through gregorio (i.e. are in gtex format).  #1 is string specifying the filename and location of the score (relative or absolute).
	Deprecated version: \verb=\greincludetexscore=

\verb=\includegabcscore#1=%tex.tex
	Macro for including scores which are in gabc format.  This macro will run gregorio on the file before including it.  #1 is string specifying the filename and location of the score (relative or absolute).
	Deprecated version: \verb=\greincludegabcscore=

\verb=\GreUseNormalHyphen=%tex.tex
	Tell gregoriotex to use normal hyphens at the end of lines (default behavior).
	Note: Placing an explicit hyphen in between special character brackets (\ie <sp>-</sp>) in the gabc file will always result in a zero-width hyphen.
	Note: This command only affects the hyphens within a score.  Hyphens elsewhere in a document are unaffected (hence the "Gre" in the command name).
	
\verb=\GreUseZeroHyphen=%tex.tex
	Tell gregoriotex to use zero-width hyphens at the end of lines.
	Note: This command only affects the hyphens within a score.  Hyphens elsewhere in a document are unaffected (hence the "Gre" in the command name).

\verb=\GreForceHyphen=%tex.tex
	Tell gregoriotex to force the appearance of hyphens between all syllables.
	Note: This command only affects the hyphens within a score.  Hyphens elsewhere in a document are unaffected (hence the "Gre" in the command name).

\verb=\removeclef=%tex-signs.tex
	Macro to remove the clefs from a score.
	Deprecated version: \verb=\greremoveclef=

\verb=\normalclef=%tex-signs.tex
	Macro to restore clefs to a score (undoes the effects of \verb=\removeclef=).
	Deprecated version: \verb=\grenormalclef=

\verb=\removecusto=%tex-signs.tex
	Macro to empty the box containing the custos.
	%I’m not sure what the purpose of this macro is.  Need to figure out where it is used.

