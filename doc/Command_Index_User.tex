% !TEX root = UserManual.tex
\section{User Controls}

These functions are available to the user to customize elements of the
score which cannot be controlled from the gabc file. They can be added
to any \verb=.tex= file. Do not add them to any \verb=.gtex= or
\verb=.gabc= file.

\subsection{Colors}

Colors are strictly a \LaTeX\ phenomena as currently implemented.  All
commands which change the color of text simply print the text without
alteration in Plain\TeX.

All colors can be redefined using \verb=\definecolor=.  See the
\verb=xcolor= package for documentation.

Example:\par\medskip
\begin{latexcode}
  \definecolor{gregoriocolor}{RGB}{229,53,44}
\end{latexcode}

\macroname{grebackgroundcolor}{gregoriotex.sty}
%\verb=grebackgroundcolor=}%tex.sty
Color gregoriotex uses to block out elements which have been printed,
but shouldn't show (\eg, the staff line going through the interior of
a punctum cavum) \verb={RGB}{255,255,255}= (white)

\macroname{gregoriocolor}{gregoriotex.sty}
%\verb=gregoriocolor=%tex.sty
Color of elements formatted by \verb=\colored= \verb={RGB}{229,53,44}=
(red similar to what is found in liturgical documents)

\subsection{Environments}

Environments are used to apply standard formatting to elements of the
score.  Redefining the environment (via a \verb=\renewenvironment=
command) allows the user to change how these elements appear in the
score.

While environments are technically a \LaTeX\ phenomena, for users of
Plain\TeX each environment has a pseudo-environment equivalent which is
used in place of the environment.  The pseudo-enviroments are called
with a \verb=\start*= and \verb=\stop*= command (where ``*'' is the
name of the environment).  To change the format of the text in the
pseudo-environment, simply redefine these commands (usually you only
need to redefine the \verb=\start*= command).  Just make sure that any
new definition for \verb=\start*= has \verb=\begingroup= as the first
line and for \verb=\stop*= has \verb=\endgroup= as the last line.

\macroname{initialformat}{gregoriotex.sty}
%\verb=initialformat=%tex.sty, Plain\TeX version in tex.tex
Defines how the first letter of a score appears when using leading
initial.  Defaults to \verb=huge= (New Century Roman at 40pt in
Plain\TeX).

Example, to change the font of the first letter:

\begin{latexcode}
  \renewenvironment*{initialformat}%
                    {\fontfamily{ppl}\selectfont\huge}{}
\end{latexcode}

\smallskip\hspace{15pt} Deprecated version: \verb=\greinitialformat=

\macroname{biginitialformat}{gregoriotex.sty}
%\verb=biginitialformat=%tex.sty, Plain\TeX version in tex.tex
Defines how the first letter of a score appears when using a 2-line
leading initial.  Defaults to \verb=Huge= (New Century Roman at 80pt
in Plain\TeX)

\smallskip\hspace{15pt} Deprecated version: \verb=\grebiginitialformat=

\macroname{abovelinetextstyle}{gregoriotex.sty}
%\verb=abovelinetextstyle=%tex.sty, Plain\TeX version in tex.tex
Defines how the text placed above the staff lines appears.  Defaults
to \verb=small= and \verb=italic= (\verb=italic= only in Plain\TeX).

\smallskip\hspace{15pt} Deprecated version: \verb=\greabovelinestextstyle=

\macroname{translationformat}{gregoriotex.sty}
%\verb=translationformat=%tex.sty, Plain\TeX version in tex.tex
Defines how the translation text appears.  Defaults to \verb=italic=.

\macroname{normalstafflinesformat}{}
%\verb=normalstafflinesformat=%tex.sty, Plain\TeX version in tex.tex
Defines how the staff lines will appear.  Note that the staff lines
are drawn with \verb=\hrule= and so very few things will actually
affect their appearance (color being the most prominent one that
does).  Empty default.

\smallskip\hspace{15pt} Deprecated version: \verb=\grenormalstafflinesformat=

\macroname{additionalstafflinesformat}{gregoriotex.sty}
%\verb=additionalstafflinesformat=%tex.sty, Plain\TeX version in tex.tex
Defines how the additional staff lines (the little ones which are
drawn for notes above or below the normal staff lines) appear.
Defaults to \verb=normalstafflinesformat=.

\smallskip\hspace{15pt} Deprecated version: \verb=\greadditionalstafflinesformat=

\macroname{lowchoralsignstyle}{gregoriotex.sty}
%\verb=lowchoralsignstyle=%tex.sty, Plain\TeX version in tex-signs.tex
Defines how the low choral signs should appear.  Empty default.

\smallskip\hspace{15pt} Deprecated version: \verb=\grelowchoralsignstyle=

\macroname{highchoralsignstyle}{gregoriotex.sty}
%\verb=highchoralsignstyle=%tex.sty, Plain\TeX version in tex-signs.tex
Defines how the high choral signs should appear.  Empty default.

\smallskip\hspace{15pt} Deprecated version: \verb=\grelowchoralsignstyle=

\subsection{Commands}

In general, commands should not be modified.  Exceptions are noted below.

\macroname{\textbackslash colorred\{\#1\}}{gregoriotex.sty}
%\verb=\colored#1=%tex.sty
Colors its argument with \verb=gregoriocolor= Modify color by changing
\verb=gregoriocolor=.

\begin{argtable}
  \#1 & string & The text to be colored. \\
\end{argtable}

Example:\par\medskip
\begin{latexcode}
  \colorred{This text is red.}
\end{latexcode}

\macroname{\textbackslash colorredlines\{\#1\}}{gregoriotex.sty}
%\verb=\coloredlines#1=%tex.sty
Colors the staff lines.

\begin{argtable}
  \#1 & string & Must be a named color defined using \verb=\definecolor= \\
\end{argtable}

Example:\par\medskip
\begin{latexcode}
  \definecolor{myorange}{RGB}{221,86,8}
  \coloredlines{myorange}
\end{latexcode}

\smallskip\hspace{15pt} Deprecated version: \verb=\grecoloredlines=

\macroname{\textbackslash redlines}{gregoriotex.sty}
%\verb=\redlines=%tex.sty
Short cut for coloring the staff lines \verb=gregoriocolor=.

Equivalent to \verb=\coloredlines{gregoriocolor}= Modify color by
changing \verb=gregoriocolor=.

\smallskip\hspace{15pt} Deprecated version: \verb=\greredlines=

\macroname{\textbackslash normallines}{gregoriotex.sty}
%\verb=\normallines=%tex.sty
Removes all formatting from staff lines.

\smallskip\hspace{15pt} Deprecated version: \verb=\grenormallines=

\macroname{\textbackslash setstaffsize\{\#1\}}{gregoriotex.tex}
%\verb=\setstaffsize#1=%tex.tex
Changes the size of the staff (and the neumes placed on the staves).

\emph{Note:} This does not change the size of the accompanying text
(lyrics and/or translations).

\begin{argtable}
  \#1 & integer & Larger values for larger staves, smaller values for
  smaller staves. Default value is 17. \\
\end{argtable}

Example:\par\medskip
\begin{latexcode}
  \setstaffsize{20}
\end{latexcode}

\emph{Note:} If set to \verb=0= gregoriotex will use the default value.

\smallskip\hspace{15pt} Deprecated version: \verb=\setgrefactor=

\macroname{\textbackslash addtranslationspace}{gregoriotex.tex}
%\verb=\addtranslationspace=%Not sure this needs to be a user function%tex.tex
Add space below the stave for a translation.  Has no effect if space
is already allocated for a translation.

\macroname{\textbackslash removetranslationspace}{gregoriotex.tex}
%\verb=\removetranslationspace=%Not sure this needs to be a user function%tex.tex
Removes the space below the stave for the translation (un-does\\
\verb=\addtranslationspace=). Has no effect if space is currently not
allocated to a translation.

\macroname{\textbackslash setfirstannotation\{\#1\}}{gregoriotex.tex}
%\verb=\setfirstannotation#1=%tex.tex
Macro to set the first (top) annotation above the initial.  This macro
automatically aligns the top of the annotation with the 4th bar line.
This is equivalent to \verb=\setfirstlineaboveinitial{#1}{#1}=.

See also \verb=\setsecondannotation= for typesetting two lines of text
above the initial.

\begin{argtable}
  \#1 & string & The text of the annotation.\\
\end{argtable}

Example:\par\medskip
\begin{latexcode}
  \setfirstannotation{VIII g}
\end{latexcode}

\smallskip\hspace{15pt} Deprecated version: \verb=\gresetfirstannotation=

\smallskip\hspace{15pt} Deprecated version: \verb=\writemode= (\nb this
applied small caps and bold automatically while
\verb=\setfirstannotation= does
not.)%tex.sty, Plain\TeX version in tex.tex

\macroname{\textbackslash setfirstlineaboveinitial\{\#1\}\{\#2\}}{gregoriotex.tex}
%\verb=\setfirstlineaboveinitial#1#2=%tex.tex
Macro to set the first (top) annotation above the initial.  This
macros allows you to control how far below the top of the staff \#1
(string containing contents of annotation) appears via the second
argument.  When its height is 0, the baseline of the annotation aligns
with the 4th bar line.  Positive values for the height push the
annotation down.  The argument cannot have a negative height.  \#2 can
be pretty much anything with a visible representation.
\verb=\newline= and \verb=\\= are not respected in either \#1 or \#2.
Note: large annotations which stick out above the staff will push the
commentary up.

\begin{argtable}
  \#1 & string & The text of the annotation.\\
  \#2 & string & Any text with a positive heigth.\\
\end{argtable}

Example:\par\medskip
\begin{latexcode}
  \setfirstlineaboveinitial{I g2}{I}
\end{latexcode}

\smallskip\hspace{15pt} Deprecated version: \verb=\gresetfirstlineaboveinitial=

\macroname{\textbackslash setsecondannotation\{\#1\}}{gregoriotex.tex}
%\verb=\setsecondannotation#1=%tex.tex
Macro to set the second (bottom) annotation above the initial.

\begin{argtable}
  \#1 & string & The text of the annotation.\\
\end{argtable}

Example:\par\medskip
\begin{latexcode}
  \setsecondannotation{III a}
\end{latexcode}

Can be used in conjunction with \verb=\setfirstannotation= to typeset
two lines of text above the initial.

\medskip
\begin{latexcode}
  \setfirstannotation{I Ant.}
  \setsecondannotation{IV a}
\end{latexcode}

\smallskip\hspace{15pt} Deprecated version: \verb=\gresetsecondannotation=

\macroname{\textbackslash scorereference}{gregoriotex.tex}
%\verb=\scorereference=%tex.tex
Does nothing.

\macroname{\textbackslash commentary\{\#1\}}{gregoriotex.tex}
%\verb=\commentary=%tex.tex
Marco to place the commentary (usually the scriptural reference) in
the top right-hand corner of the score.  While individual calls do not
support multiple lines, the macro can be called multiple times; each
call will typeset a new line.

\begin{argtable}
  \#1 & string & The text of the commentary.\\
\end{argtable}

Example:\par\medskip
\begin{latexcode}
  \commentary{Ps. 109:3}
\end{latexcode}

\macroname{\textbackslash removelines}{gregoriotex.tex}
%\verb=\removelines=%tex.tex
Macro to remove the staff lines.

\smallskip\hspace{15pt} Deprecated version: \verb=\greremovelines=

\macroname{\textbackslash donotremovelines}{gregoriotex.tex}
%\verb=\donotremovelines=%tex.tex
Macro to force staff lines (undoes the effects of
\verb=\removelines=).

\smallskip\hspace{15pt} Deprecated version: \verb=\gredonotremovelines=

\macroname{\textbackslash settrantlationcenteringscheme\{\#1\}}{gregoriotex.tex}
%\verb=\settranslationcenteringscheme#1=%tex.tex
Macro to change the centering scheme for the translation.

\begin{argtable}
  \#1 & 0 & Translation is left aligned with the corresponding text.\\
      & 1 & Translation is centered below the corresponding text.\\
\end{argtable}

Example:\par\medskip
\begin{latexcode}
  \settranslationcenteringscheme{1}
\end{latexcode}

\smallskip\hspace{15pt} Deprecated version: \verb=\setgretranslationcenteringscheme=

\macroname{\textbackslash setnlbintranslation\{\#1\}}{gregoriotex.tex}
%\verb=\setnlbintranslation#1=%tex.tex
Macro to change whether line breaks are allowed in the translations.

\begin{argtable}
  \#1 & 0 & Line breaks are allowed.\\
      & 1 & Line breaks are prohibited.\\
\end{argtable}

Example:\par\medskip
\begin{latexcode}
  \setnlbintranslation{0}
\end{latexcode}

\macroname{\textbackslash blockcustos}{gregoriotex.tex}
%\verb=\blockcustos=%tex.tex
Macro to block custos.  Applies to all subsequent scores in group.

\smallskip\hspace{15pt} Deprecated version: \verb=\greblockcustos=

\macroname{\textbackslash GreSetStaffLinerFormat\{\#1\}}{gregoriotex.tex}
%\verb=\GreSetStaffLinesFormat#1=%tex.tex
Deprecated.  Used to set the format for the staff lines.  See above
notes on the environments for updated documentation.

\macroname{\textbackslash includescore\{\#1\}}{gregoriotex.tex}
%\verb=\includescore#1=%tex.tex
Macro for including scores.  Works on both gabc and tex files.

\begin{argtable}
  \#1 & string & Relative or absolute path to the score.\\
\end{argtable}

Example:\par\medskip
\begin{latexcode}
  \includescore{TecumPrincipium.tex}
  \includescore{Chant/VirgoVirginum.gabc}
  \includescore{/home/user/chant/AdTeLevavi}

  %The following lines include the same score:
  \includescore{Christus}
  \includescore{Christus.gtex}
  \includescore{./Christus}
  \includescore{./Christus.gabc}
\end{latexcode}

\macroname{\textbackslash includetexscore\{\#1\}}{gregoriotex.tex}
%\verb=\includetexscore#1=%tex.tex
Macro for including scores which have already been run through
gregorio (i.e. are in gtex format).

\begin{argtable}
  \#1 & string & Relative or absolute path to the .\@gtex score.\\
\end{argtable}

\smallskip\hspace{15pt} Deprecated version: \verb=\greincludetexscore=

\macroname{\textbackslash includegabescore\{\#1\}}{gregoriotex.tex}
%\verb=\includegabcscore#1=%tex.tex
Macro for including scores which are in gabc format.  This macro will
run gregorio on the file before including it.

\begin{argtable}
  \#1 & string & Relative or absolute path to the .\@gabc score.\\
\end{argtable}

\smallskip\hspace{15pt} Deprecated version: \verb=\greincludegabcscore=

\macroname{\textbackslash GreUseNormalHyphen}{gregoriotex.tex}
%\verb=\GreUseNormalHyphen=%tex.tex
Tell gregoriotex to use normal hyphens at the end of lines (default
behavior).

Note: Placing an explicit hyphen in between special character brackets
(\ie <sp>-</sp>) in the gabc file will always result in a zero-width
hyphen.

Note: This command only affects the hyphens within a score.  Hyphens
elsewhere in a document are unaffected (hence the "Gre" in the command
name).

\macroname{\textbackslash GreUseZeroHyphen}{gregoriotex.tex}
%\verb=\GreUseZeroHyphen=%tex.tex
Tell gregoriotex to use zero-width hyphens at the end of lines.

Note: This command only affects the hyphens within a score.  Hyphens
elsewhere in a document are unaffected (hence the "Gre" in the command
name).

\macroname{\textbackslash GreForceHyphen}{gregoriotex.tex}
%\verb=\GreForceHyphen=%tex.tex
Tell gregoriotex to force the appearance of hyphens between all
syllables.

Note: This command only affects the hyphens within a score.  Hyphens
elsewhere in a document are unaffected (hence the "Gre" in the command
name).

\macroname{\textbackslash removeclef}{gregoriotex-signs.tex}
%\verb=\removeclef=%tex-signs.tex
Macro to remove the clefs from a score.

\smallskip\hspace{15pt} Deprecated version: \verb=\greremoveclef=

\macroname{\textbackslash normalclef}{gregoriotex-signs.tex}
%\verb=\normalclef=%tex-signs.tex
Macro to restore clefs to a score (undoes the effects of
\verb=\removeclef=).

\smallskip\hspace{15pt} Deprecated version: \verb=\grenormalclef=

\macroname{\textbackslash removecusto}{gregoriotex-signs.tex}
%\verb=\removecusto=%tex-signs.tex
Macro to empty the box containing the custos.
% I’m not sure what the purpose of this macro is.  Need to figure out
% where it is used.

\macroname{\textbackslash GreHidePCLines}{gregoriotex-signs.tex}
%\verb=\GreHidePCLines=%tex-signs.tex
Macro to tell gregoriotex to hide the lines through the middle of a
punctum cavum.  Default.

\macroname{\textbackslash GreDontHidePCLines}{gregoriotex-signs.tex}
%\verb=\GreDontHidePCLines=%tex-signs.tex
Macro to tell gregoriotex not to hide the lines through the middle of
a punctum cavum.

\macroname{\textbackslash GreHideAltLines}{gregoriotex-signs.tex}
%\verb=\GreHideAltLines=%tex-signs.tex
Macro to tell gregoriotex to hide the lines behind alterations.

\macroname{\textbackslash GreDontHideAltLines}{gregoriotex-signs.tex}
%\verb=\GreDontHideAltLines=%tex-signs.tex
Macro to tell gregoriotex not to hide the lines behind alterations.
Default.

\macroname{\textbackslash UseAlternatePunctumCavum}{gregoriotex-signs.tex}
%\verb=\UseAlternatePunctumCavum=%tex-signs.tex
Changes settings to use the alternative punctum cavum.

\macroname{\textbackslash UseNormalPunctumCavum}{gregoriotex-signs.tex}
%\verb=\UseNormalPunctumCavum=%tex-signs.tex
Changes settings to use the normal punctum cavum.  Default.

\macroname{\textbackslash englishcentering}{gregoriotex-syllable.tex}
Changes settings to align syllables in the English style (center of syllable aligns with center of first glyph in neume).  Will only affect scores which do not have “centering-scheme: english;” in their gabc header.

\macroname{\textbackslash defaultcentering}{gregoriotex-syllable.tex}
Changes settings to align syllables in the Latin style (center of vowel aligns with center of first glyph in neume).  Will only affect scores which do not have “centering-scheme: english;” in their gabc header.  Default.

%%% Local Variables:
%%% mode: latex
%%% TeX-master: "UserManual"
%%% End:
