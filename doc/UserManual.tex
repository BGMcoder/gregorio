\documentclass[12pt]{article}
\usepackage[utf8]{inputenc}
\usepackage[T1]{fontenc}
\usepackage{lmodern}
\usepackage{enumitem}
\usepackage{xspace}
\usepackage{multicol}

\usepackage{makeidx}
\makeindex

\usepackage[table]{xcolor}
\definecolor{lightgray}{gray}{0.9}
\definecolor{green}{HTML}{0c700c}
\definecolor{myred}{HTML}{FF3333}

\usepackage{minted} % must be after xcolor
\newminted{latex}{gobble=2,bgcolor=lightgray}

\usepackage{hyperref}
\hypersetup{colorlinks, citecolor=black, filecolor=black, linkcolor=green, urlcolor=green}

\newcommand*{\eg}{e.g.\@\xspace}
\newcommand*{\nb}{n.b.\@\xspace}
\newcommand*{\ie}{i.e.\@\xspace}

\newcommand{\macroname}[2]{\vspace{3.25ex plus 1ex minus .2ex}%
                           \makebox[\linewidth]{\ttfamily\bfseries #1%
                             \hspace{\fill}\normalfont\itshape #2}%
                         \vspace{1.5ex plus .2ex}%
                         \index{#1}}
                       %\addcontentsline{toc}{subsubsection}{#1}}


\newenvironment{argtable}{\bigskip\rowcolors{1}{lightgray}{lightgray}
                           \begin{tabular}{clp{10cm plus .5cm}}
                             Arg & Value & Description \\
                             \hline}%
                          {\end{tabular}\bigskip}

\setlength{\parindent}{0mm} % Default is 15pt

\begin{document}

\begin{titlepage}
  \begin{center}
    \Huge
    \textcolor{myred}{Gregorio} and \textcolor{myred}{gregoriotex}:

    A project which provides tools for gregorian score engraving.

    \vspace{1cm}\large\href{http://home.gna.org/gregorio/}{Homepage}

    Source code available on
    \href{http://github.com/gregorio-project/gregorio}{GitHub}.
  \end{center}
  \vspace{2cm}
  \begin{multicols}{2}
    \tableofcontents
  \end{multicols}
\end{titlepage}

\section{Gregoriotex Macros}
The following sections document the macros available in the gregoriotex package. The format is as follows:

\macroname{Macroname}{source file}

\begin{argtable}
  Arg \# & Data type & Descriptor\\
\end{argtable}

The source file where the macro is defined is included for developers
who wish to consult it.\\
Some of the macros intended for inclusion in the main.tex file by the user
include usage examples.

% !TEX root = UserManual.tex
\section{User Controls}

These functions are available to the user to customize elements of the
score which cannot be controlled from the gabc file. They can be added
to any \verb=.tex= file. Do not add them to any \verb=.gtex= or
\verb=.gabc= file.

\subsection{Commands}

In general, commands should not be modified.  Exceptions are noted below.

\macroname{\textbackslash usepackage[\textit{(options)}]\{gregoriotex\}}{gregoriotex.sty}

Usepackage macro for gregoriotex. Has the following optional arguments:

\bigskip\rowcolors{1}{lightgray}{lightgray}
\begin{tabular}{lp{10cm plus .5cm}}
  Argument & Description \\
  \hline
  nevercompile & Default. The classic behavior of gregoriotex. The user is
                 responsible for compiling gabc scores into gtex files.\\
  autocompile & Gregoriotex will automatically compile gtex files from gabc
                files when necessary. If the gabc has been modified, or the
                gtex has an outdated version, or the gtex file does not exist,
                THEN gregoriotex will compile a new gtex file.\\
  forcecompile & Gregoriotex will compile all scores from their gabc files.\\
\end{tabular}\bigskip

\macroname{\textbackslash includescore[\textit{\#1}]\{\#2\}}{gregoriotex.tex}
%\verb=\includescore#1=%tex.tex
Macro for including scores.  Works on both gabc and tex files.

\begin{argtable}
  \#1 & \texttt{n} & Optional. Bypasses compilation checks.\\
      & \texttt{c} & Optional. Forces gregoriotex to compile the gabc file.\\
  \#2 & string & Relative or absolute path to the score.\\
\end{argtable}

Example:\par\medskip
\begin{latexcode}
  \includescore[n]{TecumPrincipium.tex}
  \includescore{Chant/VirgoVirginum.gabc}
  \includescore{/home/user/chant/AdTeLevavi}
  \includescore[c]{AveMaria}

  %The following lines include the same score:
  \includescore{Christus}
  \includescore{Christus.gtex}
  \includescore{./Christus}
  \includescore{./Christus.gabc}

  %With the optional arg, #2 must be a file usable by TeX.
  \includescore[n]{TecumPrincipium.gabc} % Wrong
\end{latexcode}

For the sake of clarity it is suggested that the file extension be
omitted from \texttt{\#2}.

When called with the optional argument \texttt{[c]} gregoriotex will
automatically generate a \texttt{gtex} file in this format:
\texttt{\textit{scorename}-x\_x\_x.gtex} where \texttt{x\_x\_x} is the
gregotio version.

When called with the optional argument \texttt{[n]} gregoriotex will
include the score without doing anything else. This is the same as the
old behavior of gregoriotex and the default behavior.

\macroname{\textbackslash forcecompilegabc}{gregoriotex.tex} A switch
to change the behavior of the way gregoriotex includes scores. When
used, all later calls of \verb=\includescore= will compile the gabc
file into a gtex file. This is similar to using the package option
\verb=[forcecompile]=, but does not necessarly apply to the entire
document.

\macroname{\textbackslash autocompilegabc}{gregoriotex.tex} A switch
to change the behavior of the way gregoriotex includes scores. When
used, all later calls of \verb=\includescore= will use gregoriotex's
automatic compilation of gabc files. This is similar to using the
packape option \verb=[autocompile]=, but does not necessarly apply to
the entire document.

\macroname{\textbackslash nevercompilegabc}{gregoriotex.tex} A switch
to change the behavior of the way gregoriotex includes scores. When
used, all later calls of \verb=\includescore= will include the score
without doing anything else. This is the same as the traditional
behavior of gregoriotex. It is the package default.

\medskip The three previous macros can be combined in the same document to
switch between the different includescore behaviors: \par\medskip
\begin{latexcode}
  \usepackage{gregoriotex}
  ----
  \includescore{TecumPrincipium} % gabc never compiled.
  \includescore[c]{TecumPrincipium} % gabc always compiled.

  \autocompile
  \includescore{TecumPrincipium} % gabc auto compiled.
  \includescore[n]{TecumPrincipium} % gabc never compiled.
  \includescore[c]{TecumPrincipium} % gabc always compiled.

  \forcecompile
  \includescore{TecumPrincipium} % gabc auways compiled.
  \includescore[n]{TecumPrincipium} % gabc never compiled.
\end{latexcode}

\macroname{\textbackslash englishcentering}{gregoriotex-syllable.tex}
Changes settings to align syllables in the English style (center of
syllable aligns with center of first glyph in neume).  Will only
affect scores which do not have “centering-scheme: english;” in their
gabc header.

\macroname{\textbackslash defaultcentering}{gregoriotex-syllable.tex}
Changes settings to align syllables in the Latin style (center of
vowel aligns with center of first glyph in neume).  Will only affect
scores which do not have “centering-scheme: english;” in their gabc
header.

Default.

\macroname{\textbackslash gresetdim\{\#1\}\{\#2\}\{\#3\}}{gregoriotex-spaces.tex}
Macro to set one of gregoriotex’s distances.  Does not update dependent distances.  Used primarily to initialize distances in a space configuration file.  This function if \texttt{gre} prefixed to highlight that it should not be used for distances not defined in Gregorio\TeX.

\begin{argtable}
\#1 & string & The name of the distance to be changed.  See \nameref{distances} below.\\
\#2 & string & The distance in string format.  \textbf{Note:} You cannot use a length register for this argument.  You \emph{must} use a string because of the way that gregoriotex handles spaces.\\
\#3 & 0 & Distance will not scale when staff size is changed.\\
& 1 & Distance will scale when staff size is changed.
\end{argtable}

\macroname{\textbackslash grechangedim\{\#1\}\{\#2\}\{\#3\}}{gregoriotex-spaces.tex}
Macro to change one of gregoriotex’s distances and update any dependent distances.  This function if \texttt{gre} prefixed to highlight that it should not be used for distances not defined in Gregorio\TeX.

\begin{argtable}
\#1 & string & The name of the distance to be changed.  See \nameref{distances} below.\\
\#2 & string & The distance in string format.  \textbf{Note:} You cannot use a length register for this argument.  You \emph{must} use a string because of the way that gregoriotex handles spaces.\\
\#3 & 0 & Distance will not scale when staff size is changed.\\
& 1 & Distance will scale when staff size is changed.
\end{argtable}

\macroname{\textbackslash grenoscaledim\{\#1\}}{gregoriotex-spaces.tex}
Macro to turn off scaling for a particular distance.  This function if \texttt{gre} prefixed to highlight that it should not be used for distances not defined in Gregorio\TeX.

\begin{argtable}
  \#1 & string & The name of the distance for which scaling it to be turned off.  See \nameref{distances} below.
\end{argtable}

\macroname{\textbackslash grescaledim\{\#1\}}{gregoriotex-spaces.tex}
Macro to turn on scaling for a particular distance.  This function if \texttt{gre} prefixed to highlight that it should not be used for distances not defined in Gregorio\TeX.

\begin{argtable}
  \#1 & string & The name of the distance for which scaling it to be turned on.  See \nameref{distances} below.
\end{argtable}

\macroname{\textbackslash GreLoadSpaceConf\{\#1\}}{gregoriotex-spaces.tex}
Macro to load a space configuration file.  Space configuration file names have the format \verb=gsp-identifier.tex= and must be in the same directory as your project or in your texmf directory.  This function if \texttt{gre} prefixed to highlight that it only loads distances for Gregorio\TeX.

\begin{argtable}
\#1 & string & The identifier of the space configuration file.
\end{argtable}

Example:\par\medskip
\begin{latexcode}
  % loads gsp-default.tex, the default configuration file
  \GreLoadSpaceConf{default}
  % loads a custom configuration called gsp-myspaces.tex
  \GreLoadSpaceConf{myspaces}
\end{latexcode}

\macroname{\textbackslash setaboveinitialseparation\{\#1\}\{\#2\}}{gregoriotex-spaces.tex}
Macro to set the spacing between the annotations.

\begin{argtable}
\#1 & string & The distance in string format.  \textbf{Note:} You cannot use a length register for this argument.  You \emph{must} use a string because of the way that gregoriotex handles spaces.\\
\#2 & 0 & Distance will not scale when staff size is changed.\\
& 1 & Distance will scale when staff size is changed.
\end{argtable}

\textbf{Warning:} This function formerly had only one arguments.  It now takes 2.

Deprecated version: \verb=\GreSetAboveInitialSeparation= which only took the first argument.  If you use the deprecated command the second argument will be assumed to be 1.

\macroname{\textbackslash setspaceafterinitial\{\#1\}\{\#2\}}{gregoriotex-spaces.tex}
Macro to set the spacing after the initial (and before the staff lines).

\begin{argtable}
\#1 & string & The distance in string format.  \textbf{Note:} You cannot use a length register for this argument.  You \emph{must} use a string because of the way that gregoriotex handles spaces.\\
\#2 & 0 & Distance will not scale when staff size is changed.\\
& 1 & Distance will scale when staff size is changed.
\end{argtable}

\textbf{Warning:} This function formerly had only one arguments.  It now takes 2.

Deprecated version: \verb=\GreSetSpaceAfterInitial= which only took the first argument.  If you use the deprecated command the second argument will be assumed to be 1.

\macroname{\textbackslash setspacebeforeinitial\{\#1\}\{\#2\}}{gregoriotex-spaces.tex}
Macro to set the spacing before the initial.

\begin{argtable}
\#1 & string & The distance in string format.  \textbf{Note:} You cannot use a length register for this argument.  You \emph{must} use a string because of the way that gregoriotex handles spaces.\\
\#2 & 0 & Distance will not scale when staff size is changed.\\
& 1 & Distance will scale when staff size is changed.
\end{argtable}

\textbf{Warning:} This function formerly had only one arguments.  It now takes 2.

Deprecated version: \verb=\GreSetSpaceBeforeInitial= which only took the first argument.  If you use the deprecated command the second argument will be assumed to be 1.

\macroname{\textbackslash setinitialspacing\{\#1\}\{\#2\}\{\#3\}\{\#4\}}{gregoriotex-spaces.tex}
Macro to set the spacing around and for the initial.

\begin{argtable}
\#1 & string & The distance before the initial in string format.  \textbf{Note:} You cannot use a length register for this argument.  You \emph{must} use a string because of the way that gregoriotex handles spaces.\\
\#2 & string & The distance for the initial in string format.  This distance is ignored if set to 0pt.  \textbf{Note:} You cannot use a length register for this argument.  You \emph{must} use a string because of the way that gregoriotex handles spaces.\\
\#3 & string & The distance after the initial in string format.  \textbf{Note:} You cannot use a length register for this argument.  You \emph{must} use a string because of the way that gregoriotex handles spaces.\\
\#4 & 0 & Distance will not scale when staff size is changed.\\
& 1 & Distance will scale when staff size is changed.
\end{argtable}

\textbf{Warning:} This function formerly had only three arguments.  It now takes 4.

\macroname{\textbackslash setstafflinethickness\{\#1\}}{gregoriotex-spaces.tex}
Macro to adjust the thickness of the staff lines.

\begin{argtable}
\#1 & integer & The relative thickness of the staff lines.  The default value is 10.  Higher numbers yield thicker staff lines.
\end{argtable}

Deprecated version: \verb=\gresetstafflinefactor=.




\subsection{Distances}\label{distances}

Each of the following distances controls some aspect of the spacing of the Gregorio score.  They are changed using commands documented above (\eg \verb=\changedim=).  If the distance permits a rubber value then the default value will indicate the stretch and shrink (even if they are zero by default).  Distances whose default value does not include a stretch or shrink may not take a rubber value.

\emph{Note:} Because of the way Gregorio handles distances these cannot be manipulated as if they were normal \TeX\ dimensions or skips.  As a result they should only be changed using the commands defined by gregoriotex for this purpose.

\macroname{additionallineswidth}{gsp-default.tex}
The additional width of the additional lines (compared to the width of the glyph they're associated with).  

Default: \unit[0.14584]{cm}

\macroname{alterationspace}{gsp-default.tex}
Space between an alteration (flat or natural) and the next glyph. 

Default: \unit[0.07747]{cm} plus \unit[0.01276]{cm} minus \unit[0.00455]{cm}

\macroname{beforealterationspace}{gsp-default.tex}
Negative space, difference between the normal space between two notes and the space between a note and a flat.  

Default: $\unit[-0.32816]{cm}$ plus \unit[0.01093]{cm} minus \unit[0.01093]{cm}

\macroname{beforechoralsignspace}{gsp-default.tex}
Space before a choral sign. 

Default: \unit[0.04556]{cm} plus \unit[0.00638]{cm} minus \unit[0.00638]{cm}

\macroname{clefflatspace}{gsp-default.tex}
Space between a clef and a flat (for clefs with flat).  

Default: \unit[0.05469]{cm} plus \unit[0.00638]{cm} minus \unit[0.00638]{cm}

\macroname{interglyphspace}{gsp-default.tex}
Space between glyphs in the same element. 

Default: \unit[0.06927]{cm} plus \unit[0.00363]{cm} minus \unit[0.00363]{cm}

\macroname{zerowidthspace}{gsp-default.tex}
Null space.  

Default: \unit[0]{cm} plus \unit[0]{cm} minus \unit[0]{cm}

\macroname{interelementspace}{gsp-default.tex}
Space between elements.  

Default: \unit[0.06927]{cm} plus \unit[0.00182]{cm} minus \unit[0.00363]{cm}

\macroname{largerspace}{gsp-default.tex}
Larger space between elements.  

Default: \unit[0.10938]{cm} plus \unit[0.01822]{cm} minus \unit[0.00911]{cm}

\macroname{glyphspace}{gsp-default.tex}
Space between elements which has the size of a note.  

Default: \unit[0.21877]{cm} plus \unit[0.01822]{cm} minus \unit[0.01822]{cm}

\macroname{intersyllablespace}{gsp-default.tex}
Minimum space between two notes of different syllables.  

Default: \unit[0.25523]{cm} plus \unit[0.31903]{cm} minus \unit[0]{cm}

\macroname{spacebeforecusto}{gsp-default.tex}
Space before custo.  

Default: \unit[0.1823]{cm} plus \unit[0.31903]{cm} minus \unit[0.0638]{cm}

\macroname{spacebeforesigns}{gsp-default.tex}
Space before punctum mora and augmentum duplex.  

Default: \unit[0.05469]{cm} plus \unit[0.00455]{cm} minus \unit[0.00455]{cm}

\macroname{spaceaftersigns}{gsp-default.tex}
Space after punctum mora and augmentum duplex.  

Default: \unit[0.08203]{cm} plus \unit[0.0082]{cm} minus \unit[0.0082]{cm}

\macroname{spaceafterlineclef}{gsp-default.tex}
Space after a clef at the beginning of a line.  

Default: \unit[0.27345]{cm} plus \unit[0.14584]{cm} minus \unit[0.01367]{cm}

\macroname{interwordspacenotes}{gsp-default.tex}
Space after at the end of a word when the last written symbol is a note and the first is a note.  

Default: \unit[0.29169]{cm} plus \unit[0.08751]{cm} minus \unit[0.05469]{cm}

\macroname{interwordspacenotestext}{gsp-default.tex}
Space after at the end of a word when the last written symbol is a note and the first is text.  

Default: \unit[0.27345]{cm} plus \unit[0.27345]{cm} minus \unit[0.07292]{cm}

\macroname{interwordspacetextnotes}{gsp-default.tex}
Space after at the end of a word when the last written symbol is text and the first is a note.  

Default: \unit[0.27345]{cm} plus \unit[0.27345]{cm} minus \unit[0.07292]{cm}

\macroname{interwordspacetext}{gsp-default.tex}
Space after at the end of a word when the last written symbol is text and the first is text.  

Default: \unit[0.22787]{cm} plus \unit[0.41019]{cm} minus \unit[0.07292]{cm}

\macroname{bitrivirspace}{gsp-default.tex}
Space between notes of a bivirga or trivirga.  

Default: \unit[0.06927]{cm} plus \unit[0.00182]{cm} minus \unit[0.00546]{cm}

\macroname{bitristrospace}{gsp-default.tex}
Space between notes of a bistropha or tristrophae.  

Default: \unit[0.06927]{cm} plus \unit[0.00182]{cm} minus \unit[0.00546]{cm}

\macroname{punctuminclinatumshift}{gsp-default.tex}
Space between two punctum inclinatum.  

Default: \unit[-0.03918]{cm} plus \unit[0.0009]{cm} minus \unit[0.0009]{cm}

\macroname{beforepunctainclinatashift}{gsp-default.tex}
Space before puncta inclinata.  

Default: \unit[0.05286]{cm} plus \unit[0.00728]{cm} minus \unit[0.00455]{cm}

\macroname{punctuminclinatumanddebilisshift}{gsp-default.tex}
Space between a punctum inclinatum and a punctum inclinatum deminutus.  

Default: \unit[-0.02278]{cm} plus \unit[0.0009]{cm} minus \unit[0.0009]{cm}

\macroname{punctuminclinatumdebilisshift}{gsp-default.tex}
Space between two punctum inclinatum deminutus.  

Default: \unit[-0.00728]{cm} plus \unit[0.0009]{cm} minus \unit[0.0009]{cm}

\macroname{punctuminclinatumbigshift}{gsp-default.tex}
Space between puncta inclinata, larger ambitus (range=3rd).  

Default: \unit[0.07565]{cm} plus \unit[0.0009]{cm} minus \unit[0.0009]{cm}

\macroname{punctuminclinatummaxshift}{gsp-default.tex}
Space between puncta inclinata, larger ambitus (range=4th -or more?-).  

Default: \unit[0.17865]{cm} plus \unit[0.0009]{cm} minus \unit[0.0009]{cm}

\macroname{spacearoundsmallbar}{gsp-default.tex}
Space around virgula and divisio minima.  

Default: \unit[0.1823]{cm} plus \unit[0.22787]{cm} minus \unit[0.05469]{cm}

\macroname{spacearoundminor}{gsp-default.tex}
Space around divisio minor.  

Default: \unit[0.1823]{cm} plus \unit[0.22787]{cm} minus \unit[0.05469]{cm}

\macroname{spacearoundmaior}{gsp-default.tex}
Space around divisio maior.  

Default: \unit[0.1823]{cm} plus \unit[0.22787]{cm} minus \unit[0.05469]{cm}

\macroname{spacearoundfinalis}{gsp-default.tex}
Space around divisio finalis.  

Default: \unit[0.1823]{cm} plus \unit[0.1823]{cm} minus \unit[0.05469]{cm}

\macroname{spacebeforefinalfinalis}{gsp-default.tex}
A special space for finalis, for when it is the last glyph.  

Default: \unit[0.29169]{cm} plus \unit[0.07292]{cm} minus \unit[0.27345]{cm}

\macroname{spacearoundclefbars}{gsp-default.tex}
Additional space that will appear around bars that are preceded by a custo and followed by a key.  

Default: \unit[0.03645]{cm} plus \unit[0.00455]{cm} minus \unit[0.0009]{cm}

\macroname{textbartextspace}{gsp-default.tex}
Space between the text and the text of the bar.  

Default: \unit[0.24611]{cm} plus \unit[0.13672]{cm} minus \unit[0.04921]{cm}

\macroname{notebarspace}{gsp-default.tex}
Minimal space between a note and a bar.  

Default: \unit[0.31903]{cm} plus \unit[0.27345]{cm} minus \unit[0.02824]{cm}

\macroname{maximumspacewithoutdash}{gsp-default.tex}
Maximal space between two syllables for which we consider a dash is not needed.  

Default: \unit[0.02005]{cm}

\macroname{afterclefnospace}{gsp-default.tex}
An extensible space for the beginning of lines.  

Default: \unit[0]{cm} plus \unit[0.27345]{cm} minus \unit[0]{cm}

\macroname{additionalcustoslineswidth}{gsp-default.tex}
Width of the additional lines, used only for the custos.  The width is the one for the custos at end of lines, the line for custos in the middle of a score is the same multiplied by 2.  

Default: \unit[0.09114]{cm}

\macroname{afterinitialshift}{gsp-default.tex}
Space between the initial and the beginning of the score.  

Default: \unit[0.2457]{cm} plus \unit[0]{cm} minus \unit[0]{cm}

\macroname{beforeinitialshift}{gsp-default.tex}
Space between the initial and the beginning of the score.  

Default: \unit[0.2457]{cm} plus \unit[0]{cm} minus \unit[0]{cm}

\macroname{minimalspaceatlinebeginning}{gsp-default.tex}
Minimal space at the beginning of a line.

Default: \unit[1.7]{cm}

\macroname{manualinitialwidth}{gsp-default.tex}
Space to force the initial width to.  Ignored when 0.  

Default: \unit[0]{cm}

\macroname{aboveinitialseparation}{gsp-default.tex}
This space is the one between the bottom of the first annotation line and the top of the second annotation line (above the initial).  

Default: \unit[0.85]{cm}

\macroname{noclefspace}{gsp-default.tex}
Space at the beginning of the lines if there is no clef.  

Default: \unit[0.1]{cm}

\macroname{abovesignsspace}{gsp-default.tex}
Space above the chironomic signs.  

Default: \unit[0.8]{cm}

\macroname{belowsignsspace}{gsp-default.tex}
Space below the chironomic signs.  

Default: \unit[0]{cm}

\macroname{lowchoralsignshift}{gsp-default.tex}
The shift for the low choral sign and the high choral signs on totally free interlines.  

Default: \unit[0.00911]{cm}

\macroname{highchoralsignshift}{gsp-default.tex}
The shift for the high choral sign.  

Default: $\unit[-0.04556]{cm}$

\macroname{translationheight}{gsp-default.tex}
The space for the translation.  

Default: \unit[0.5]{cm}

\macroname{spaceabovelines}{gsp-default.tex}
The space above the lines.  

Default: \unit[0.45576]{cm} plus \unit[0.36461]{cm} minus \unit[0.09114]{cm}

\macroname{spacelinestext}{gsp-default.tex}
The space between the lines and the bottom of the text.  

Default: \unit[0.60617]{cm} plus \unit[0]{cm} minus \unit[0]{cm}

\macroname{spacebeneathtext}{gsp-default.tex}
The space beneath the text.  

Default: \unit[0]{cm} plus \unit[0]{cm} minus \unit[0]{cm}

\macroname{abovelinestextraise}{gsp-default.tex}
Height of the text above the note line.  

Default: \unit[1.7]{cm}

\macroname{abovelinestextheight}{gsp-default.tex}
Height that is added at the top of the lines if there is text above the lines (it must be bigger than the text for it to be taken into consideration).  

Default: \unit[0.3]{cm}

\macroname{braceshift}{gsp-default.tex}
An additional shift you can give to the brace above the bars.  

Default: \unit[0]{cm}

\macroname{curlybraceaccentusshift}{gsp-default.tex}
A shift you can give to the accentus above the curly brace.  

Default: $\unit[-0.05]{cm}$

%%% Local Variables:
%%% mode: latex
%%% TeX-master: "UserManual"
%%% End:

\section{Gregoriotex Controls}

These functions are the ones used by Gregoriotex internally as it process the commands listed above.  They should not appear in any user document and are listed here for programmer documentation purposes only.

\verb=\gre@coloredlines#1=%tex.sty
	Changes the color of the staff lines to \#1.  \#1 must be a named color defined using \verb=\definecolor=

\verb=\gre@redlines=%tex.sty
	Changes the color of the staff lines to \verb=gregoriocolor=.

\verb=\gre@normallines=%tex.sty
	Resets the formatting of the staff lines.

\verb=\gre@error#1=%tex.tex
	Raises an error which is identified as coming from GregorioTeX.  Uses \LaTeX \verb=\PackageError= or PlainTeX \verb=\errmessage= as appropriate.  \#1 (a string) is the accompanying message.
	
\verb=\gre@warn#1=%tex.tex
	Raises a warning which is identified as coming from GregorioTeX.  Uses \LaTeX \verb=\PackageWarning= or PlainTeX \verb=\message= as appropriate.  <string is the accompanying message.
	
\verb=\gre@localleftbox=%tex.tex
	Alias for \verb=\luatexlocalleftbox=.  Used to make propagating changes in latex easier.

\verb=\gre@localrightbox=%tex.tex
	Alias for \verb=\luatexlocalrightbox=.  Used to make propagating changes in latex easier.
	
\verb=\gre@unsetattribute=%tex.tex
	Alias for \verb=\unsetlutexattribute{\gregorioattr}=

\verb=\gregorioattr=%tex.tex
	A luatex attribute we put on the text nodes.
	If it is 1, it means that there may be a dash here if this syllable is at the end of a line.
	If it is 2, it means that it's never useful to typeset a dash.
	If it is 0, it just means that we are in a score.

\verb=\gregoriocenterattr=%tex.tex
	A luatex attribute used for translation centering.

\verb=\gregoriotexversion#1=%tex.tex
	Defines the current version of GregorioTeX API.  \#1 is an integer date in format: yyyymmdd

\verb=\gre@internalversion#1=%tex.tex
	The version of GregorioTeX.  \#1 is an integer date in format: yyyymmdd

\verb=\gre@declarefileversion#1#2=%tex.tex
	Checks to see if GregorioTeX component \#1 (a string containing the name) with version \#2 (an integer in date format yyyymmdd) is compatible with \verb=\gre@internalversion=

\verb=\gre@factor=%tex.tex
	Count representing the size of the staff.  Initialized to 0, but changed to 17 by \verb=\begingregorianscore= if user hasn't changed it.
	
\verb=\gre@stafflinewidth=%tex.tex
	Dimension representing the width of a line of staff.  Can vary, for example, at the first line.

\verb=\gre@linewidth=%tex.tex
	Dimension representing the width of the score (including initial).

\verb=\gre@calculateconstantglyphraise=%tex.tex
	Macro to caluclate \verb=\gre@constantglyphraise=

\verb=\gre@constantglyphraise=%tex.tex
	Dimension representing ??

\verb=\gre@currenttranslationheight=%tex.tex
	Dimension representing the space for the translation beneath the text.

\verb=\gre@addtranslationspace=%tex.tex
	Macro to tell Gregorio to set space for the translation.

\verb=\gre@removetranslationspace=%tex.tex
	Macro to tell Gregorio to remove the space allocated to the translation.

\verb=\gre@kernbeforeeol=%tex.tex
	Macro describing a kern to make before ending the line, which we sometimes want (see \verb=\gresyllable=)

\verb=\gre@newlinecommon#1#2#3#4#5=%tex.tex
	Macro we call each time we force a changing of line, it automatically sets \verb=\greknownline=, and adjusts left spaces.
	\#1 --- integer: 0 = no note above staff; 1 = note above 4th line (position k); 2 = note on 5th line (position l); 3 = note above 5th line (position m)
	\#2 --- integer: 0 = no note below staff; 1 note below 1st line (position c); 2 = note on 0th line (position b); 3 = note below 0th line (position a); 4 = note below 0th line (position a) with vertical episemus
	\#3 --- integer: 0 = no translation; 1 = translation present
	\#4 --- integer: 0 = justify line; 1 = do not justify line
	\#5 --- integer: 0 = no extra space above staff; 1 = extra space above staff

\verb=\gre@additionalbottomspace=%tex.tex
	Dimension representing extra space below the staff needed for low notes.

\verb=\gre@additionaltopspace=%tex.tex
	Dimension representing extra space above the staff needed for high notes.

\verb=\gre@updateadditionalspaces#1#2=%tex.tex
	Macro which updates \verb=\gre@additionalbottomspace= and \verb=\gre@additionaltopspace=
%	\#1 --- integer: 0 = no note above staff; 1 = note above 4th line (position k); 2 = note on 5th line (position l); 3 = note above 5th line (position m)
%	\#2 --- integer: 0 = no note below staff; 1 note below 1st line (position c); 2 = note on 0th line (position b); 3 = note below 0th line (position a); 4 = note below 0th line (position a) with vertical episemus

\verb=\gre@textlower=%tex.tex
	Dimension representing the height of the separation between the 0th line (which is invisible except for notes in the a or b position) and the bottom of the text.

\verb=\gre@Tempwidth=%tex.tex
	Box used to calculate \verb=\gre@tempwidth=
	
\verb=\gre@tempwidth=%tex.tex
	Dimension representing width of some element.

\verb=\gre@widthof#1=%tex.tex
	Macro which calculates \verb=\gre@tempwidth= as width of \#1.

\verb=\gre@textaligncenter=%tex.tex
	Dimension representing the width from the beginning of the letters in a syllable to the middle of the middle letters.  Used for lining up neumes and syllables.
	
\verb=\gre@findtextaligncenter#1#2#3=%tex.tex
	Macro for calculating \verb=\gre@textaligncenter=.
	\#1 --- string: The first part of the syllable (any preceding consonants in Latin)
	\#2 --- string: The middle part of the syllable (the vowel in Latin)
	\#3 --- integer: syllable the calculation is being performed for; 0 = current syllable; 1 = next syllable

\verb=\gre@additionalleftspace=%tex.tex
	Dimension representing the additional space that has to be added to the localleftbox for a big initial (one taking two lines).

\verb=\Gre@Initial=%tex.tex
	Box containing the initial.

\verb=\gre@initialwidth=%tex.tex
	Dimension representing the width of the initial (and the space after).

\verb=\gre@biginitial=%tex.tex
	Count indicating whether initial takes 2 lines: 0 = it doesn't; 1 = it does.

\verb=\gre@knowline=%tex.tex
	Line gregoriotex thinks is being set.

\verb=\gre@updateleftbox=%tex.tex
	Macro which adjusts size of line and clef placement after initial.

\verb=\gre@updatelinewidth=%tex.tex
	Macro which adjusts the width of the line.

\verb=\gre@initialformat#1=%tex.tex
	Macro which applies formatting in \verb=initialformat= environment to initial (\#1, a character).

\verb=\gre@biginitialformat#1=%tex.tex
	Macro which applies formatting in \verb=biginitialformat= environment to initial (\#1, a character).

\verb=\Gre@Aboveinitialfirstbox=%tex.tex
	The box which contains the first (top) annotation above the initial.

\verb=\gre@aboveinitialfirstraise=%tex.tex
	Dimension representing the space allocated to the first annotation.

\verb=\Gre@Aboveinitialsecondbox=%tex.tex
	The box which contains the second (bottom) annotation above the initial.

\verb=\gre@aboveinitialsecondraise=%tex.tex
	Dimension representing the space allocated to the second annotation.

\verb=\gre@setaboveinitialrais=%tex.tex
	Macro to give \verb=\gre@aboveinitialfirstraise= and \verb=\gre@aboveinitialsecondraise= their working values.
	
\verb=\gre@currentabovelinestextheight=%tex.tex
	Dimension representing the space allocated above the lines for text.	

\verb=\gre@abovelinestextstyle#1=%tex.tex
	Macro to apply the above line text style to argument (a character).

\verb=\gre@addspaceabove=%tex.tex
	Macro to allocate space for text placed above the staff lines.

\verb=\gre@removespaceabove=%tex.tex
	Macro to unallocate space for text above the staff lines.

\verb=\gre@currenttextabovelines#1=%tex.tex
	Macro containing the text (\#1) which is currently being placed above the staff lines.

\verb=\gre@typsettextabovelines#1=%tex.tex
	Macro to typeset argument (a string) above the staff lines.

\verb=\gre@removelinescount=%tex.tex
	Boolean indicating whether staff lines should be printed.  Has value 0 when lines are printed, 1 when they are not.

\verb=\gre@drawfirstlines=%tex.tex
	Macro to draw the staff lines for the first line of the score (\ie it accounts for the space taken up by the initial).

\verb=\Gre@Lines=%tex.tex
	Box to contain the staff lines for lines other than the first line.

\verb=\gre@generatelines=%tex.tex
	Macro to fill \verb=\Gre@Lines=.

\verb=\gre@smallsecondline=%tex.tex
	Macro called when the initial is big to make the second set of staff lines the same length as the first.

\verb=\gre@normallines=%tex.tex
	Marco called after the second set of staff lines when the initial is big to go back to normal width lines.

\verb=\gre@translationcenteringscheme=%tex.tex
	Boolean to indicate the centering scheme used for the translation.
	Possible values: 0=translation is left aligned with corresponding text; 1=translation is centered with the corresponding text

\verb=\gre@nlbintranslation=%tex.tex
	Variable used to indicate whether line breaks are allowed in the translation.
	Possible values: 0=line breaks are allowed; 1=line breaks are prohibited

\verb=\gre@translationformat#1=%tex.tex (is this defined somewhere else too?)
	Macro to apply the translation format to string.

\verb=\gre@mustdotranslationcenterend=%tex.tex
	Boolean to indicate if the translation is at the end of a line?
	Possible values: 0=not at the end of a line; 1=at the end of a line

\verb=\gre@dotranslationcenterend=%tex.tex
	Macro to set things up for a translation at the end of a line.

\verb=\gre@printchirovbars=%tex.tex
	Count that is 1 or 0 if we need to print the small vertical bars in the chironomic line.

\verb=\gre@nolastline=%tex.tex
	Macro to call when there is just a little thing that will go to the last line, when it is not necessary
	It doesn't seem to be used, so it's a good candidate for deprecation!

\verb=\gre@endofword#1=%tex.tex
	Macro called at the end of each word.  Does extra stuff when argument (an integer) is 1?

\verb=\gre@endbeforebar#1=%tex.tex
	Macro called at end of word when next element is a bar.  Argument (an integer) has same function as in \verb=\gre@endofword=.

\verb=\gre@endafterbar#1=%tex.tex
	Macro called after a bar.  Argument (an integer) has same function as in \verb=\gre@endofword=.

\verb=\gre@lastoflinecount=%tex.tex
	Count which marks the last syllable of the line.
	Possible values: 0=nothing; 1=last syllable of line; 2=first syllable of line

\verb=\gre@disableeolshifts=%tex.tex
	Boolean which indicates that the syllable should be shifted left a bit.
	Possible values: 0=shift happens; 1=shift doesn't happen.

\verb=\Gre@DisableEOLShifts=%tex.tex
	Macro to set \verb=\gre@disableeolshifts= to 1.

\verb=\Gre@EnableEOLShifts=%tex.tex
	Macro to set \verb=\gre@disableeolshifts= to 0.

\verb=\gre@blockcusto=%tex.tex
	Count to indicate if the custo should be blocked.
	Possible values: 0=do not block custo; 1=block custo

\verb=\gre@endofsyllable=%tex.tex
	Macro called at the end of a syllable which does not end a word.

\verb=\gre@nlbstate=%tex.tex
	Count to indicate the no line break areas.
	Possible values: 0 = not a no line break area; 1 = no line break due to translation centering; 2 = no line break due to <nlba> tag

\verb=\gregoriofontname=%tex.tex
	Macro which holds the default font name (greciliae).

\verb=\gre@usestylefont=%tex.tex
	Count to indicate if gregoriostylefont should be used.
	Possible values: 0 = do not use; 1 = use

\verb=\gre@setstylefont=%tex.tex
	Macro to load greextra as gregoriostylefont at correct size.
	
\verb=\gre@normalstafflinesformat=%tex.tex
	Macro to apply formatting to the normal staff lines.
	
\verb=\gre@additionalstafflinesformat=%tex.tex
	Macro to apply formatting to the additional staff lines.

\verb=\gre@includetexscore#1=%tex.tex
	Macro to include scores in gtex format.

\verb=\gre@includegabcscore#1=%tex.tex
	Macro to include scores in gabc format.

\verb=\gre@usestylecommon=%tex-signs.tex
	Checks to see if gregoriostylefont has been loaded and loads it if it has not.

\verb=\gre@falsepenalty#1=%tex-signs.tex
	Does nothing.

\verb=\gre@truepenalty#1=%tex-signs.tex
	Alias for \verb=\penalty#1=.  Used in combination with \verb=\gre@falsepenalty= to avoid placing \verb=\penalty= where it isn't allowed.

\verb=\gre@hskip=%tex-signs.tex
	Alias for \verb=\hskip= outside of discretionaries and \verb=\kern= inside of them.

\verb=\gre@penalty#1=%tex-signs.tex
	Alias for \verb=\gre@truepenalty= outside of discretionaries and \verb=\gre@falsepenalty= inside of them.

\verb=\gre@insidediscretionary=%tex-signs.tex
	Count for tracking whether we are in a discretionary (1) or not (0).

\verb=\gre@discretionary#1#2=%tex-signs.tex
	Sets \verb=\gre@hskip=, \verb=\gre@penalty=, \verb=\gre@insidediscretionary= to their within discretionary values and then calls \verb=\discretionary#1#2#3=.  \#2 (a string) is a non-printing string which sets \verb=\gre@lastoflinecount= to 2.  After resolving the \verb=\discretionary= command \verb=\gre@hskip=, \verb=\gre@penalty=, and \verb=\gre@insidediscretionary= are restored to their outside values.

\verb=\gre@removeclefcount=%tex-signs.tex
	Count for indicating if the clef should be printed (1) or not (0).

\verb=\gre@clefnum=%tex-signs.tex
	Count indicated the clef line and pitch: 1- c on bottom line, 2 - c on second line, 3 - c on third line, 4 - c on top line, 5 - f on bottom line, 6 - f on second line, 7 - f on third line, 8 - f on top line.

\verb=\gre@clefwidth=%tex-signs.tex
	Width of the clef.

\verb=\gre@setlinesclef#1#2#3#4=%tex-signs.tex
	Macro to define the clef that will appear at the beginning of the lines.
	\#1 --- character: type of clef (c or f)
	\#2 --- character: height of clef (using gabc height letters)
	\#3 --- integer: add a space after the clef (1), or not (0)
	\#4 --- character: height of flat after clef (using gabc height letters, a = no flat)

\verb=\gre@calculateclefnum#1#2#3=%tex-signs.tex
	Macro to calculate the clef number and store it in \verb=\gre@clefnum=.
	\#1 --- character: type of clef (c or f)
	\#2 --- integer: line on which the clef appears (1--4)
	\#3 --- character: height of flat after clef (using gabc height letters, a = no flat)

\verb=\gre@updatelinesclef=%tex-signs.tex
	Macro redrawing a key from \verb=\gre@clefnum=.  Useful for vertical space changes.
	
\verb=\gre@typekey#1#2#3#4#5=%tex-signs.tex
	Macro for typesetting the key signature.
	\#1 --- character: type of clef (c or f)
	\#2 --- integer: line of the key (1 to 4)
	\#3 --- integer: Use small key characters (1), or not (0)
	\#4 --- integer: Print a space after the key (1), or not (0)
	\#5 --- character: Height of flat in key (‘a’ for no flat)

\verb=\gre@inchangeclef#1#2#3=%tex-signs.tex
	Macro for changing key inside a syllable.
	\#1 --- character: type of clef (c or f)
	\#2 --- integer: line on which the clef appears (1--4)
	\#3 --- character: height of flat after clef (using gabc height letters, a = no flat)

\verb=\gre@setcusto#1=%tex-signs.tex
	Macro for typesetting custos.
	\#1 --- character: height of custom

\verb=\gre@additionaltopcusotolineend=%tex-signs.tex
	Macro for typesetting the line behind a high custos at end of line.

\verb=\gre@additionalbottomcusotlineend=%tex-signs.tex
	Macro for typesetting the line behind a low custos at end of line.

\verb=\gre@additionaltopcusotolinemiddle=%tex-signs.tex
	Macro for typesetting the line behind a high custos in middle of line.

\verb=\gre@additionalbottomcusotlinemiddle=%tex-signs.tex
	Macro for typesetting the line behind a low custos in middle of line.

\verb=\gre@custocahar#1=%tex-signs.tex
	Selects the appropriate custos character.
	\#1 --- character: height of the custos

\verb=\gre@curlybracechar=%tex-signs.tex
	Macro for the char ‘\{‘ (which is rotated 90 and placed over the notes)

\verb=\gre@bracechar=%tex-signs.tex
	Macro for the char ‘(‘ (which is rotated 90 and placed over the notes)

\verb=\gre@resizebox=%tex-signs.tex
	Local alias for \verb=\resizebox= (currently borrowed from graphics)

\verb=\gre@lowchoralsignstyle#1=%tex.sty and tex-signs.tex
	Applies the formatting to the low choral signs specified by the appropriate (pseudo-)environment.

\verb=\gre@highchoralsignstyle#1=%tex.sty and tex-signs.tex
	Applies the formatting to the high choral signs specified by the appropriate (pseudo-)environment.

\verb=\Gre@Tempsign=%tex-signs.tex
	Box for temporarily holding a sign needed for calculations.

\verb=\gre@tempdimsignwidth=%tex-signs.tex
	Temporary dimension used to calculate placement of signs.

\verb=\gre@vepisemusorrareaux#1#2#3#4#5#6#7=%tex-signs.tex
	Macro help with typesetting of episemus.
	\#1 --- character: A preceding symbol needed for calculating position of episemus.
	\#2 --- character: Another preceding symbol.
	\#3 --- integer: Indicates how far to move back; 0 = go back to beginning of glyph, 1 = go back width of \#2, 2 = go back width of \#1 and forward width of \#2, 3 = go to beginning of glyph, forward by width of \#1 and back width of \#2
	\#4 --- integer: An additional shift (in sp) to apply to the episemus.
	\#5 --- integer: Use ictus arsicus (1) or ictus theticus (2)
	\#6 --- integer: Set a vertical episemus (1), a rare sign (2), a choral sign (3), or a brace above the bar (4)
	\#7 --- integer: If \#6=3 then the choral sign.

\verb=\gre@vepisemusorrare#1#2#3#4#5=%tex-signs.tex
	Macro for typesetting vertical episemus or rare accent.
	\#1 --- character: height of episemus
	\#2 --- integer: Type of glyph the episemus is attached to.  See episemus special argument for description of options.
	\#3 --- integer: Glyph number of glyph being typeset
	\#4 --- integer: Set a vertical episemus (1), a rare sign (2), a choral sign (3), or a brace above the bar (4)
	\#5 --- string: If \#4=3 then the choral sign.

\verb=\grewriteaux#1=%tex.tex
	Macro for writing to the aux file.
	\#1 --- string: what is written to the aux file.

\verb=\opengreaux=%tex.tex
	Macro for opening the aux file.

\verb=\closegreaux=%tex.tex
	Macro for closing the aux file.

\verb=\gre@ictus#1=%tex-signs.tex
	Macro for writing the position of an ictus to the aux file.
	\#1 --- integer: type of ictus which was typeset, 1 = type a, 2 = type t

\verb=\gre@hepisorlineaux#1#2#3#4=%tex-signs.tex
	Macro for helping to typeset a horizontal episemus and additional line.
	\#1 --- character: a glyph having the same width as the width between the end of the glyph and the beginning of the episemus
	\#2 --- integer: glyph number of the episemus
	\#3 --- integer: go to beginning of glyph (0), or not (else)
	\#4 --- integer: set horizontal episemus (0), horizontal episemus under a note (1), line at top of staff (2), line at bottom of staff (3), choral sign (4)

\verb=\gre@hepisorline#1#2#3#4=%tex-signs.tex
	Macro for typesetting a horizontal line (additional line or episemus).
	\#1 --- character: Height of the episemus
	\#2 --- special: See Episemus special
	\#3 --- integer: The ambitus of the porrectus or porrectus flexes if the second argument is 9, 10, 11, 21, 22, 23; ignored otherwise
	\#4 --- integer: set horizontal episemus (0), horizontal episemus under a note (1), line at top of staff (2), line at bottom of staff (3), choral sign (4)

\verb=\gre@AddHEpisemusBridges=%tex-signs.tex
	Macro to activate bridging horizontal episemus.
	
\verb=\gre@RemoveHEpisemusBridges=%tex-signs.tex
	Macro to suspend bridging horizontal episemus.

\verb=\gre@writebar#1#2#3=%tex-signs.tex
	Macro to typeset a bar line.
	\#1 --- integer: Type of bar line: virgula (0), minima (2), maior (3), finalis (4), last finalis (5), dominican bar 1 (6), dominican bar 2 (7), dominican bar 3 (8), dominican bar 4 (9), dominican bar 5 (10), or dominican bar 6 (11)
	\#2 --- integer: The bar appears within a syllable (1), or not (0)
	\#3 --- code: Macros which may happen before the skip but after the bar (typically \verb=\grevepisemus=)

\verb=\gre@tempdimtwo=
	A temporary dimension used in calculations.

\verb=\gre@divisiomaiorsymbol=%tex-signs.tex
	Macro defining the divisio maior symbol.

\verb=\gre@divisiofinalissymbol=%tex-signs.tex
	Macro defining the divisio finalis symbol.

\verb=\keeprightbox=%tex-signs.tex
	A count to tell if we have to keep the localrightbox until the end.

\verb=\gre@hidepclines=%tex-signs.tex
	An integer indicating whether the lines behind a punctum cavum are hidden (1) or not (0).

\verb=\gre@hidealtlines=%tex-signs.tex
	An integer indicating whether the lines behind an alteration are hidden (1) or not (0).

\verb=\gre@fillhole#1=%tex-signs.tex
	Character to fill a hole in a character with the color grebackgroundcolor.
	\#1 --- character: The shape of the hole.

\verb=\gre@firstisalteration=%tex-signs.tex
	Count to track if the first glyph is an alteration.

\verb=\gre@alteration#1#2#3#4=%tex-signs.tex
	Macro to typeset an alteration (sharp, flat, or natural).
	\#1 --- character: The height of the alteration
	\#2 --- character: The alteration itself.
	\#3 --- character: The character of the hole for the alteration.
	\#4 --- integer: Indicates a flat for a key change (1), or not (0).

\section{Special arguments}

These arguments are used by multiple functions and take a lot of space to describe so we describe them once here and refer to this section rather than have multiple definitions.

\subsection{Episemus special}
Integer with the following possibilities:
    0: last note, which is a standard punctum (works with pes)
    1: same, but the last note is a deminutus
    2: the note before the last note, which is a standard punctum
    3: idem, but the note is the note preceding a deminutus
    4: the note before the note before the last note (for porrectus flexus)
    5: idem, but when the two last notes are a deminutus
    6: the first note, if it is a standard punctum
    7: the first note, if it is an initio debilis
    8: the first note, if it is a porrectus
    % the three next arguments make no sense for a vepisemus
    9: the two first notes, if it is a porrectus
    10: the two first notes, if it is a porrectus flexus
    11: the notes two and three of a torculus resupinus
    12: the last note, if it is a punctum inclinatum
    13: idem, if it is a punctum inclinatum deminutus
    14: idem, if it is a stropha
    15: idem, with a quilisma
    16: idem, with an oriscus
    17: same of 2 but for ambitus of one
    18: same of 0, but the last note is a smaller punctum (concerning simple podatus, podatus, and torculus resupinus)
    19: the first note, if it is an oriscus
    20: the first note, if it is a quilisma
    21: the second note of a torculus resupinus with first ambitus of at least two
    22: idem with ambitus of one
    23: idem with initio debilis
    24: the last note, if it is a linea punctum (or linea punctum cavum)
    25: the last note, if it is a bar
    26: the last note, if it is a virgula
    27: the last note, if it is a divisio finalis


%%% Local Variables:
%%% mode: latex
%%% TeX-master: "Documentation"
%%% End:

\section{Gregorio Controls}

These functions are the ones written by Gregorio to the gtex file.  While one could, in theory, use/change them to alter the appearance of elements of the score, it is far better to make your changes in the gabc file and let Gregorio make the changes to the gtex file.


\verb=\begingregorioscore=%tex.tex
	Macro to start a score.
\verb=\endgregorioscore=%tex.tex
	Macro to end a score.

\verb=\greaccentus#1#2=%tex-signs.tex
	Macro for typesetting an accentus.
	#1 --- character: height of episemus
	#2 --- integer: Type of glyph the episemus is attached to.  See episemus special argument for description of options.
\verb=\greactiveatechironomy=%tex.tex
	Macro called at the beginning of the score to enable chironomy.
\verb=\greadditionalline#1#2#3=%tex-signs.tex
	Macro to typeset the additional line above or below the staff.
	#1 --- special: See Episemus special
	#2 --- integer: The ambitus of the porrectus or porrectus flexes if the first argument is 9, 10, 11, 21, 22, 23; ignored otherwise
	#3 --- integer: set horizontal episemus (0), horizontal episemus under a note (1), line at top of staff (2), line at bottom of staff (3), choral sign (4)
\verb=\greadjustsecondline=%didn't actually find this one in gregoriotex-write.c%tex.tex
	Macro to call before first syllable, but after \verb=\gresetinitialclef=
\verb=\greadjustthirdline=%tex.tex
	Macro to call during the second line.
\verb=\greaugmentumduplex#1#2#3=%tex-signs.tex
	Macro for typesetting an augmentum duplex (a pair of punctum mora)
	#1 --- character: height for first punctum mora
	#2 --- character: height for second punctum mora
	#3 --- integer: First punctum mora occurs before last note of a podatus, prorectus, or toculus resupinus (1), or not (0)

\verb=\grebarbrace#1=%tex-signs.tex
	Macro for typesetting a bar brace.
	#1 --- integer: Type of glyph the episemus is attached to.  See episemus special argument for description of options.
\verb=\grebarsyllable=
\verb=\grebarvepisemus#1=%tex-signs.tex
	Macro to typeset a vertical episemus around a bar.
	#1 --- integer: Type of glyph the episemus is attached to.  See episemus special argument for description of options.
\verb=\grebarvepisemusictusa#1=%tex-signs.tex
	Macro to typeset a vertical episemus and ictus (type a) around a bar.
	#1 --- integer: Type of glyph the episemus is attached to.  See episemus special argument for description of options.
\verb=\grebarvepisemusictust#1=%tex-signs.tex
	Macro to typeset a vertical episemus and ictus (type t) around a bar.
	#1 --- integer: Type of glyph the episemus is attached to.  See episemus special argument for description of options.
\verb=\grebeginnlbarea#1#2=%tex.tex
	Macro called at beginning of a no line break area.
	#1 --- integer: 0 = not in the neumes; 1 = in the neumes
	#2 --- integer: 0 = call didn't come from translation centering; 1 = call came from translation centering.
\verb=\grebeginnotes=%tex.tex
	Macro to draw the staff lines.  Comes after the initial but before the clef.
\verb=\greboldfont#1=%tex.sty, PlainTeX version in tex.tex
	Makes argument (a string) bold.  Accesses \LaTeX \verb=\textbf= or PlainTeX \verb=\bf= as appropriate.  Corresponds to ``<b></b>'' tags in gabc.

\verb=\grechangeclef#1#2#3#4=%tex-signs.tex
	Macro called when key changes
	#1 --- character: type of new clef (c or f)
	#2 --- integer: line of new clef
	#3 --- integer: Print space before clef (1), or not (0)
	#4 --- character: Height of flat in key (‘a’ for no flat)
\verb=\grecirculus#1#2=%tex-signs.tex
	Macro for typesetting a circulus.
	#1 --- character: height of circulus
	#2 --- integer: Type of glyph the circulus is attached to.  See episemus special argument for description of options.
\verb=\grecolored#1=%tex.sty, PlainTeX version in tex.tex
	Colors argument (a string) in \verb=gregoriocolor.=  Corresponds to ``<c></c>'' tags in gabc.  Does nothing in PlainTeX
\verb=\grecusto#1=
	Typesets a custo.
	#1 --- character: Height of custo

\verb=\gredagger=
\verb=\grediscretionary{}=
\verb=\gredivisiofinalis#1=%tex-signs.tex
	Macro to typeset a divisio finalis.
	#1 --- code: Macros which may happen before the skip but after the divisio finalis (typically \verb=\grevepisemus=)
\verb=\gredivisiomaior#1=%tex-signs.tex
	Macro to typeset a divisio maior.
	#1 --- code: Macros which may happen before the skip but after the divisio maior (typically \verb=\grevepisemus=)
\verb=\gredivisiominima#1=%tex-signs.tex
	Macro to typeset a divisio minima.
	#1 --- code: Macros which may happen before the skip but after the divisio minima (typically \verb=\grevepisemus=)
\verb=\gredivisiominor#1=%tex-signs.tex
	Macro to typeset a divisio minor.
	#1 --- code: Macros which may happen before the skip but after the divisio minor (typically \verb=\grevepisemus=)
\verb=\gredominica#1#2=%tex-signs.tex
	Macro to typeset a dominican bar.
	#1 --- integer: Type of dominican bar. 6--11 corresponds to types 1--6
	#2 --- code: Macros which may happen before the skip but after the divisio minor (typically \verb=\grevepisemus=)

\verb=\greendnlbarea#1#2=%tex.tex
	Macro called at beginning of a no line break area.
	#1 --- integer: 0 = not in the neumes; 1 = in the neumes
	#2 --- integer: 0 = call didn't come from translation centering; 1 = call came from translation centering.
\verb=\greendofelement#1#2=%tex.tex
	Macro to end elements.
	#1 --- integer: 0 = default space; 1 = larger space; 2 = glyph space; 3 = zero-width space
	#2 --- integer: 0 = space is breakable; 1 = space is unbreakable
\verb=\greendofglyph#1=%tex.tex
	Macro to end a glyph without ending the element.
	#1 --- integer: 0 = default space; 1 = zero-width space; 2 = space between flat or natural and a note; 3 = space between two puncta inclinata; 4 = space between bivirga or trivirga; 5 = space between bistropha or tristropha; 6 = space after a punctum mora XXX: not used yet, not so sure it is a good idea...; 7 = space between a punctum inclinatum and a punctum inclinatum debilis; 8 = space between two puncta inclinata debilis; 9 = space before a punctum (or something else) and a punctum inclinatum; 10 = space between puncta inclinata (also debilis for now), larger ambitus (range=3rd).; 11 = space between puncta inclinata (also debilis for now), larger ambitus (range=4th or more)

\verb=\grefinaldivisiofinalis#1=%tex-signs.tex
	Macro to end a line with a divisio finalis.
	#1 --- integer: There is something which needs to be placed after the divisio finalis (1), or not (0).
\verb=\grefinaldivisiomaior#1=%tex-signs.tex
	Macro to end a line with a divisio maior.
	#1 --- integer: There is something which needs to be placed after the divisio maior (1), or not (0).
\verb=\grefirstlinebottomspace#1#2=%tex.tex
	Macro for additional bottom space for the first line
	#1 --- integer: 0 = no note below staff; 1 = note below 1st line (position c); 2 = note on 0th line (position b); 3 = note below 0th line (position a); 4 = note below 0th line (position a) with vertical episemus
	#2 --- integer: 0 = no translation; 1 = translation present
\verb=\greflat#1#2=%tex-signs.tex
	Macro to typeset a flat.
	#1 --- character: Height of the flat.
	#2 --- integer: Indicates the a flat for a key change (1), or not (0)

\verb=\greglyph=
\verb=\gregorianmode#1=%tex.tex
	If the gabc file contains a mode in the header, then this function places said mode as the first (top) annotation.  This function effectively disables \verb=\setfirstlineaboveinitial=.  #1 is an integer from 1 to 8.  Other values are ignored (and \verb=\setfirstlineaboveinitial= should still work).
	Bug: This macro needs to appear before \verb=\greinitial= in the gtex file but \verb=gregoriotex-write.c= places it after.
\verb=\gregorioapiverstion#1=%tex.tex
	Checks to see if GregorioTeX API is version specified by argument (and therefore compatible with the score.  #1 is date in format: yyyymmdd

\verb=\grehepisemus#1#2#3#4=%tex-signs.tex
	Macro to typeset a episemus.
	#1 --- character: Height of the episemus
	#2 --- special: See Episemus special
	#3 --- integer: The ambitus of the porrectus or porrectus flexes if the second argument is 9, 10, 11, 21, 22, 23; ignored otherwise
	#4 --- character: replacement for #1 if a bridge causes a height substitution	
\verb=\grehepisemusbottom#1#2#3=%tex-signs.tex
	Macro to typeset a episemus at the bottom of a note
	#1 --- character: Height of the episemus
	#2 --- special: See Episemus special
	#3 --- integer: The ambitus of the porrectus or porrectus flexes if the second argument is 9, 10, 11, 21, 22, 23; ignored otherwise
\verb=\grehepisemusbridge#1#2#3=%tex-signs.tex
	Macro to typeset a bridge episemus father the last note of a glyph (element, syllable) if the next episemus is at the same height.
	#1 --- character: Height of the episemus
	#2 --- special: See Episemus special
	#3 --- integer: The ambitus of the porrectus or porrectus flexes if the second argument is 9, 10, 11, 21, 22, 23; ignored otherwise
\verb=\grehighchoralsign#1#2#3=%tex-signs.tex
	Macro for typesetting high choral signs.
	#1 --- character: height of the sign
	#2 --- string: the choral sign
	#3 --- integer: choral sign occurs before last note of podatus, porrectus, or torculus resupinus (1), or not (0)
\verb=\grehyph=%tex.tex
	Macro used for end of line hyphens.  Defaults to =\grenormalhyph=.

\verb=\greictusa#1=%tex-signs.tex
	Macro for typesetting an ictus (type a)
	#1 --- integer: Type of glyph the ictus is attached to.  See episemus special argument for description of options.
\verb=\greictust#1=%tex-signs.tex
	Macro for typesetting an ictus (type t)
	#1 --- integer: Type of glyph the ictus is attached to.  See episemus special argument for description of options.
	
\verb=\grein=
\verb=\greindivisiofinalis#1=%tex-signs.tex
	Macro to typeset a divisio finalis inside a syllable.
	#1 --- code: Macros which may happen before the skip but after the divisio finalis (typically \verb=\grevepisemus=)
\verb=\greindivisiomaior#1=%tex-signs.tex
	Macro to typeset a divisio maior inside a syllable.
	#1 --- code: Macros which may happen before the skip but after the divisio maior (typically \verb=\grevepisemus=)
\verb=\greindivisiominima#1=%tex-signs.tex
	Macro to typeset a divisio minima inside a syllable.
	#1 --- code: Macros which may happen before the skip but after the divisio minima (typically \verb=\grevepisemus=)
\verb=\greindivisiominor#1=%tex-signs.tex
	Macro to typeset a divisio minor inside a syllable.
	#1 --- code: Macros which may happen before the skip but after the divisio minor (typically \verb=\grevepisemus=)
\verb=\greindominica#1#2=%tex-signs.tex
	Macro to typeset a dominican bar inside a syllable.
	#1 --- integer: Type of dominican bar. 6--11 corresponds to types 1--6
	#2 --- code: Macros which may happen before the skip but after the divisio minor (typically \verb=\grevepisemus=)
\verb=\greinitial#1=%tex.tex
	Macro to set the initial (#1, a character) in the score.
\verb=\greinsertchiroline=%tex.tex
	Macro called at the beginning of a line to insert the chironomic signs.
\verb=\greinvirgula#1=%tex-signs.tex
	Macro to typeset a virgula inside a syllable.
	#1 --- code: Macros which may happen before the skip but after the virgula (typically \verb=\grevepisemus=)
\verb=\greitalic#1=%tex.sty, PlainTeX version in tex.tex
	Makes argument (a string) italic.  Accesses \LaTeX \verb=\textit= or PlainTeX \verb=\it= as appropriate.  Corresponds to ``<i></i>'' tags in gabc.

\verb=\grelastofline=%tex.tex
	Macro to set \verb=\gre@lastoflinecount= to 1 (i.e. mark that this syllable is the last of the line).
\verb=\grelastofscore=%tex.tex
	Macro to mark the syllable as the last of the score.
\verb=\grelinea#1#2#3=%tex-signs.tex
	Macho for typesetting a linea.
	#1 --- : %argument 2 from greglyph
	#2 --- : %argument 3 from greglyph
	#3 --- : %argument 4 from greglyph
\verb=\grelineapunctumcavum=
	Macro to typeset a linea punctum cavum.
	#1 --- %argument 2 from greglyph
	#2 --- %argument 3 from greglyph
	#3 --- %argument 4 from greglyph
	#4 --- code: Code executed before the punctum cavum is written.
	#5 --- %argument 5 from greglyph
\verb=\grelowchoralsign#1#2#3=%tex-signs.tex
	Macro for typesetting low choral signs.
	#1 --- character: height of the sign
	#2 --- string: the choral sign
	#3 --- integer: choral sign occurs before last note of podatus, porrectus, or torculus resupinus (1), or not (0)

\verb=\grenatural#1#2=%tex-signs.tex
	Macro to typeset a natural.
	#1 --- character: Height of the flat.
	#2 --- integer: Indicates the a flat for a key change (1), or not (0)
\verb=\grenewline=%tex.tex
	Macro to call if you want to go to the next line simply.
\verb=\grenewlinewithspace#1#2#3#4=%tex.tex
	Macro called to go to the next line but when there are additional vertical spaces to add
	#1 --- integer: 0 = no note above staff; 1 = note above 4th line (position k); 2 = note on 5th line (position l); 3 = note above 5th line (position m)
	#2 --- integer: 0 = no note below staff; 1 note below 1st line (position c); 2 = note on 0th line (position b); 3 = note below 0th line (position a); 4 = note below 0th line (position a) with vertical episemus
	#3 --- integer: 0 = no translation; 1 = translation present
	#4 --- integer: 0 = no extra space above staff; 1 = extra space above staff
\verb=\grenewparline=%tex.tex
	Same as \verb=\grenewline= except line is not justified.
\verb=\grenewparlinewithspace#1#2#3#4=%tex.tex
	Same as \verb=\grenewlinewithspace= except line is not justified.
\verb=\grenormalhyph=%tex.tex, not actually found in gregoriotex-write.c
	Macro to typeset a normal hyphen.
\verb=\grenoinitial=%tex.tex
	Macro called when no initial is being set.
\verb=\grenormalinitial=%not actually found in gregoriotex-write.c%tex.tex
	Macro to cancel a 2-line initial.

\verb=\greoverbrace#1#2#3#4=%tex-signs.tex
	Macro to typeset the curved brace over notes.
	#1 --- dimension: the width
	#2 --- dimension: the vertical shift
	#3 --- dimension: the horizontal shift
	#4 --- integer: Shift to the beginning of the last glyph (1), or not (0)
\verb=\greovercurlybrace#1#2#3#4#5=%tex-signs.tex
	Macro to typeset the curly brace over notes.
	#1 --- dimension: the width
	#2 --- integer: Put an accent above (1), or not (0)
	#3 --- dimension: the vertical shift
	#4 --- dimension: the horizontal shift
	#5 --- integer: Shift to the beginning of the last glyph (1), or not (0)

\verb=\grepunctumcavum#1#2#3#4#5=%tex-signs.tex
	Macro to typeset a punctum cavum.
	#1 --- %argument 2 from greglyph
	#2 --- %argument 3 from greglyph
	#3 --- %argument 4 from greglyph
	#4 --- code: Code executed before the punctum cavum is written.
	#5 --- %argument 5 from greglyph
\verb=\grepunctummora#1#2#3#4=%tex-signs.tex
	Macro for typesetting punctum mora.
	#1 --- character: height of punctum mora
	#2 --- integer: Go back to end of punctum (1), shift left width of 1 punctum (2), or shift left width of 1 punctum and ambitus of 1 (3)
	#3 --- integer: Punctum mora occurs before last note of podatus, porrectus, or torculus resupinus (1), or not (0)
	#4 --- integer: Punctum inclinatum (1), or not (0)

\verb=\grereversedaccentus#1#2=%tex-signs.tex
	Macro for typesetting a reversed accentus.
	#1 --- character: height of accentus
	#2 --- integer: Type of glyph the accentus is attached to.  See episemus special argument for description of options.
\verb=\grereversedsemicirculus#1#2=%tex-signs.tex
	Macro for typesetting a reversed semicirculus.
	#1 --- character: height of semicirculus
	#2 --- integer: Type of glyph the semicirculus is attached to.  See episemus special argument for description of options.

\verb=\grescorereference=%tex.tex
	Currently does nothing.
\verb=\gresemicirculus#1#2=%tex-signs.tex
	Macro for typesetting a semicirculus.
	#1 --- character: height of semicirculus
	#2 --- integer: Type of glyph the semicirculus is attached to.  See episemus special argument for description of options.
\verb=\gresetbiginitial=%tex.tex
	Macro which indicates that a 2-line initial is desired.
\verb=\gresetfixednexttextformat=
\verb=\gresetfixedtextformat=
\verb=\gresetinitialclef#1#2#3=%tex-signs.tex
	Macro for writing initial key.
	#1 --- character: type of clef (c or f)
	#2 --- integer: line of key (1 to 4)
	#3 --- character: Height of flat in key (‘a’ for no flat)
\verb=\gresetlinesclef=%#1#2#3#4=%tex.tex
%	possible values for <character>: c, f
%	possible values for <integer2>: 0, 1 (1 = space before clef, 0 = no space)
\verb=\gresettextabovelines#1=
	Macro to place argument (a string) above the lines and empty \verb=\gre@currenttextabovelines= when done.
\verb=\gresharp#1#2=%tex-signs.tex
	Macro to typeset a sharp.
	#1 --- character: Height of the flat.
	#2 --- integer: Indicates the a flat for a key change (1), or not (0)
\verb=\gresmallcaps#1=%tex.sty, PlainTeX version in tex.tex
	Makes argument (a string) small capitals.  Accesses \LaTeX \verb=\textsc= or PlainTeX \verb=\sc= as appropriate  Corresponds to ``<sc></sc>'' tags in gabc.
\verb=\grestar=
\verb=\gresyllable=

\verb=\gretilde=%tex.tex
	Macro to print $\sim$.
\verb=\gretranslationcenterend=%tex.tex
	Macro to set \verb=\gre@mustdotranslationcenerend= to 1.
\verb=\grett#1=%tex.sty, PlainTeX version in tex.tex
	Makes argument (a string) typewriter font.  Accesses \LaTeX \verb=\texttt= or PlainTeX \verb=\tt= as appropriate.

\verb=\greul#1=%tex.sty, PlainTeX version in tex.tex
	Makes argument (a string) underlined under \LaTeX using \verb=\underline=.  Does nothing in PlainTeX

\verb=\grevepisemus#1#2=%tex-signs.tex
	Macro for typesetting the vertical episemus.
	#1 --- character: height of episemus
	#2 --- integer: Type of glyph the episemus is attached to.  See episemus special argument for description of options.
\verb=\grevepisemusictusa#1#2=%tex-signs.tex
	Macro for typesetting the vertical episemus and ictus (type a) on the same glyph.
	#1 --- character: height of episemus
	#2 --- integer: Type of glyph the episemus and ictus are attached to.  See episemus special argument for description of options.
\verb=\grevepisemusictust#1#2=%tex-signs.tex
	Macro for typesetting the vertical episemus and ictus (type t) on the same glyph.
	#1 --- character: height of episemus
	#2 --- integer: Type of glyph the episemus and ictus are attached to.  See episemus special argument for description of options.
\verb=\grevirgula#1=%tex-signs.tex
	Macro to typeset a virgula.
	#1 --- code: Macros which may happen before the skip but after the virgula (typically \verb=\grevepisemus=)

\verb=\grewritetranslation#1=%tex.tex
	Macro to typeset argument (a string) in the translation position.
\verb=\grewritetranslationwithcenterbeginning#1=%tex.tex
	Macro to typeset argument (a string) in the translation position (at the beginning of a line?).

\verb=\grezerhyph=%tex.tex
	Macro to typeset a zero-width hyphen (the hyphen is visible, it is treated as if it had 0 width though.  Used for fine tuning spacing (especially at line endings).

\verb=\setgregoriofont#1=%tex.tex
	Macro to set the font used for the glyphs.
	#1 --- string: gregorio; parmesan; greciliae; gregoria


\addcontentsline{toc}{section}{Index}
\printindex

\end{document}
