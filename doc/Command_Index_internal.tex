% !TEX root = UserManual.tex
\section{Gregoriotex Controls}

These functions are the ones used by Gregoriotex internally as it
process the commands listed above.  They should not appear in any user
document and are listed here for programmer documentation purposes
only.

\macroname{\textbackslash gre@calculate@constantglyphraise}{gregoriotex-spaces.tex}
%\verb=\gre@calculateconstantglyphraise=%tex.tex
Macro to caluclate \verb=\gre@constantglyphraise=

\macroname{\textbackslash gre@addtranslationspace}{gregoriotex-spaces.tex}
%\verb=\gre@addtranslationspace=%tex.tex
Macro to tell Gregorio to set space for the translation.

\macroname{\textbackslash gre@removetranslationspace}{gregoriotexspaces.tex}
%\verb=\gre@removetranslationspace=%tex.tex
Macro to tell Gregorio to remove the space allocated to the translation.


\macroname{\textbackslash gre@calculate@additionalspaces\#1\#2}{gregoriotex-spaces.tex}
%\verb=\gre@updateadditionalspaces#1#2=%tex.tex
Macro which calculates \verb=\gre@additionalbottomspace= and\\
\verb=\gre@additionaltopspace=

\begin{argtable}
  \#1 & 0 & no note above staff\\
  & 1 & note above 4th line (position k)\\
  & 2 & note on 5th line (position l)\\
  & 3 & note above 5th line (position m)\\
  \#2 & 0 & no note below staff\\
  & 1 & note below 1st line (position c)\\
  & 2 & note on 0th line (position b)\\
  & 3 & note below 0th line (position a)\\
  & 4 & note below 0th line (position a) with vertical episemus
 \end{argtable}

\macroname{\textbackslash gre@calculate@textaligncenter\#1\#2\#3}{gregoriotex-spaces.tex}
%\verb=\gre@findtextaligncenter#1#2#3=%tex.tex
Macro for calculating \verb=\gre@textaligncenter=.

\begin{argtable}
  \#1 & string & The first part of the syllable (any preceding consonants in Latin).\\
  \#2 & string & The middle part of the syllable (the vowel in Latin, the whole syllable in English).\\
  \#3 & 0 & Calculation is being performed for the current syllable.\\
      & 1 & Calculation is being performed for the next syllable.\\
\end{argtable}

\macroname{\textbackslash gre@calculate@aboveinitialraise}{gregoriotex.tex}
%\verb=\gre@setaboveinitialrais=%tex.tex
Macro to give \verb=\gre@aboveinitialfirstraise= and\\
\verb=\gre@aboveinitialsecondraise= their working values.
	
\macroname{\textbackslash gre@englishcentering}{gregoriotex-syllable.tex}
Count to track if we are using english centering scheme (1), or not (0).

\macroname{\textbackslash gre@calculate@textlower}{gregoriotex-spaces.tex}
Calculates the default value of \texttt{textlower}.  Default is \texttt{spacebeneathtext}.

\macroname{\textbackslash gre@calculate@linewidth}{gregoriotex-spaces.tex}
Calculates the default line width.  Default is the width of the printable space (\verb=\hsize=).

\macroname{\textbackslash gre@calculate@stafflinewidth}{gregoriotex-spaces.tex}
Calculates the default width of the staff lines.  Default is \texttt{linewidth}.

\macroname{\textbackslash gre@calculate@stafflineheight}{gregoriotex-spaces.tex}
Calculates the height (thickness) of the staff lines.  Dependent on \texttt{stafflineheightfactor} and \texttt{grefactor}.

\macroname{\textbackslash gre@calculate@interstafflinespace}{gregoriotex-spaces.tex}
Calculates the distance between the staff lines.  Dependent on \texttt{stafflineheight} and \texttt{grefactor}

\macroname{\textbackslash gre@calculate@stafflinediff}{gregoriotex-spaces.tex}
Calculates a correction factor for when the staff lines are not their default thickness.  Dependent on \texttt{stafflineheight} and \texttt{grefactor}.

\macroname{\textbackslash gre@calculate@staffheight}{gregoriotex-spaces.tex}
Calculates the total height of the staff.  Dependent on \texttt{stafflineheight} and \texttt{interstafflinespace}.

\macroname{\textbackslash gre@calculate@constantglyphraise}{gregoriotex-spaces.tex}
Calculates the baseline correction for the glyphs.  Dependent on \texttt{grefactor}, \texttt{additionalbottomspace}, \texttt{spacebeneathtext}, \texttt{spacelinestext}, \texttt{interstafflinespace}, \texttt{stafflineheight}, \texttt{currenttranslationheight}, and \texttt{stafflinediff}.

\macroname{\textbackslash gre@computespaces}{gregoriotex-spaces.tex}
Aggregates all of the global distance calculations and calls them in the order needed to respect dependencies.

\macroname{\textbackslash gre@calculate@glyphraisevalue\#1\#2}{gregoriotex-spaces.tex}
Calculates the raise values for a glyph (glyphraisevalue and addedraisevalue) based on where it is to be placed and what kind of a glyph it is.  This is a time of use calculation.

\begin{argtable}
  \#1 & integer & The number for where the glyph is located.  \texttt{a} in gabc is 1, \texttt{b} is 2, \etc\\
  \#2 & 0 & no modification\\
  & 1 & puts the value on the interline just above if it is on a line\\
  & 2 & puts the value on the interline just beneath if it is on a line\\
  & 3 & case of the vertical episemus, which is not placed at the same place if the corresponding note is on a line or not\\
  & 4 & case of the punctum mora, for the same reason\\
  & 5 & case of the horizontal episemus under a note, that must be placed a bit lower if the note is on a line\\
  & 6 & case of the signs above (accentus, \etc)\\
  & 8 & case of the punctum mora of the first note of a podatus or the 2nd note of a porrectus, \etc\\
  & 9 & case of the horizontal episemus, that must be placed a bit lower if the note is on a line\\
  & 10 & case of the low choral sign\\
  & 11 & case of the high choral sign\\
  & 12 & case of the low choral sign which is lower than the note\\
  & 13 & case of the brace above the bars
\end{argtable}

\macroname{\textbackslash gre@stafflinefactor}{gregoriotex-spaces.tex}
A number indicating the thickness of the staff lines.

\macroname{\textbackslash gre@scale@stafflinefactor}{gregoriotex-spaces.tex}
Flag indicating whether the stafflinefactor should scale with changes of grefactor (1), or not (0).

\macroname{\textbackslash gre@calculate@additionalspaces\#1\#2}{gregoriotex-spaces.tex}
Calculates the extra spaces above and below the staff lines.  This is a time of use calculation.

\begin{argtable}
  \#1 & 0 & no notes above the staff\\
  & 1 & there is a note above the 4th line\\
  & 2 & there is a note on the 5th line\\
  & 3 & there is a note above the 5th line\\
  \#2 & 0 & no notes below the staff\\
  & 1 & there is a note below the 1st line\\
  & 2 & there is a note on the 0th line\\
  & 3 & there is a note below the 0th line\\
  & 4 & there is a note below the 0th line with a vertical episemus attached
\end{argtable}

\macroname{\textbackslash gre@calculate@textaligncenter\#1\#2\#3}{gregoriotex-spaces.tex}
Macro to calculate the distance from the beginning of the text of a syllable to its alignment point (the center of the vowel for Latin centering, the center of the syllable for English centering).  This is a time of use calculation.

\begin{argtable}
  \#1 & string & the first part of the syllable\\
  \#2 & string & the middle part of the syllable\\
  \#3 & 0 & perform this calculation for the current syllable\\
  & 1 & perform this calculation for the next syllable
\end{argtable}

\macroname{\textbackslash gre@calculate@enddifference\#1\#2\#3\#4\#5}{gregoriotex-spaces.tex}
Calculates the difference between the end of the notes and the end of the syllable text.  Also stores the value for the previous syllable if needed.  This is a time of use calculation.

\begin{argtable}
  \#1 & length & the total width of the notes\\
  \#2 & length & the total width of the syllable text\\
  \#3 & length & the alignment distance for the text (\texttt{textaligncenter})\\
  \#4 & length & the alignment distance for the notes (\texttt{notesaligncenter})\\
  \#5 & 0 & do not save the value for the previous syllable before calculating the new value\\
  & 1 & save the value for the previous syllable before calculating the new value
\end{argtable}

\macroname{\textbackslash gre@changeonedimenfactor\#1\#2\#3}{gregoriotex-spaces.tex}
Change the scale of a single distance from one factor to another.

\begin{argtable}
  \#1 & string & name of the distance to be scaled.  See \nameref{distances}.\\
  \#2 & integer & the factor the distance is currently in\\
  \#3 & integer & the factor the distance is to be put into\\
\end{argtable}

\macroname{\textbackslash gre@changedimenfactor\#1\#2}{gregoriotex-spaces.tex}
Rescales all the distances (and stafflinefactor) which are supposed to scale with a change in factor.

\begin{argtable}
  \#1 & integer & the factor the distance is currently in\\
  \#2 & integer & the factor the distance is to be put into\\
\end{argtable}

\macroname{\textbackslash gre@calculate@nextbegindifference\#1\#2\#3\#4}{gregoriotex-spaces.tex}
Macro to calculate \texttt{nextbegindifference}.

\begin{argtable}
  \#1 & string & the first letters of the next syllable\\
  \#2 & string & the middle letters of the next syllable (the vowel in Latin, the whole syllable in English)\\
  \#3 & string & the end letters of the next syllable\\
  \#4 & $0 <$ integer $< 19$ & the type of notes alignment.  See \nameref{notesalign}.\\
  & $20 <$ integer $< 39$ & Same as below 20 except there is a flat before the notes.  Subtract 20 to get the type of notes alignment.\\
  & $40 <$ integer $< 59$ & Same as below 20 except there is a natural before the notes.  Subtract 40 to get the type of notes alignment.
\end{argtable}

\macroname{\textbackslash gre@tempdimcount}{gregoriotex-spaces.tex}
Temporary count used in calculations.

\macroname{\textbackslash ifchecklength}{gregoriotex-spaces.tex}
Boolean flag used in \verb=\gresetdim= to indicate if we are attempting to set a rubber length.

\macroname{\textbackslash ifbadlength}{gregoriotex-spaces.tex}
Boolean flag used in \verb=\gresetdim= to indicate that we are attempting to assign a rubber length to a distance which cannot accept a rubber value.

\macroname{\textbackslash ifrubber}{gregoriotex-spaces.tex}
Boolean flag used in \verb=\gre@changeonedimenfactor= to indicate if we are dealing with one of the distances which can accept a rubber length.

\macroname{\textbackslash gre@makein\#1}{gregoriotex-spaces.tex}
Strips the decimals and units from a distance.

\begin{argtable}
  \#1 & distance & should be in the form ``[0-9]+.[0-9]+pt''. (\ie the result of applying \verb=\the= to a distance register)
\end{argtable}

\macroname{\textbackslash gre@makenum\#1}{gregoriotex-spaces.tex}
Strips the units from a distance. 

\begin{argtable}
  \#1 & distance & should be in the form ``[0-9]+.[0-9]+pt''. (\ie the result of applying \verb=\the= to a distance register)
\end{argtable}

\macroname{\textbackslash gre@unitfactor}{gregoriotex-spaces.tex}
Temporary count used by \verb=\gre@convertto=.

\macroname{\textbackslash gre@basefactor}{gregoriotex-spaces.tex}
Temporary count used by \verb=\gre@convertto=.

\macroname{\textbackslash gre@temp@count@one}{gregoriotex-spaces.tex}
Temporary count used by \verb=\gre@convertto=.

\macroname{\textbackslash gre@convertto\#1\#2}{gregoriotex-spaces.tex}
Macro which converts a distance into a particular set of units.  Result is placed in \verb=\gre@converted= as a string.

\begin{argtable}
  \#1 & string & two letter abbreviation for the units.  Should recognize all legal TeX units.\\
  \#2 & distance & Distance to be converted.
\end{argtable}

\macroname{\textbackslash gre@converted}{gregoriotex-spaces.tex}
Macro holding result of last call to \verb=\gre@convertto=.

\macroname{\textbackslash gre@consistentunits\#1\#2}{gregoriotex-spaces.tex}
This function takes a distance and formats it as a string so that its units conform to the pattern set by a string representation of a distance.  Result is placed in \verb=\gre@stringdist=.

\begin{argtable}
  \#1 & string & the standard whose format is to be matched.\\
  \#2 & distance & the distance to be adjusted.
\end{argtable}

\macroname{\textbackslash gre@stringdist}{gregoriotex-spaces.tex}
Macro holding result of last call to \verb=\gre@consistentunits=.

\macroname{\textbackslash gre@temp@count@two}{gregoriotex-spaces.tex}
Temporary count used by \verb=\gre@changedimenfactor=.

\macroname{\textbackslash gre@includescore\#1}{gregoriotex.tex}
Macro that handles \verb=\includescore= calls when they do not have an
optional argument.

\begin{argtable}
  \#1 & string & Relative or absolute path to the score.\\
\end{argtable}

\macroname{\textbackslash gre@includescorewithoption[\#1]\#2}{gregoriotex.tex}
Macro that handles \verb=\includescore= calls when they have an optional
argument.

\begin{argtable}
  \#1 & \texttt{n} & Optional. \#2 will be included as is. \\
      & \texttt{a} & Optional. Gregoriotex will automatically compile gabc files if necessary.\\
      & \texttt{f} & Optional. Forces gregoriotex to compile the gabc file.\\
  \#2 & string & Relative or absolute path to the score.\\
\end{argtable}


\subsection{Distances}
All of the distances listed in \nameref{distances} have two internals
associated with them.  These take the form of \verb=\gre@*= and
\verb=\gre@scale@*=.  The first stores the value of the distance (in
string representation).  The second is a numerical flag which
indicates if the distance should scale when the staff size changes (1)
or not (0).

These additional distances are calculated by Gregorio based on the values for the user customizable distances and what may be going on in the score at the time of their use.

\macroname{\textbackslash gre@clefwidth}{gregoriotex-spaces.tex}
%\verb=\gre@clefwidth=%tex-signs.tex
Width of the clef.

\macroname{\textbackslash gre@tempdimsignwidth}{gregoriotex-spaces.tex}
%\verb=\gre@tempdimsignwidth=%tex-signs.tex
Temporary dimension used to calculate placement of signs.

\macroname{gre@tempdimtwo}{gregoriotex-spaces.tex}
%\verb=\gre@tempdimtwo=%tex-signs.tex
A temporary dimension used in calculations.

\macroname{\textbackslash gre@constantglyphraise}{gregoriotex-spaces.tex}
%\verb=\gre@constantglyphraise=%tex.tex
Dimension representing the space between the 0 of the gregorian fonts and the effective 0 of the TeX score.

\macroname{\textbackslash gre@currenttranslationheight}{gregoriotex-spaces.tex}
%\verb=\gre@currenttranslationheight=%tex.tex
Dimension representing the space for the translation beneath the text.

\macroname{\textbackslash gre@stafflinewidth}{gregoriotex-spaces.tex}
%\verb=\gre@stafflinewidth=%tex.tex
Dimension representing the width of a line of staff.  Can vary, for
example, at the first line.

\macroname{\textbackslash gre@linewidth}{gregoriotex-spaces.tex}
%\verb=\gre@linewidth=%tex.tex
Dimension representing the width of the score (including initial).

\macroname{\textbackslash gre@additionalbottomspace}{gregoriotex-spaces.tex}
%\verb=\gre@additionalbottomspace=%tex.tex
Dimension representing extra space below the staff needed for low notes.

\macroname{\textbackslash gre@additionaltopspace}{gregoriotex-spaces.tex}
%\verb=\gre@additionaltopspace=%tex.tex
Dimension representing extra space above the staff needed for high notes.

\macroname{\textbackslash gre@textlower}{gregoriotex-spaces.tex}
%\verb=\gre@textlower=%tex.tex
Dimension representing the height of the separation between the 0th
line (which is invisible except for notes in the a or b position) and
the bottom of the text.

\macroname{\textbackslash gre@tempwidth}{gregoriotex-spaces.tex}
%\verb=\gre@tempwidth=%tex.tex
Dimension representing width of some element.

\macroname{\textbackslash gre@textaligncenter}{gregoriotex-spaces.tex}
%\verb=\gre@textaligncenter=%tex.tex
Dimension representing the width from the beginning of the letters in
a syllable to the middle of the middle letters.  Used for lining up
neumes and syllables.
	
\macroname{\textbackslash gre@additionalleftspace}{gregoriotex-spaces.tex}
%\verb=\gre@additionalleftspace=%tex.tex
Dimension representing the additional space that has to be added to
the localleftbox for a big initial (one taking two lines).

\macroname{\textbackslash gre@initialwidth}{gregoriotex-spaces.tex}
%\verb=\gre@initialwidth=%tex.tex
Dimension representing the width of the initial (and the space after).

\macroname{\textbackslash gre@aboveinitialfirstraise}{gregoriotex-spaces.tex}
%\verb=\gre@aboveinitialfirstraise=%tex.tex
Dimension representing the space allocated to the first annotation.

\macroname{\textbackslash gre@aboveinitialsecondraise}{gregoriotex-spaces.tex}
%\verb=\gre@aboveinitialsecondraise=%tex.tex
Dimension representing the space allocated to the second annotation.

\macroname{\textbackslash gre@currentabovelinestextheight}{gregoriotex-spaces.tex}
%\verb=\gre@currentabovelinestextheight=%tex.tex
Dimension representing the space allocated above the lines for text.	

\macroname{\textbackslash gre@staffheight}{gregoriotex-spaces.tex}
The total height of the staff including the width of the lines and the spaces between them.

\macroname{\textbackslash gre@stafflinediff}{gregoriotex-spaces.tex}
Distance representing the difference between the actual size of the staff lines and the ``standard'' size.

\macroname{\textbackslash gre@stafflineheight}{gregoriotex-spaces.tex}
The height of the staff line.

\macroname{\textbackslash gre@interstafflinespace}{gregoriotex-spaces.tex}
The space between the lines.

\macroname{\textbackslash gre@glyphraisevalue}{gregoriotex-spaces.tex}
The value that a particular glyph must be raised to be set in the correct position.

\macroname{\textbackslash gre@addedraisevalue}{gregoriotex-spaces.tex}
The additional raise needed for the vertical episema and the puncta.

\macroname{\textbackslash gre@enddifference}{gregoriotex-spaces.tex}
Distance from the end of the notes to the end of the text for the previous syllable.  Positive values when notes go further than text, negative in the other case.

\macroname{\textbackslash gre@previousenddifference}{gregoriotex-spaces.tex}
Stored value of enddifference prior to the current one.

\macroname{\textbackslash gre@nextbegindifference}{gregoriotex-spaces.tex}
The difference between the start of the notes and the start of the text for the next syllable.  Positive when when text begins first, negative in other case.

\macroname{\textbackslash gre@begindifference}{gregoriotex-spaces.tex}
The difference between the start of the notes and the start of the text for the current syllable.  Positive when when text begins first, negative in other case.

\macroname{\textbackslash gre@lastglyphwidth}{gregoriotex-spaces.tex}
The width of the last glyph.

\macroname{\textbackslash gre@notesaligncenter}{gregoriotex-spaces.tex}
Distance from beginning of notes to their point of alignment.

\macroname{\textbackslash gre@tempdim}{gregoriotex-spaces.tex}
Temporary dimension used in calculations.

\macroname{\textbackslash gre@tempdimskip}{gregoriotex-spaces.tex}
Temporary skip used in calculations.

\macroname{\textbackslash gre@skipone}{gregoriotex-spaces.tex}
Temporary skip used in calculations.

\macroname{\textbackslash gre@temp}{gregoriotex-spaces.tex}
Temporary skip used in calculations.

\macroname{\textbackslash gre@temp@dimen@one}{gregoriotex-spaces.tex}
Temporary dimension used in calculations.

\macroname{\textbackslash gre@temp@skip@one}{gregoriotex-spaces.tex}
Temporary skip used in calculations.

\macroname{\textbackslash gre@unit}{gregoriotex-spaces.tex}
Temporary dimension used by \verb=\gre@convertto=.

\macroname{\textbackslash gre@base}{gregoriotex-spaces.tex}
Temporary dimension used by \verb=\gre@convertto=.

\macroname{\textbackslash gre@maxlen}{gregoriotex-spaces.tex}
Distance holding the maximum legal length in TeX.



\section{Special arguments}

These arguments are used by multiple functions and take a lot of space
to describe so we describe them once here and refer to this section
rather than have multiple definitions.

\subsection{Note Alignment Type}\label{notesalign}
\rowcolors{1}{lightgray}{lightgray}
\begin{tabular}{cp{10.5cm plus .5cm}}
  \multicolumn{2}{c}{Integer with the following possibilities:} \\
  \hline
  0 & one-note glyph or more than two notes glyph except porrectus : here we must put the aligncenter in the middle of the first note\\
  1 & two notes glyph (podatus is considered as a one-note glyph) : here we put the aligncenter in the middle of the glyph\\
  2 & porrectus : has a special align center\\
  3 & initio-debilis : same as 1 but the first note is much smaller\\
  4 & case of a glyph starting with a quilisma\\
  5 & case of a glyph starting with an oriscus\\
  6 & case of a punctum inclinatum\\
  7 & case of a stropha\\
  8 & flexus with an ambitus of one\\
  9 & flexus deminutus\\
  10 & virgula\\
  11 & divisio minima, minor and maior\\
  12 & divisio finalis
 \end{tabular}

%%% Local Variables:
%%% mode: latex
%%% TeX-master: "UserManual"
%%% End:
