\section{Gregoriotex Controls}

These functions are the ones used by Gregoriotex internally as it
process the commands listed above.  They should not appear in any user
document and are listed here for programmer documentation purposes
only.

\macroname{\textbackslash gre@coloredlines\#1}{gregoriotex.sty}
%\verb=\gre@coloredlines#1=%tex.sty
Changes the color of the staff lines to \#1.

\begin{argtable}
  \#1 & string & Must be a named color defined using \verb=\definecolor=.
\end{argtable}

\macroname{\textbackslash gre@redlines}{gregoriotex.sty}
%\verb=\gre@redlines=%tex.sty
Changes the color of the staff lines to \verb=gregoriocolor=.

\macroname{\textbackslash gre@normallines}{gregoriotex.sty}
%\verb=\gre@normallines=%tex.sty
Resets the formatting of the staff lines.

\macroname{\textbackslash gre@error\#1}{gregoriotex.tex}
%\verb=\gre@error#1=%tex.tex
Raises an error which is identified as coming from GregorioTeX.  Uses
\LaTeX\ \verb=\PackageError= or Plain\TeX\ \verb=\errmessage= as
appropriate.

\begin{argtable}
  \#1 & string & Error message.\\
\end{argtable}
	
\macroname{\textbackslash gre@warn\#1}{gregoriotex.tex}
%\verb=\gre@warn#1=%tex.tex
Raises a warning which is identified as coming from GregorioTeX.  Uses
\LaTeX\ \verb=\PackageWarning= or Plain\TeX\ \verb=\message= as
appropriate.

\begin{argtable}
  \#1 & string & Warning message.\\
\end{argtable}
	
\macroname{\textbackslash gre@localleftbox}{gregoriotex.tex}
%\verb=\gre@localleftbox=%tex.tex
Alias for \verb=\luatexlocalleftbox=.  Used to make propagating
changes in latex easier.

\macroname{\textbackslash gre@localrightbox}{gregoriotex.tex}
%\verb=\gre@localrightbox=%tex.tex
Alias for \verb=\luatexlocalrightbox=.  Used to make propagating
changes in latex easier.
	
\macroname{\textbackslash gre@unsetattribute}{gregoriotex.tex}
%\verb=\gre@unsetattribute=%tex.tex
Alias for \verb=\unsetlutexattribute{\gregorioattr}=

\macroname{\textbackslash gregorioattr}{gregoriotex.tex}
%\verb=\gregorioattr=%tex.tex
A luatex attribute we put on the text nodes.\\
If it is 1, it means that there may be a dash here if this syllable is at the end of a line.\\
If it is 2, it means that it's never useful to typeset a dash.\\
If it is 0, it just means that we are in a score.\\

\macroname{\textbackslash gregoriocenterattr}{gregoriotex.tex}
%\verb=\gregoriocenterattr=%tex.tex
A luatex attribute used for translation centering.

\macroname{\textbackslash gregoriotexversion\#1}{gregoriotex.tex}
%\verb=\gregoriotexversion#1=%tex.tex
Defines the current version of GregorioTeX API.

\begin{argtable}
  \#1 & integer & Date in format: yyyymmdd
\end{argtable}

\macroname{\textbackslash gre@internalversion\#1}{gregoriotex.tex}
%\verb=\gre@internalversion#1=%tex.tex
The version of GregorioTeX.

\begin{argtable}
  \#1 & integer & Date in format: yyyymmdd
\end{argtable}

\macroname{\textbackslash gre@declarefileversion\#1\#2}{gregoriotex.tex}
%\verb=\gre@declarefileversion#1#2=%tex.tex
Checks to see if GregorioTeX component \#1 with version \#2 is
compatible with \verb=\gre@internalversion=

\begin{argtable}
  \#1 & string  & Name\\
  \#2 & integer & Date in format yyyymmdd\\
\end{argtable}

\macroname{\textbackslash gre@factor}{gregoriotex.tex}
%\verb=\gre@factor=%tex.tex
Count representing the size of the staff.  Initialized to 0, but
changed to 17 by \verb=\begingregorianscore= if user hasn't changed
it.
	
\macroname{\textbackslash gre@stafflinewidth}{gregoriotex.tex}
%\verb=\gre@stafflinewidth=%tex.tex
Dimension representing the width of a line of staff.  Can vary, for
example, at the first line.

\macroname{\textbackslash gre@linewidth}{gregoriotex.tex}
%\verb=\gre@linewidth=%tex.tex
Dimension representing the width of the score (including initial).

\macroname{\textbackslash gre@calculateconstantglyphraise}{gregoriotex.tex}
%\verb=\gre@calculateconstantglyphraise=%tex.tex
Macro to caluclate \verb=\gre@constantglyphraise=

\macroname{\textbackslash gre@constantglyphraise}{gregoriotex.tex}
%\verb=\gre@constantglyphraise=%tex.tex
Dimension representing ??

\macroname{\textbackslash gre@currenttranslationheight}{gregoriotex.tex}
%\verb=\gre@currenttranslationheight=%tex.tex
Dimension representing the space for the translation beneath the text.

\macroname{\textbackslash gre@addtranslationspace}{gregoriotex.tex}
%\verb=\gre@addtranslationspace=%tex.tex
Macro to tell Gregorio to set space for the translation.

\macroname{\textbackslash gre@removetranslationspace}{gregoriotex.tex}
%\verb=\gre@removetranslationspace=%tex.tex
Macro to tell Gregorio to remove the space allocated to the translation.

\macroname{\textbackslash gre@kernbeforeeol}{gregoriotex.tex}
%\verb=\gre@kernbeforeeol=%tex.tex
Macro describing a kern to make before ending the line, which we
sometimes want (see \verb=\gresyllable=).

\macroname{\textbackslash gre@newlinecommon\#1\#2\#3\#4\#5}{gregoriotex.tex}
%\verb=\gre@newlinecommon#1#2#3#4#5=%tex.tex
Macro we call each time we force a changing of line, it automatically
sets \verb=\greknownline=, and adjusts left spaces.

\begin{argtable}
  \#1 & 0 & No note above staff.\\
      & 1 & Note above 4th line (position k).\\
      & 2 & Note on 5th line (position l).\\
      & 3 & Note above 5th line (position m).\\
  \#2 & 0 & No note below staff.\\
      & 1 & Note below 1st line (position c).\\
      & 2 & Note on 0th line (position b).\\
      & 3 & Note below 0th line (position a).\\
      & 4 & Note below 0th line (position a) with vertical episemus.\\
  \#3 & 0 & No translation.\\
      & 1 & Translation present.\\
  \#4 & 0 & Justify line.\\
      & 1 & Do not justify line.\\
  \#5 & 0 & No extra space above staff.\\
      & 1 & Extra space above staff.\\
\end{argtable}

\macroname{\textbackslash gre@additionalbottomspace}{gregoriotex.tex}
%\verb=\gre@additionalbottomspace=%tex.tex
Dimension representing extra space below the staff needed for low notes.

\macroname{\textbackslash gre@additionaltopspace}{gregoriotex.tex}
%\verb=\gre@additionaltopspace=%tex.tex
Dimension representing extra space above the staff needed for high notes.

\macroname{\textbackslash gre@updateadditionalspaces\#1\#2}{gregoriotex.tex}
%\verb=\gre@updateadditionalspaces#1#2=%tex.tex
Macro which updates \verb=\gre@additionalbottomspace= and\\
\verb=\gre@additionaltopspace=

%	\#1 --- integer: 0 = no note above staff; 1 = note above 4th line (position k); 2 = note on 5th line (position l); 3 = note above 5th line (position m)
%	\#2 --- integer: 0 = no note below staff; 1 note below 1st line (position c); 2 = note on 0th line (position b); 3 = note below 0th line (position a); 4 = note below 0th line (position a) with vertical episemus

\macroname{\textbackslash gre@textlower}{gregoriotex.tex}
%\verb=\gre@textlower=%tex.tex
Dimension representing the height of the separation between the 0th
line (which is invisible except for notes in the a or b position) and
the bottom of the text.

\macroname{\textbackslash gre@Tempwidth}{gregoriotex.tex}
%\verb=\gre@Tempwidth=%tex.tex
Box used to calculate \verb=\gre@tempwidth=
	
\macroname{\textbackslash gre@tempwidth}{gregoriotex.tex}
%\verb=\gre@tempwidth=%tex.tex
Dimension representing width of some element.

\macroname{\textbackslash gre@widthof\#1}{gregoriotex.tex}
%\verb=\gre@widthof#1=%tex.tex
Macro which calculates \verb=\gre@tempwidth= as width of \#1.

\macroname{\textbackslash gre@textaligncenter}{gregoriotex.tex}
%\verb=\gre@textaligncenter=%tex.tex
Dimension representing the width from the beginning of the letters in
a syllable to the middle of the middle letters.  Used for lining up
neumes and syllables.
	
\macroname{\textbackslash gre@findtextaligncenter\#1\#2\#3}{gregoriotex.tex}
%\verb=\gre@findtextaligncenter#1#2#3=%tex.tex
Macro for calculating \verb=\gre@textaligncenter=.

\begin{argtable}
  \#1 & string & The first part of the syllable (any preceding consonants in Latin).\\
  \#2 & string & The middle part of the syllable (the vowel in Latin).\\
  \#3 & 0 & Calculation is being performed for the current syllable.\\
      & 1 & Calculation is being performed for the next syllable.\\
\end{argtable}

\macroname{\textbackslash gre@additionalleftspace}{gregoriotex.tex}
%\verb=\gre@additionalleftspace=%tex.tex
Dimension representing the additional space that has to be added to
the localleftbox for a big initial (one taking two lines).

\macroname{\textbackslash Gre@Initial}{gregoriotex.tex}
%\verb=\Gre@Initial=%tex.tex
Box containing the initial.

\macroname{\textbackslash gre@initialwidth}{gregoriotex.tex}
%\verb=\gre@initialwidth=%tex.tex
Dimension representing the width of the initial (and the space after).

\macroname{\textbackslash gre@biginitial}{gregoriotex.tex}
%\verb=\gre@biginitial=%tex.tex
Count indicating whether initial takes 2 lines: 0 = it doesn't; 1 = it
does.

\macroname{\textbackslash gre@knowline}{gregoriotex.tex}
%\verb=\gre@knowline=%tex.tex
Line gregoriotex thinks is being set.

\macroname{\textbackslash gre@updateleftbox}{gregoriotex.tex}
%\verb=\gre@updateleftbox=%tex.tex
Macro which adjusts size of line and clef placement after initial.

\macroname{\textbackslash gre@updatelinewidth}{gregoriotex.tex}
%\verb=\gre@updatelinewidth=%tex.tex
Macro which adjusts the width of the line.

\macroname{\textbackslash gre@initialformat\#1}{gregoriotex.tex}
%\verb=\gre@initialformat#1=%tex.tex
Macro which applies formatting in \verb=initialformat= environment to
initial.

\begin{argtable}
  \#1 & character & The initial letter.\\
\end{argtable}

\macroname{\textbackslash gre@biginitialformat\#1}{gregoriotex.tex}
%\verb=\gre@biginitialformat#1=%tex.tex
Macro which applies formatting in \verb=biginitialformat= environment
to initial.

\begin{argtable}
  \#1 & character & The big initial letter.\\
\end{argtable}

\macroname{\textbackslash Gre@Aboveinitialfirstbox}{gregoriotex.tex}
%\verb=\Gre@Aboveinitialfirstbox=%tex.tex
The box which contains the first (top) annotation above the initial.

\macroname{\textbackslash gre@aboveinitialfirstraise}{gregoriotex.tex}
%\verb=\gre@aboveinitialfirstraise=%tex.tex
Dimension representing the space allocated to the first annotation.

\macroname{\textbackslash Gre@Aboveinitialsecondbox}{gregoriotex.tex}
%\verb=\Gre@Aboveinitialsecondbox=%tex.tex
The box which contains the second (bottom) annotation above the initial.

\macroname{\textbackslash gre@aboveinitialsecondraise}{gregoriotex.tex}
%\verb=\gre@aboveinitialsecondraise=%tex.tex
Dimension representing the space allocated to the second annotation.

\macroname{\textbackslash gre@setaboveinitialrais}{gregoriotex.tex}
%\verb=\gre@setaboveinitialrais=%tex.tex
Macro to give \verb=\gre@aboveinitialfirstraise= and\\
\verb=\gre@aboveinitialsecondraise= their working values.
	
\macroname{\textbackslash gre@currentabovelinestextheight}{gregoriotex.tex}
%\verb=\gre@currentabovelinestextheight=%tex.tex
Dimension representing the space allocated above the lines for text.	

\macroname{\textbackslash gre@abovelinestextstyle\#1}{gregoriotex.tex}
%\verb=\gre@abovelinestextstyle#1=%tex.tex
Macro to apply the above line text style to argument (\#1, a character).

\macroname{\textbackslash gre@addspaceabove}{gregoriotex.tex}
%\verb=\gre@addspaceabove=%tex.tex
Macro to allocate space for text placed above the staff lines.

\macroname{\textbackslash gre@removespaceabove}{gregoriotex.tex}
%\verb=\gre@removespaceabove=%tex.tex
Macro to unallocate space for text above the staff lines.

\macroname{\textbackslash gre@currenttextabovelines\#1}{gregoriotex.tex}
%\verb=\gre@currenttextabovelines#1=%tex.tex
Macro containing the text (\#1) which is currently being placed above
the staff lines.

\macroname{\textbackslash gre@typsettextabovelines\#1}{gregoriotex.tex}
%\verb=\gre@typsettextabovelines#1=%tex.tex
Macro to typeset argument (\#1, a string) above the staff lines.

\macroname{\textbackslash gre@removelinescount}{gregoriotex.tex}
%\verb=\gre@removelinescount=%tex.tex
Boolean indicating whether staff lines should be printed.  Has value 0
when lines are printed, 1 when they are not.

\macroname{\textbackslash gre@drawfirstlines}{gregoriotex.tex}
%\verb=\gre@drawfirstlines=%tex.tex
Macro to draw the staff lines for the first line of the score (\ie it
accounts for the space taken up by the initial).

\macroname{\textbackslash Gre@Lines}{gregoriotex.tex}
%\verb=\Gre@Lines=%tex.tex
Box to contain the staff lines for lines other than the first line.

\macroname{\textbackslash gre@generatelines}{gregoriotex.tex}
%\verb=\gre@generatelines=%tex.tex
Macro to fill \verb=\Gre@Lines=.

\macroname{\textbackslash gre@smallsecondline}{gregoriotex.tex}
%\verb=\gre@smallsecondline=%tex.tex
Macro called when the initial is big to make the second set of staff
lines the same length as the first.

\macroname{\textbackslash gre@normallines}{gregoriotex.tex}
%\verb=\gre@normallines=%tex.tex
Marco called after the second set of staff lines when the initial is
big to go back to normal width lines.

\macroname{\textbackslash gre@translationcenteringscheme}{gregoriotex.tex}
%\verb=\gre@translationcenteringscheme=%tex.tex
Boolean to indicate the centering scheme used for the translation.\\
Possible values:\\
0=translation is left aligned with corresponding text.\\
1=translation is centered with the corresponding text.

\macroname{\textbackslash gre@nlbintranslation}{gregoriotex.tex}
%\verb=\gre@nlbintranslation=%tex.tex
Variable used to indicate whether line breaks are allowed in the translation.\\
Possible values: 0=line breaks are allowed; 1=line breaks are prohibited

\macroname{\textbackslash gre@translationformat\#1}{gregoriotex.tex}
%\verb=\gre@translationformat#1=%tex.tex (is this defined somewhere else too?)
Macro to apply the translation format to string. (is this defined somewhere else too?)

\macroname{\textbackslash gre@mustdotranslationcenterend}{gregoriotex.tex}
%\verb=\gre@mustdotranslationcenterend=%tex.tex
Boolean to indicate if the translation is at the end of a line?\\
Possible values: 0=not at the end of a line; 1=at the end of a line

\macroname{\textbackslash gre@dotranslationcenterend}{gregoriotex.tex}
%\verb=\gre@dotranslationcenterend=%tex.tex
Macro to set things up for a translation at the end of a line.

\macroname{\textbackslash gre@printchirovbars}{gregoriotex.tex}
%\verb=\gre@printchirovbars=%tex.tex
Count that is 1 or 0 if we need to print the small vertical bars in
the chironomic line.

\macroname{\textbackslash gre@nolastline}{gregoriotex.tex}
%\verb=\gre@nolastline=%tex.tex
Macro to call when there is just a little thing that will go to the
last line, when it is not necessary. It doesn't seem to be used, so
it's a good candidate for deprecation!

\macroname{\textbackslash gre@endofword\#1}{gregoriotex.tex}
%\verb=\gre@endofword#1=%tex.tex
Macro called at the end of each word.  Does extra stuff when argument (\#1, an integer) is 1?

\macroname{\textbackslash gre@endbeforebar\#1}{gregoriotex.tex}
%\verb=\gre@endbeforebar#1=%tex.tex
Macro called at end of word when next element is a bar.  Argument (\#1, an
integer) has same function as in \verb=\gre@endofword=.

\macroname{\textbackslash gre@endafterbar\#1}{gregoriotex.tex}
%\verb=\gre@endafterbar#1=%tex.tex
Macro called after a bar.  Argument (\#1, an integer) has same function as
in \verb=\gre@endofword=.

\macroname{\textbackslash gre@lastoflinecount}{gregoriotex.tex}
%\verb=\gre@lastoflinecount=%tex.tex
Count which marks the last syllable of the line.\\
Possible values: 0=nothing; 1=last syllable of line; 2=first syllable of line

\macroname{\textbackslash gre@disableeolshifts}{gregoriotex.tex}
%\verb=\gre@disableeolshifts=%tex.tex
Boolean which indicates that the syllable should be shifted left a bit.\\
Possible values: 0=shift happens; 1=shift doesn't happen.

\macroname{\textbackslash Gre@DisableEOLShifts}{gregoriotex.tex}
%\verb=\Gre@DisableEOLShifts=%tex.tex
Macro to set \verb=\gre@disableeolshifts= to 1.

\macroname{\textbackslash Gre@EnableEOLShifts}{gregoriotex.tex}
%\verb=\Gre@EnableEOLShifts=%tex.tex
Macro to set \verb=\gre@disableeolshifts= to 0.

\macroname{\textbackslash gre@blockcusto}{gregoriotex.tex}
%\verb=\gre@blockcusto=%tex.tex
Count to indicate if the custo should be blocked.\\
Possible values: 0=do not block custo; 1=block custo\\

\macroname{\textbackslash gre@endofsyllable}{gregoriotex.tex}
%\verb=\gre@endofsyllable=%tex.tex
Macro called at the end of a syllable which does not end a word.

\macroname{\textbackslash gre@nlbstate}{gregoriotex.tex}
%\verb=\gre@nlbstate=%tex.tex
Count to indicate the no line break areas.\\
Possible values: 0 = not a no line break area; 1 = no line break due to translation centering; 2 = no line break due to <nlba> tag\\

\macroname{\textbackslash gregoriofontname}{gregoriotex.tex}
%\verb=\gregoriofontname=%tex.tex
Macro which holds the default font name (greciliae).

\macroname{\textbackslash gre@usestylefont}{gregoriotex.tex}
%\verb=\gre@usestylefont=%tex.tex
Count to indicate if gregoriostylefont should be used.\\
Possible values: 0 = do not use; 1 = use

\macroname{\textbackslash gre@setstylefont}{gregoriotex.tex}
%\verb=\gre@setstylefont=%tex.tex
Macro to load greextra as gregoriostylefont at correct size.
	
\macroname{\textbackslash gre@normalstafflinesformat}{gregoriotex.tex}
%\verb=\gre@normalstafflinesformat=%tex.tex
Macro to apply formatting to the normal staff lines.
	
\macroname{\textbackslash gre@additionalstafflinesformat}{gregoriotex.tex}
%\verb=\gre@additionalstafflinesformat=%tex.tex
Macro to apply formatting to the additional staff lines.

\macroname{\textbackslash gre@includetexscore\#1}{gregoriotex.tex}
%\verb=\gre@includetexscore#1=%tex.tex
Macro to include scores in gtex format.

\macroname{\textbackslash gre@includegabcscore\#1}{gregoriotex.tex}
%\verb=\gre@includegabcscore#1=%tex.tex
Macro to include scores in gabc format.

\macroname{\textbackslash gre@usestylecommon}{gregoriotex-signs.tex}
%\verb=\gre@usestylecommon=%tex-signs.tex
Checks to see if gregoriostylefont has been loaded and loads it if it has not.

\macroname{\textbackslash gre@falsepenalty\#1}{gregoriotex-signs.tex}
%\verb=\gre@falsepenalty#1=%tex-signs.tex
Does nothing.

\macroname{\textbackslash gre@truepenalty\#1}{gregoriotex-signs.tex}
%\verb=\gre@truepenalty#1=%tex-signs.tex
Alias for \verb=\penalty#1=.  Used in combination with
\verb=\gre@falsepenalty= to avoid placing \verb=\penalty= where it
isn't allowed.

\macroname{\textbackslash gre@hskip}{gregoriotex-signs.tex}
%\verb=\gre@hskip=%tex-signs.tex
Alias for \verb=\hskip= outside of discretionaries and \verb=\kern=
inside of them.

\macroname{\textbackslash gre@penalty\#1}{gregoriotex-signs.tex}
%\verb=\gre@penalty#1=%tex-signs.tex
Alias for \verb=\gre@truepenalty= outside of discretionaries and\\
\verb=\gre@falsepenalty= inside of them.

\macroname{\textbackslash gre@insidediscretionary}{gregoriotex-signs.tex}
%\verb=\gre@insidediscretionary=%tex-signs.tex
Count for tracking whether we are in a discretionary (1) or not (0).

\macroname{\textbackslash gre@discretionary\#1\#2}{gregoriotex-signs.tex}
%\verb=\gre@discretionary#1#2=%tex-signs.tex
Sets \verb=\gre@hskip=, \verb=\gre@penalty=,
\verb=\gre@insidediscretionary=\\ to their within discretionary values
and then calls \verb=\discretionary#1#2#3=.  \#2 (a string) is a
non-printing string which sets \verb=\gre@lastoflinecount= to 2.
After resolving the \verb=\discretionary= command \verb=\gre@hskip=,
\verb=\gre@penalty=, and\\ \verb=\gre@insidediscretionary= are restored
to their outside values.

\macroname{\textbackslash gre@removeclefcount}{gregoriotex-signs.tex}
%\verb=\gre@removeclefcount=%tex-signs.tex
Count for indicating if the clef should be printed (1) or not (0).

\macroname{\textbackslash gre@clefnum}{gregoriotex-signs.tex}
%\verb=\gre@clefnum=%tex-signs.tex
Count indicated the clef line and pitch: 1- c on bottom line, 2 - c on
second line, 3 - c on third line, 4 - c on top line, 5 - f on bottom
line, 6 - f on second line, 7 - f on third line, 8 - f on top line.

\macroname{\textbackslash gre@clefwidth}{gregoriotex-signs.tex}
%\verb=\gre@clefwidth=%tex-signs.tex
Width of the clef.

\macroname{\textbackslash gre@setlinesclef\#1\#2\#3\#4}{gregoriotex-signs.tex}
%\verb=\gre@setlinesclef#1#2#3#4=%tex-signs.tex
Macro to define the clef that will appear at the beginning of the lines.

\begin{argtable}
  \#1 & character & Type of clef (c or f).\\
  \#2 & character & Height of clef (using gabc height letters).\\
  \#3 & 0         & Do not add a space after the clef.\\
      & 1         & Add a space after the clef.\\
  \#4 & character & Height of flat after clef (using gabc height letters, a = no flat).\\
\end{argtable}

\macroname{\textbackslash gre@calculateclefnum\#1\#2\#3}{gregoriotex-signs.tex}
%\verb=\gre@calculateclefnum#1#2#3=%tex-signs.tex
Macro to calculate the clef number and store it in \verb=\gre@clefnum=.

\begin{argtable}
  \#1 & character & Type of clef (c or f).\\
  \#2 & 1-4       & Line on which the clef appears.\\
  \#3 & character & Height of flat after clef (using gabc height letters, a = no flat).\\
\end{argtable}

\macroname{\textbackslash gre@updatelinesclef}{gregoriotex-signs.tex}
%\verb=\gre@updatelinesclef=%tex-signs.tex
Macro redrawing a key from \verb=\gre@clefnum=.  Useful for vertical
space changes.
	
\macroname{\textbackslash gre@typekey\#1\#2\#3\#4\#5}{gregoriotex-signs.tex}
%\verb=\gre@typekey#1#2#3#4#5=%tex-signs.tex
Macro for typesetting the key signature.

\begin{argtable}
  \#1 & character & Type of clef (c or f).\\
  \#2 & 1-4       & Line of the key.\\
  \#3 & 0         & Do not use small key characters.\\
      & 1         & Use small key characters.\\
  \#4 & 0         & Do not print a space after the key.\\
      & 1         & Print a space after the key.\\
  \#5 & character & Height of flat after clef (using gabc height letters, a = no flat).\\
\end{argtable}

\macroname{\textbackslash gre@inchangeclef\#1\#2\#3}{gregoriotex-signs.tex}
%\verb=\gre@inchangeclef#1#2#3=%tex-signs.tex
Macro for changing key inside a syllable.

\begin{argtable}
  \#1 & character & Type of clef (c or f).\\
  \#2 & 1-4       & Line on which the clef appears.\\
  \#3 & character & Height of flat after clef (using gabc height letters, a = no flat).\\
\end{argtable}

\macroname{\textbackslash gre@setcusto\#1}{gregoriotex-signs.tex}
%\verb=\gre@setcusto#1=%tex-signs.tex
Macro for typesetting custos.

\begin{argtable}
  \#1 & character & Height of custos.\\
\end{argtable}

\macroname{\textbackslash gre@additionaltopcusotolineend}{gregoriotex-signs.tex}
%\verb=\gre@additionaltopcusotolineend=%tex-signs.tex
Macro for typesetting the line behind a high custos at end of line.

\macroname{\textbackslash gre@additionalbottomcusotlineend}{gregoriotex-signs.tex}
%\verb=\gre@additionalbottomcusotlineend=%tex-signs.tex
Macro for typesetting the line behind a low custos at end of line.

\macroname{\textbackslash gre@additionaltopcusotolinemiddle}{gregoriotex-signs.tex}
%\verb=\gre@additionaltopcusotolinemiddle=%tex-signs.tex
Macro for typesetting the line behind a high custos in middle of line.

\macroname{\textbackslash gre@additionalbottomcusotlinemiddle}{gregoriotex-signs.tex}
%\verb=\gre@additionalbottomcusotlinemiddle=%tex-signs.tex
Macro for typesetting the line behind a low custos in middle of line.

\macroname{\textbackslash gre@custocahar\#1}{gregoriotex-signs.tex}
%\verb=\gre@custocahar#1=%tex-signs.tex
Selects the appropriate custos character.

\begin{argtable}
  \#1 & character & Height of the custos.\\
\end{argtable}

\macroname{\textbackslash gre@curlybracechar}{gregoriotex-signs.tex}
%\verb=\gre@curlybracechar=%tex-signs.tex
Macro for the char ‘\{‘ (which is rotated 90 and placed over the
notes)

\macroname{\textbackslash gre@bracechar}{gregoriotex-signs.tex}
%\verb=\gre@bracechar=%tex-signs.tex
Macro for the char ‘(‘ (which is rotated 90 and placed over the notes)

\macroname{\textbackslash gre@resizebox}{gregoriotex-signs.tex}
%\verb=\gre@resizebox=%tex-signs.tex
Local alias for \verb=\resizebox= (currently borrowed from graphics)

\macroname{\textbackslash gre@lowchoralsignstyle\#1}{gregoriotex.sty and tex-signs.tex}
%\verb=\gre@lowchoralsignstyle#1=%tex.sty and tex-signs.tex
Applies the formatting to the low choral signs specified by the
appropriate (pseudo-)environment.

\macroname{\textbackslash gre@highchoralsignstyle\#1}{gregoriotex.sty and tex-signs.tex}
%\verb=\gre@highchoralsignstyle#1=%tex.sty and tex-signs.tex
Applies the formatting to the high choral signs specified by the
appropriate (pseudo-)environment.

\macroname{\textbackslash Gre@Tempsign}{gregoriotex-signs.tex}
%\verb=\Gre@Tempsign=%tex-signs.tex
Box for temporarily holding a sign needed for calculations.

\macroname{\textbackslash gre@tempdimsignwidth}{gregoriotex-signs.tex}
%\verb=\gre@tempdimsignwidth=%tex-signs.tex
Temporary dimension used to calculate placement of signs.

\macroname{\textbackslash gre@vepisemusorrareaux\#1\#2\#3\#4\#5\#6\#7}{gregoriotex-signs.tex}
%\verb=\gre@vepisemusorrareaux#1#2#3#4#5#6#7=%tex-signs.tex
Macro help with typesetting of episemus.

\begin{argtable}
  \#1 & character & A preceding symbol needed for calculating position of episemus.\\
  \#2 & character & Another preceding symbol.\\
  \#3 & 0 & Go back to beginning of glyph.\\
      & 1 & Go back width of \#2.\\
      & 2 & Do back width of \#1 and forward width of \#2.\\
      & 3 & Go to beginning of glyph, forward by width of \#1 and back width of \#2.\\
  \#4 & integer & An additional shift in space to apply to the episemus.\\
  \#5 & 1       & Use ictus arsicus.\\
      & 2       & Use ictus theticus.\\
  \#6 & 1       & Set a vertical episemus.\\
      & 2       & Set a rare sign.\\
      & 3       & Set a choral sign above the bar.\\
  \#7 & integer & If \#6 = 3 then the choral sign.\\
\end{argtable}

\macroname{\textbackslash gre@vepisemusorrare\#1\#2\#3\#4\#5}{gregoriotex-signs.tex}
%\verb=\gre@vepisemusorrare#1#2#3#4#5=%tex-signs.tex
Macro for typesetting vertical episemus or rare accent.

\begin{argtable}
  \#1 & character & Height of episemus.\\
  \#2 & integer   & Type of glyph the episemus is attached to. See \nameref{EpisemusSpecial} argument for description of options.\\
  \#3 & integer   & Glyph number of glyph being typeset.\\
  \#4 & 1         & Set a vertical episemus.\\
      & 2         & Set a rare sign.\\
      & 3         & Set a choral sign above the bar.\\
  \#5 & string    & If \#4 = 3 then the choral sign.\\
\end{argtable}

\macroname{\textbackslash grewriteaux\#1}{gregoriotex.tex}
%\verb=\grewriteaux#1=%tex.tex
Macro for writing to the aux file.

\begin{argtable}
  \#1 & string & What is written to the aux file.\\
\end{argtable}

\macroname{\textbackslash opengreaux}{gregoriotex.tex}
%\verb=\opengreaux=%tex.tex
Macro for opening the aux file.

\macroname{\textbackslash closegreaux}{gregoriotex.tex}
%\verb=\closegreaux=%tex.tex
Macro for closing the aux file.

\macroname{\textbackslash gre@ictus\#1}{gregoriotex-signs.tex}
%\verb=\gre@ictus#1=%tex-signs.tex
Macro for writing the position of an ictus to the aux file.

\begin{argtable}
  \#1 & 1 & Ictus type a was typeset.\\
      & 2 & Ictus type t was typeset.\\
\end{argtable}

\macroname{\textbackslash gre@hepisorlineaux\#1\#2\#3\#4}{gregoriotex-signs.tex}
%\verb=\gre@hepisorlineaux#1#2#3#4=%tex-signs.tex
Macro for helping to typeset a horizontal episemus and additional line.

\begin{argtable}
  \#1 & character & A glyph having the same width as the width between the end of the glyph and the beginning of the episemus.\\
  \#2 & integer   & Glyph number of the episemus.\\
  \#3 & 0         & Go to beginning of glyph.\\
      & int !0    & Do not go to beginning of glyph.\\
  \#4 & 0         & Set horizontal episemus.\\
      & 1         & Set horizontal episemus under a note.\\
      & 2         & Set line at top of staff.\\
      & 3         & Set line at bottom of staff.\\
      & 4         & Set choral sign.\\
\end{argtable}

\macroname{\textbackslash gre@hepisorline\#1\#2\#3\#4}{gregoriotex-signs.tex}
%\verb=\gre@hepisorline#1#2#3#4=%tex-signs.tex
Macro for typesetting a horizontal line (additional line or episemus).

\begin{argtable}
  \#1 & character & Height of the episemus.\\
  \#2 & special   & See \nameref{EpisemusSpecial}.\\
  \#3 & integer   & The ambitus of the porrectus or porrectus flexes if the second argument is 9, 10, 11, 21, 22, 23; ignored otherwise.\\
  \#4 & 0         & Set horizontal episemus.\\
      & 1         & Set horizontal episemus under a note.\\
      & 2         & Set line at top of staff.\\
      & 3         & Set line at bottom of staff.\\
      & 4         & Set choral sign.\\
\end{argtable}

\macroname{\textbackslash gre@AddHEpisemusBridges}{gregoriotex-signs.tex}
%\verb=\gre@AddHEpisemusBridges=%tex-signs.tex
Macro to activate bridging horizontal episemus.
	
\macroname{\textbackslash gre@RemoveHEpisemusBridges}{gregoriotex-signs.tex}
%\verb=\gre@RemoveHEpisemusBridges=%tex-signs.tex
Macro to suspend bridging horizontal episemus.

\macroname{\textbackslash gre@writebar\#1\#2\#3}{gregoriotex-signs.tex}
%\verb=\gre@writebar#1#2#3=%tex-signs.tex
Macro to typeset a bar line.

\begin{argtable}
  \#1 & 0 & Virgula bar line.\\
      & 1 & Minima bar line.\\
      & 2 & Minor bar line.\\
      & 3 & Maior bar line.\\
      & 4 & Finalis bar line.\\
      & 5 & Last finalis bar line.\\
      & 6 & Dominican bar 1.\\
      & 7 & Dominican bar 2.\\
      & 8 & Dominican bar 3.\\
      & 9 & Dominican bar 4.\\
      & 10& Dominican bar 5.\\
      & 11& Dominican bar 6.\\
  \#2 & 0 & The bar does not appear within a syllable.\\
      & 1 & The bar appears within a syllable.\\
  \#3 & code & Macros which may happen before the skip but after the bar (typically \verb=\grevepisemus=).\\
\end{argtable}

\macroname{gre@tempdimtwo}{gregoriotex-signs.tex}
%\verb=\gre@tempdimtwo=%tex-signs.tex
A temporary dimension used in calculations.

\macroname{\textbackslash gre@divisiomaiorsymbol}{gregoriotex-signs.tex}
%\verb=\gre@divisiomaiorsymbol=%tex-signs.tex
Macro defining the divisio maior symbol.

\macroname{\textbackslash gre@divisiofinalissymbol}{gregoriotex-signs.tex}
%\verb=\gre@divisiofinalissymbol=%tex-signs.tex
Macro defining the divisio finalis symbol.

\macroname{\textbackslash keeprightbox}{gregoriotex-signs.tex}
%\verb=\keeprightbox=%tex-signs.tex
A count to tell if we have to keep the localrightbox until the end.

\macroname{\textbackslash gre@hidepclines}{gregoriotex-signs.tex}
%\verb=\gre@hidepclines=%tex-signs.tex
An integer indicating whether the lines behind a punctum cavum are
hidden (1) or not (0).

\macroname{\textbackslash gre@hidealtlines}{gregoriotex-signs.tex}
%\verb=\gre@hidealtlines=%tex-signs.tex
An integer indicating whether the lines behind an alteration are
hidden (1) or not (0).

\macroname{\textbackslash gre@fillhole\#1}{gregoriotex-signs.tex}
%\verb=\gre@fillhole#1=%tex-signs.tex
Character to fill a hole in a character with the color
grebackgroundcolor.

\begin{argtable}
  \#1 & character & The shape of the hole.\\
\end{argtable}

\macroname{\textbackslash gre@firstisalteration}{gregoriotex-signs.tex}
%\verb=\gre@firstisalteration=%tex-signs.tex
Count to track if the first glyph is an alteration.

\macroname{\textbackslash gre@alteration\#1\#2\#3\#4}{gregoriotex-signs.tex}
%\verb=\gre@alteration#1#2#3#4=%tex-signs.tex
Macro to typeset an alteration (sharp, flat, or natural).

\begin{argtable}
  \#1 & character & The height of the alteration.\\
  \#2 & character & The alteration itself.\\
  \#3 & character & The character of the hole for the alteration.\\
  \#4 & 0         & No flat for a key change.\\
      & 1         & Indicates a flat for a key change.\\
\end{argtable}

\section{Special arguments}

These arguments are used by multiple functions and take a lot of space
to describe so we describe them once here and refer to this section
rather than have multiple definitions.

\subsection{Episemus Special}\label{EpisemusSpecial}
\rowcolors{1}{lightgray}{lightgray}
\begin{tabular}{cp{10.5cm plus .5cm}}
  \multicolumn{2}{c}{Integer with the following possibilities:} \\
  \hline
  0 & Last note, which is a standard punctum (works with pes).\\
  1 & Same, but the last note is a deminutus.\\
  2 & The note before the last note, which is a standard punctum.\\
  3 & Idem, but the note is the note preceding a deminutus.\\
  4 & The note before the note before the last note (for porrectus flexus).\\
  5 & Idem, but when the two last notes are a deminutus.\\
  6 & The first note, if it is a standard punctum.\\
  7 & The first note, if it is an initio debilis.\\
  8 & The first note, if it is a porrectus.\\
  \multicolumn{2}{l}{The three next arguments make no sense for a vepisemus}\\
  9 & The two first notes, if it is a porrectus.\\
  10 & The two first notes, if it is a porrectus flexus.\\
  11 & The notes two and three of a torculus resupinus.\\
  12 & The last note, if it is a punctum inclinatum.\\
  13 & Idem, if it is a punctum inclinatum deminutus.\\
  14 & Idem, if it is a stropha.\\
  15 & Idem, with a quilisma.\\
  16 & Idem, with an oriscus.\\
  17 & Same of 2 but for ambitus of one.\\
  18 & Same of 0, but the last note is a smaller punctum (concerning simple podatus, podatus, and torculus resupinus).\\
  19 & The first note, if it is an oriscus.\\
  20 & The first note, if it is a quilisma.\\
  21 & The second note of a torculus resupinus with first ambitus of at least two.\\
  22 & Idem with ambitus of one.\\
  23 & Idem with initio debilis.\\
  24 & The last note, if it is a linea punctum (or linea punctum cavum).\\
  25 & The last note, if it is a bar.\\
  26 & The last note, if it is a virgula.\\
  27 & The last note, if it is a divisio finalis.\\
\end{tabular}

%%% Local Variables:
%%% mode: latex
%%% TeX-master: "UserManual"
%%% End:
