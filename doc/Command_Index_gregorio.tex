\section{Gregorio Controls}

These functions are the ones written by Gregorio to the gtex file.  While one could, in theory, use/change them to alter the appearance of elements of the score, it is far better to make your changes in the gabc file and let Gregorio make the changes to the gtex file.


\verb=\begingregorioscore=%tex.tex
	Macro to start a score.
\verb=\endgregorioscore=%tex.tex
	Macro to end a score.

\verb=\greaccentus#1#2=%tex-signs.tex
	Macro for typesetting an accentus.
	\#1 --- character: height of episemus
	\#2 --- integer: Type of glyph the episemus is attached to.  See episemus special argument for description of options.
\verb=\greactiveatechironomy=%tex.tex
	Macro called at the beginning of the score to enable chironomy.
\verb=\greadditionalline#1#2#3=%tex-signs.tex
	Macro to typeset the additional line above or below the staff.
	\#1 --- special: See Episemus special
	\#2 --- integer: The ambitus of the porrectus or porrectus flexes if the first argument is 9, 10, 11, 21, 22, 23; ignored otherwise
	\#3 --- integer: set horizontal episemus (0), horizontal episemus under a note (1), line at top of staff (2), line at bottom of staff (3), choral sign (4)
\verb=\greadjustsecondline=%didn't actually find this one in gregoriotex-write.c%tex.tex
	Macro to call before first syllable, but after \verb=\gresetinitialclef=
\verb=\greadjustthirdline=%tex.tex
	Macro to call during the second line.
\verb=\greaugmentumduplex#1#2#3=%tex-signs.tex
	Macro for typesetting an augmentum duplex (a pair of punctum mora)
	\#1 --- character: height for first punctum mora
	\#2 --- character: height for second punctum mora
	\#3 --- integer: First punctum mora occurs before last note of a podatus, prorectus, or toculus resupinus (1), or not (0)

\verb=\grebarbrace#1=%tex-signs.tex
	Macro for typesetting a bar brace.
	\#1 --- integer: Type of glyph the episemus is attached to.  See episemus special argument for description of options.
\verb=\grebarsyllable=
\verb=\grebarvepisemus#1=%tex-signs.tex
	Macro to typeset a vertical episemus around a bar.
	\#1 --- integer: Type of glyph the episemus is attached to.  See episemus special argument for description of options.
\verb=\grebarvepisemusictusa#1=%tex-signs.tex
	Macro to typeset a vertical episemus and ictus (type a) around a bar.
	\#1 --- integer: Type of glyph the episemus is attached to.  See episemus special argument for description of options.
\verb=\grebarvepisemusictust#1=%tex-signs.tex
	Macro to typeset a vertical episemus and ictus (type t) around a bar.
	\#1 --- integer: Type of glyph the episemus is attached to.  See episemus special argument for description of options.
\verb=\grebeginnlbarea#1#2=%tex.tex
	Macro called at beginning of a no line break area.
	\#1 --- integer: 0 = not in the neumes; 1 = in the neumes
	\#2 --- integer: 0 = call didn't come from translation centering; 1 = call came from translation centering.
\verb=\grebeginnotes=%tex.tex
	Macro to draw the staff lines.  Comes after the initial but before the clef.
\verb=\greboldfont#1=%tex.sty, PlainTeX version in tex.tex
	Makes argument (a string) bold.  Accesses \LaTeX \verb=\textbf= or PlainTeX \verb=\bf= as appropriate.  Corresponds to ``<b></b>'' tags in gabc.

\verb=\grechangeclef#1#2#3#4=%tex-signs.tex
	Macro called when key changes
	\#1 --- character: type of new clef (c or f)
	\#2 --- integer: line of new clef
	\#3 --- integer: Print space before clef (1), or not (0)
	\#4 --- character: Height of flat in key (‘a’ for no flat)
\verb=\grecirculus#1#2=%tex-signs.tex
	Macro for typesetting a circulus.
	\#1 --- character: height of circulus
	\#2 --- integer: Type of glyph the circulus is attached to.  See episemus special argument for description of options.
\verb=\grecolored#1=%tex.sty, PlainTeX version in tex.tex
	Colors argument (a string) in \verb=gregoriocolor.=  Corresponds to ``<c></c>'' tags in gabc.  Does nothing in PlainTeX
\verb=\grecusto#1=
	Typesets a custo.
	\#1 --- character: Height of custo

\verb=\gredagger=
\verb=\grediscretionary{}=
\verb=\gredivisiofinalis#1=%tex-signs.tex
	Macro to typeset a divisio finalis.
	\#1 --- code: Macros which may happen before the skip but after the divisio finalis (typically \verb=\grevepisemus=)
\verb=\gredivisiomaior#1=%tex-signs.tex
	Macro to typeset a divisio maior.
	\#1 --- code: Macros which may happen before the skip but after the divisio maior (typically \verb=\grevepisemus=)
\verb=\gredivisiominima#1=%tex-signs.tex
	Macro to typeset a divisio minima.
	\#1 --- code: Macros which may happen before the skip but after the divisio minima (typically \verb=\grevepisemus=)
\verb=\gredivisiominor#1=%tex-signs.tex
	Macro to typeset a divisio minor.
	\#1 --- code: Macros which may happen before the skip but after the divisio minor (typically \verb=\grevepisemus=)
\verb=\gredominica#1#2=%tex-signs.tex
	Macro to typeset a dominican bar.
	\#1 --- integer: Type of dominican bar. 6--11 corresponds to types 1--6
	\#2 --- code: Macros which may happen before the skip but after the divisio minor (typically \verb=\grevepisemus=)

\verb=\greendnlbarea#1#2=%tex.tex
	Macro called at beginning of a no line break area.
	\#1 --- integer: 0 = not in the neumes; 1 = in the neumes
	\#2 --- integer: 0 = call didn't come from translation centering; 1 = call came from translation centering.
\verb=\greendofelement#1#2=%tex.tex
	Macro to end elements.
	\#1 --- integer: 0 = default space; 1 = larger space; 2 = glyph space; 3 = zero-width space
	\#2 --- integer: 0 = space is breakable; 1 = space is unbreakable
\verb=\greendofglyph#1=%tex.tex
	Macro to end a glyph without ending the element.
	\#1 --- integer: 0 = default space; 1 = zero-width space; 2 = space between flat or natural and a note; 3 = space between two puncta inclinata; 4 = space between bivirga or trivirga; 5 = space between bistropha or tristropha; 6 = space after a punctum mora XXX: not used yet, not so sure it is a good idea...; 7 = space between a punctum inclinatum and a punctum inclinatum debilis; 8 = space between two puncta inclinata debilis; 9 = space before a punctum (or something else) and a punctum inclinatum; 10 = space between puncta inclinata (also debilis for now), larger ambitus (range=3rd).; 11 = space between puncta inclinata (also debilis for now), larger ambitus (range=4th or more)

\verb=\grefinaldivisiofinalis#1=%tex-signs.tex
	Macro to end a line with a divisio finalis.
	\#1 --- integer: There is something which needs to be placed after the divisio finalis (1), or not (0).
\verb=\grefinaldivisiomaior#1=%tex-signs.tex
	Macro to end a line with a divisio maior.
	\#1 --- integer: There is something which needs to be placed after the divisio maior (1), or not (0).
\verb=\grefirstlinebottomspace#1#2=%tex.tex
	Macro for additional bottom space for the first line
	\#1 --- integer: 0 = no note below staff; 1 = note below 1st line (position c); 2 = note on 0th line (position b); 3 = note below 0th line (position a); 4 = note below 0th line (position a) with vertical episemus
	\#2 --- integer: 0 = no translation; 1 = translation present
\verb=\greflat#1#2=%tex-signs.tex
	Macro to typeset a flat.
	\#1 --- character: Height of the flat.
	\#2 --- integer: Indicates the a flat for a key change (1), or not (0)

\verb=\greglyph=
\verb=\gregorianmode#1=%tex.tex
	If the gabc file contains a mode in the header, then this function places said mode as the first (top) annotation.  This function effectively disables \verb=\setfirstlineaboveinitial=.  \#1 is an integer from 1 to 8.  Other values are ignored (and \verb=\setfirstlineaboveinitial= should still work).
	Bug: This macro needs to appear before \verb=\greinitial= in the gtex file but \verb=gregoriotex-write.c= places it after.
\verb=\gregorioapiverstion#1=%tex.tex
	Checks to see if GregorioTeX API is version specified by argument (and therefore compatible with the score.  \#1 is date in format: yyyymmdd

\verb=\grehepisemus#1#2#3#4=%tex-signs.tex
	Macro to typeset a episemus.
	\#1 --- character: Height of the episemus
	\#2 --- special: See Episemus special
	\#3 --- integer: The ambitus of the porrectus or porrectus flexes if the second argument is 9, 10, 11, 21, 22, 23; ignored otherwise
	\#4 --- character: replacement for \#1 if a bridge causes a height substitution	
\verb=\grehepisemusbottom#1#2#3=%tex-signs.tex
	Macro to typeset a episemus at the bottom of a note
	\#1 --- character: Height of the episemus
	\#2 --- special: See Episemus special
	\#3 --- integer: The ambitus of the porrectus or porrectus flexes if the second argument is 9, 10, 11, 21, 22, 23; ignored otherwise
\verb=\grehepisemusbridge#1#2#3=%tex-signs.tex
	Macro to typeset a bridge episemus father the last note of a glyph (element, syllable) if the next episemus is at the same height.
	\#1 --- character: Height of the episemus
	\#2 --- special: See Episemus special
	\#3 --- integer: The ambitus of the porrectus or porrectus flexes if the second argument is 9, 10, 11, 21, 22, 23; ignored otherwise
\verb=\grehighchoralsign#1#2#3=%tex-signs.tex
	Macro for typesetting high choral signs.
	\#1 --- character: height of the sign
	\#2 --- string: the choral sign
	\#3 --- integer: choral sign occurs before last note of podatus, porrectus, or torculus resupinus (1), or not (0)
\verb=\grehyph=%tex.tex
	Macro used for end of line hyphens.  Defaults to \verb=\grenormalhyph=.

\verb=\greictusa#1=%tex-signs.tex
	Macro for typesetting an ictus (type a)
	\#1 --- integer: Type of glyph the ictus is attached to.  See episemus special argument for description of options.
\verb=\greictust#1=%tex-signs.tex
	Macro for typesetting an ictus (type t)
	\#1 --- integer: Type of glyph the ictus is attached to.  See episemus special argument for description of options.
	
\verb=\grein=
\verb=\greindivisiofinalis#1=%tex-signs.tex
	Macro to typeset a divisio finalis inside a syllable.
	\#1 --- code: Macros which may happen before the skip but after the divisio finalis (typically \verb=\grevepisemus=)
\verb=\greindivisiomaior#1=%tex-signs.tex
	Macro to typeset a divisio maior inside a syllable.
	\#1 --- code: Macros which may happen before the skip but after the divisio maior (typically \verb=\grevepisemus=)
\verb=\greindivisiominima#1=%tex-signs.tex
	Macro to typeset a divisio minima inside a syllable.
	\#1 --- code: Macros which may happen before the skip but after the divisio minima (typically \verb=\grevepisemus=)
\verb=\greindivisiominor#1=%tex-signs.tex
	Macro to typeset a divisio minor inside a syllable.
	\#1 --- code: Macros which may happen before the skip but after the divisio minor (typically \verb=\grevepisemus=)
\verb=\greindominica#1#2=%tex-signs.tex
	Macro to typeset a dominican bar inside a syllable.
	\#1 --- integer: Type of dominican bar. 6--11 corresponds to types 1--6
	\#2 --- code: Macros which may happen before the skip but after the divisio minor (typically \verb=\grevepisemus=)
\verb=\greinitial#1=%tex.tex
	Macro to set the initial (\#1, a character) in the score.
\verb=\greinsertchiroline=%tex.tex
	Macro called at the beginning of a line to insert the chironomic signs.
\verb=\greinvirgula#1=%tex-signs.tex
	Macro to typeset a virgula inside a syllable.
	\#1 --- code: Macros which may happen before the skip but after the virgula (typically \verb=\grevepisemus=)
\verb=\greitalic#1=%tex.sty, PlainTeX version in tex.tex
	Makes argument (a string) italic.  Accesses \LaTeX \verb=\textit= or PlainTeX \verb=\it= as appropriate.  Corresponds to ``<i></i>'' tags in gabc.

\verb=\grelastofline=%tex.tex
	Macro to set \verb=\gre@lastoflinecount= to 1 (i.e. mark that this syllable is the last of the line).
\verb=\grelastofscore=%tex.tex
	Macro to mark the syllable as the last of the score.
\verb=\grelinea#1#2#3=%tex-signs.tex
	Macho for typesetting a linea.
	\#1 --- : %argument 2 from greglyph
	\#2 --- : %argument 3 from greglyph
	\#3 --- : %argument 4 from greglyph
\verb=\grelineapunctumcavum=
	Macro to typeset a linea punctum cavum.
	\#1 --- %argument 2 from greglyph
	\#2 --- %argument 3 from greglyph
	\#3 --- %argument 4 from greglyph
	\#4 --- code: Code executed before the punctum cavum is written.
	\#5 --- %argument 5 from greglyph
\verb=\grelowchoralsign#1#2#3=%tex-signs.tex
	Macro for typesetting low choral signs.
	\#1 --- character: height of the sign
	\#2 --- string: the choral sign
	\#3 --- integer: choral sign occurs before last note of podatus, porrectus, or torculus resupinus (1), or not (0)

\verb=\grenatural#1#2=%tex-signs.tex
	Macro to typeset a natural.
	\#1 --- character: Height of the flat.
	\#2 --- integer: Indicates the a flat for a key change (1), or not (0)
\verb=\grenewline=%tex.tex
	Macro to call if you want to go to the next line simply.
\verb=\grenewlinewithspace#1#2#3#4=%tex.tex
	Macro called to go to the next line but when there are additional vertical spaces to add
	\#1 --- integer: 0 = no note above staff; 1 = note above 4th line (position k); 2 = note on 5th line (position l); 3 = note above 5th line (position m)
	\#2 --- integer: 0 = no note below staff; 1 note below 1st line (position c); 2 = note on 0th line (position b); 3 = note below 0th line (position a); 4 = note below 0th line (position a) with vertical episemus
	\#3 --- integer: 0 = no translation; 1 = translation present
	\#4 --- integer: 0 = no extra space above staff; 1 = extra space above staff
\verb=\grenewparline=%tex.tex
	Same as \verb=\grenewline= except line is not justified.
\verb=\grenewparlinewithspace#1#2#3#4=%tex.tex
	Same as \verb=\grenewlinewithspace= except line is not justified.
\verb=\grenormalhyph=%tex.tex, not actually found in gregoriotex-write.c
	Macro to typeset a normal hyphen.
\verb=\grenoinitial=%tex.tex
	Macro called when no initial is being set.
\verb=\grenormalinitial=%not actually found in gregoriotex-write.c%tex.tex
	Macro to cancel a 2-line initial.

\verb=\greoverbrace#1#2#3#4=%tex-signs.tex
	Macro to typeset the curved brace over notes.
	\#1 --- dimension: the width
	\#2 --- dimension: the vertical shift
	\#3 --- dimension: the horizontal shift
	\#4 --- integer: Shift to the beginning of the last glyph (1), or not (0)
\verb=\greovercurlybrace#1#2#3#4#5=%tex-signs.tex
	Macro to typeset the curly brace over notes.
	\#1 --- dimension: the width
	\#2 --- integer: Put an accent above (1), or not (0)
	\#3 --- dimension: the vertical shift
	\#4 --- dimension: the horizontal shift
	\#5 --- integer: Shift to the beginning of the last glyph (1), or not (0)

\verb=\grepunctumcavum#1#2#3#4#5=%tex-signs.tex
	Macro to typeset a punctum cavum.
	\#1 --- %argument 2 from greglyph
	\#2 --- %argument 3 from greglyph
	\#3 --- %argument 4 from greglyph
	\#4 --- code: Code executed before the punctum cavum is written.
	\#5 --- %argument 5 from greglyph
\verb=\grepunctummora#1#2#3#4=%tex-signs.tex
	Macro for typesetting punctum mora.
	\#1 --- character: height of punctum mora
	\#2 --- integer: Go back to end of punctum (1), shift left width of 1 punctum (2), or shift left width of 1 punctum and ambitus of 1 (3)
	\#3 --- integer: Punctum mora occurs before last note of podatus, porrectus, or torculus resupinus (1), or not (0)
	\#4 --- integer: Punctum inclinatum (1), or not (0)

\verb=\grereversedaccentus#1#2=%tex-signs.tex
	Macro for typesetting a reversed accentus.
	\#1 --- character: height of accentus
	\#2 --- integer: Type of glyph the accentus is attached to.  See episemus special argument for description of options.
\verb=\grereversedsemicirculus#1#2=%tex-signs.tex
	Macro for typesetting a reversed semicirculus.
	\#1 --- character: height of semicirculus
	\#2 --- integer: Type of glyph the semicirculus is attached to.  See episemus special argument for description of options.

\verb=\grescorereference=%tex.tex
	Currently does nothing.
\verb=\gresemicirculus#1#2=%tex-signs.tex
	Macro for typesetting a semicirculus.
	\#1 --- character: height of semicirculus
	\#2 --- integer: Type of glyph the semicirculus is attached to.  See episemus special argument for description of options.
\verb=\gresetbiginitial=%tex.tex
	Macro which indicates that a 2-line initial is desired.
\verb=\gresetfixednexttextformat=
\verb=\gresetfixedtextformat=
\verb=\gresetinitialclef#1#2#3=%tex-signs.tex
	Macro for writing initial key.
	\#1 --- character: type of clef (c or f)
	\#2 --- integer: line of key (1 to 4)
	\#3 --- character: Height of flat in key (‘a’ for no flat)
\verb=\gresetlinesclef=%#1#2#3#4=%tex.tex
%	possible values for <character>: c, f
%	possible values for <integer2>: 0, 1 (1 = space before clef, 0 = no space)
\verb=\gresettextabovelines#1=
	Macro to place argument (a string) above the lines and empty \verb=\gre@currenttextabovelines= when done.
\verb=\gresharp#1#2=%tex-signs.tex
	Macro to typeset a sharp.
	\#1 --- character: Height of the flat.
	\#2 --- integer: Indicates the a flat for a key change (1), or not (0)
\verb=\gresmallcaps#1=%tex.sty, PlainTeX version in tex.tex
	Makes argument (a string) small capitals.  Accesses \LaTeX \verb=\textsc= or PlainTeX \verb=\sc= as appropriate  Corresponds to ``<sc></sc>'' tags in gabc.
\verb=\grestar=
\verb=\gresyllable=

\verb=\gretilde=%tex.tex
	Macro to print $\sim$.
\verb=\gretranslationcenterend=%tex.tex
	Macro to set \verb=\gre@mustdotranslationcenerend= to 1.
\verb=\grett#1=%tex.sty, PlainTeX version in tex.tex
	Makes argument (a string) typewriter font.  Accesses \LaTeX \verb=\texttt= or PlainTeX \verb=\tt= as appropriate.

\verb=\greul#1=%tex.sty, PlainTeX version in tex.tex
	Makes argument (a string) underlined under \LaTeX using \verb=\underline=.  Does nothing in PlainTeX

\verb=\grevepisemus#1#2=%tex-signs.tex
	Macro for typesetting the vertical episemus.
	\#1 --- character: height of episemus
	\#2 --- integer: Type of glyph the episemus is attached to.  See episemus special argument for description of options.
\verb=\grevepisemusictusa#1#2=%tex-signs.tex
	Macro for typesetting the vertical episemus and ictus (type a) on the same glyph.
	\#1 --- character: height of episemus
	\#2 --- integer: Type of glyph the episemus and ictus are attached to.  See episemus special argument for description of options.
\verb=\grevepisemusictust#1#2=%tex-signs.tex
	Macro for typesetting the vertical episemus and ictus (type t) on the same glyph.
	\#1 --- character: height of episemus
	\#2 --- integer: Type of glyph the episemus and ictus are attached to.  See episemus special argument for description of options.
\verb=\grevirgula#1=%tex-signs.tex
	Macro to typeset a virgula.
	\#1 --- code: Macros which may happen before the skip but after the virgula (typically \verb=\grevepisemus=)

\verb=\grewritetranslation#1=%tex.tex
	Macro to typeset argument (a string) in the translation position.
\verb=\grewritetranslationwithcenterbeginning#1=%tex.tex
	Macro to typeset argument (a string) in the translation position (at the beginning of a line?).

\verb=\grezerhyph=%tex.tex
	Macro to typeset a zero-width hyphen (the hyphen is visible, it is treated as if it had 0 width though.  Used for fine tuning spacing (especially at line endings).

\verb=\setgregoriofont#1=%tex.tex
	Macro to set the font used for the glyphs.
	\#1 --- string: gregorio; parmesan; greciliae; gregoria

%%% Local Variables:
%%% mode: latex
%%% TeX-master: "UserManual"
%%% End:
